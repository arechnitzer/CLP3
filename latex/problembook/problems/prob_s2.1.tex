%\documentclass[12pt]{article}

\questionheader{ex:s2.1}


%%%%%%%%%%%%%%%%%%
\subsection*{\Conceptual}
%%%%%%%%%%%%%%%%%%
%%%%%%%%%%%%%%%%%%%
%%%%%%%%%%%%%%%%%%%
\begin{question}
Suppose $f(x,y)$ is a function such that 
         $\lim\limits_{(x,y)\to(0,0)}f(x,y)=10$. 

True or false: $|f(0.1,0.1)-10|<|f(0.2,0.2)-10|$
\end{question}
\begin{hint}
How does the behaviour of a function far away from $(0,0)$ affect its limit at $(0,0)$?
\end{hint}
\begin{answer}
in general, false.
\end{answer}
\begin{solution}
In general, this is false. Consider $f(x,y)=12-(1-10x)^2-(1-10y)^2$. 
\begin{itemize}
\item $\lim\limits_{(x,y)\to(0,0)}f(x,y)=12-1-1=10$ (the function is continuous)
\item $f(0.1,0.1)=12-(1-1)^2-(1-1)^2=12$
\item $f(0.2,0.2)=12-(1-2)^2-(1-2)^2=10$
\end{itemize}

We often (somewhat lazily) interpret the limit\quad ``$\lim\limits_{(x,y)\to(0,0)}f(x,y)=10$" \quad to mean that, as $(x,y)$ gets 
closer and closer to the origin, $f(x,y)$ gets closer and closer to 10. This 
isn't exactly what the definition means, though. The definition tells us that, 
we can guarantee that $f(x,y)$ be very close to 10 by choosing 
$(x,y)$ very close to $(0,0)$.

The function $f(x,y)$ can also be very close to 10 for some $(x,y)$'s that are not close to $(0,0)$. Moreover, we don't know \emph{how close} to $(0,0)$ 
we have to be in order for $f(x,y)$ to be ``very close" to 10.
\end{solution}
%%%%%%%%%%%%%%%%%%%
%%%%%%%%%%%%%%%%%%%

\begin{question}
A millstone pounds wheat into flour. The wheat sits in a basin, and the millstone pounds up and down.

Samples of wheat are taken from various places along the basin. Their diameters are measured and their position on the basin is recorded.

Consider this claim: ``As the particles get very close to the millstone, the diameters of the particles approach 50 $\mu$m."
In this context, describe the variables below from Definition~\eref{CLP200}{def limit} in the CLP-3 text.

\begin{enumerate}[(a)]
\item $\mathbf x$
\item $\mathbf a$
\item $\mathbf L$
\end{enumerate}
\end{question}
\begin{hint}
In this analogy, $f(x,y)$ is the diameter of a particle taken from the position $(x,y)$ in the basin.
\end{hint}
\begin{answer}
(a) the position of the particle in the basin \\
(b) the position in the basin that the millstone hits\\
(c) 50 $\mu$m
\end{answer}
\begin{solution}

\begin{enumerate}[(a)]
\item The function we're taking the limit of has its input as the position of the particle, and its output the size of the particle. So, $f(x,y)$ gives the size of particles found at position $(x,y)$. In the definition, we write $\mathbf x = (x,y)$. So, $\mathbf x$ is the position in the basin the particle was taken from.
\item Our claim deals with particles very close to where the millstone hits the basin, so $\mathbf a$ is the position in the basinfwhere the millstone hits.
\item $\mathbf L$ is the limit of the function: in this case, 50 $\mu$m.
\end{enumerate}

\end{solution}
%%%%%%%%%%%%%%%%%%%
%%%%%%%%%%%%%%%%%%%

\begin{question}
Let $f(x,y)=\dfrac{x^2}{x^2+y^2}$.
\begin{enumerate}[(a)]
\item Find a ray approaching the origin along which $f(x,y)=1$.
\item Find a ray approaching the origin along which $f(x,y)=0$.
\item What does the above work show about a limit of $f(x,y)$?
\end{enumerate}
\end{question}
\begin{hint}
You can probably solve (a) and (b) by just staring at $f(x,y)$.
\end{hint}
\begin{answer}
(a) along the $x$-axis\qquad (b) along the $y$-axis \qquad (c) $\lim\limits_{(x,y)\to(0,0)}f(x,y)$ does not exist
\end{answer}
\begin{solution}
\begin{enumerate}[(a)]
\item By inspection, when $y=0$, then $f(x,y)=1$ as long as $x \neq 0$. So, if we follow the $x$-axis in towards the origin, $f(x,y)=1$ along this route.
\item Also by inspection, when $x=0$, then $f(x,y)=0$ as long as $y \neq 0$. So, if we follow the $y$-axis in towards the origin, $f(x,y)=0$ along this route.
\item Since two different directions give us different values as we approach the origin,  $\lim\limits_{(x,y)\to(0,0)}f(x,y)$ does not exist.
\begin{center}
\begin{tikzpicture}
\YEaxis{1.5}{1.5}
\draw[very thick,<-,dashed] (0,.2)--(0,1) node[above left]{$f=0$};
\draw[very thick,<-,dashed] (0,-.2)--(0,-1) node[below right]{$f=0$};
\draw[thick,<-] (.2,0)--(1,0) node[above right]{$f=1$};
\draw[thick,<-] (-0.2,0)--(-1,0) node[below left]{$f=1$};
\end{tikzpicture}
\end{center}
\end{enumerate}

\end{solution}
%%%%%%%%%%%%%%%%%%%
%%%%%%%%%%%%%%%%%%%
\begin{question}
Let $f(x,y)=x^2-y^2$
\begin{enumerate}[(a)]
\item Express the function in terms of the polar coordinates $r$ and $\theta$, and simplify.
\item Suppose $(x,y)$ is a distance of 1 from the origin. 
What are the largest and smallest values of $f(x,y)$?
\item Let $r>0$. Suppose $(x,y)$ is a distance of $r$ from the origin. 
What are the largest and smallest values of $f(x,y)$?
\item Let $\epsilon>0$. Find a positive value of $r$ that guarantees $|f(x,y)|<\epsilon$ whenever $(x,y)$ is at most $r$ units from the origin.
\item What did you just show?
\end{enumerate}
\end{question}
\begin{hint}
Recall $\cos^2\theta-\sin^2\theta=\cos(2\theta)$
\end{hint}
\begin{answer}
(a) $r^2\cos(2\theta)$\quad 
(b) $\text{min}=-1,\ \text{max}=1$\quad 
(c) $\text{min}=-r^2,\ \text{max}=r^2$\quad 
(d) $r<\sqrt\epsilon$ 

(e) $\lim\limits_{(x,y)\to(0,0)}f(x,y)=0$

\end{answer}
\begin{solution}
\begin{enumerate}[(a)]
\item Since $x=r\cos\theta$ and $y=r\sin\theta$, we have that 
\begin{equation*}
f=x^2-y^2=r^2\cos^2\theta-r^2\sin^2\theta=r^2\cos(2\theta)
\end{equation*}
\item 
When $r=1$, $f=\cos(2\theta)$. So, $f(x,y)$ runs between $-1$ and $1$.
It smallest value is $-1$ and its largest value is $+1$.
\item 
The distance from $(x,y)$ to the origin is $r$ (for $r\ge0)$. So, at a distance $r$, our function is $r^2\cos(2\theta)$. Then $f(x,y)$ runs over the interval $[-r^2,r^2]$. It smallest value is $-r^2$ and its largest value is $+r^2$.
\item 
Using our answer to the last part, we have that $|f|\le r^2$. So for $0<r<\sqrt\epsilon$, we necessarily have that $|f(x,y)|<\epsilon$ 
whenever the distance from $(x,y)$ to the origin is at most $r$.
\item 
For every $\epsilon>0$, if we choose $(x,y)$ to be sufficiently close to $(0,0)$ (in particular, within a distance $r<\sqrt\epsilon$), then $f(x,y)$ is 
within distance $\epsilon$ of $0$. By Definition~\eref{CLP200}{def limit} 
in the CLP-3 text, we have that $\lim\limits_{(x,y)\to(0,0)}f(x,y)=0$.
\end{enumerate}

\end{solution}
%%%%%%%%%%%%%%%%%%%
%%%%%%%%%%%%%%%%%%%
\begin{question}
Suppose $f(x,y)$ is a polynomial. Evaluate $\lim\limits_{(x,y)\to(a,b)}f(x,y)$, where $(a,b)\in\mathbb R^2$.
\end{question}
\begin{hint}
Theorem~\eref{CLP200}{thm one d continuity}%2.1.6
in the CLP-3 text
\end{hint}
\begin{answer}
$f(a,b)$
\end{answer}
\begin{solution}
By Theorem~\eref{CLP200}{thm one d continuity},%2.1.6 
$f(x,y)$ is continuous over its domain. The domain of a polynomial is everywhere; in this case, $\mathbb R^2$. So, $f(x,y)$ is continuous at $(a,b)$. By the definition of continuity, $\lim\limits_{(x,y)\to(a,b)}f(x,y)=f(a,b)$.
\end{solution}
%%%%%%%%%%%%%%%%%%%
%%%%%%%%%%%%%%%%%%
\subsection*{\Procedural}
%%%%%%%%%%%%%%%%%%

%%%%%%%%%%%%%%%%%%%%%%%%%%%%%%%%
\begin{question}
Evaluate, if possible,
\begin{enumerate}[(a)]
\item $\dst\lim_{(x,y)\rightarrow(2,-1)}\ \big(xy+x^2\big)$
\item $\dst\lim_{(x,y)\rightarrow(0,0)}\ \frac{x}{x^2+y^2}$
\item $\dst\lim_{(x,y)\rightarrow(0,0)}\ \frac{x^2}{x^2+y^2}$
\item $\dst\lim_{(x,y)\rightarrow(0,0)}\ \frac{x^3}{x^2+y^2}$
\item $\dst\lim_{(x,y)\rightarrow(0,0)}\ \frac{x^2y^2}{x^2+y^4}$
\item $\dst\lim_{(x,y)\rightarrow (0,0)}\ 
                        \frac{(\sin x)\left(e^y-1\right)}{xy}$
\end{enumerate}
\end{question}

\begin{hint}
For parts (b), (c), (d), (e), switch to polar coordinates.
For part (f),
\begin{equation*} 
\lim_{(x,y)\rightarrow (0,0)}\ 
                        \frac{(\sin x)\left(e^y-1\right)}{xy}
=\left[\lim_{x\rightarrow 0}\ 
                        \frac{\sin x}{x}\right]\ 
 \left[\lim_{y\rightarrow 0}\ 
                        \frac{e^y-1}{y}\right]
\end{equation*}
\end{hint}

\begin{answer}
(a) $2$ \qquad
(b) undefined\qquad
(c) undefined\qquad
(d) $0$\qquad
(e) $0$ \qquad
(f) $1$
\end{answer}

\begin{solution}
(a) $\dst\lim_{(x,y)\rightarrow(2,-1)}\ \big(xy+x^2\big)=2(-1)+2^2=2$

(b) Switching to polar coordinates,
\begin{align*}
\lim_{(x,y)\rightarrow(0,0)}\ \frac{x}{x^2+y^2}
    &=\lim_{\Atop{r\rightarrow0^+}{0\le\theta<2\pi}}\ \frac{r\cos\theta}{r^2}
     =\lim_{\Atop{r\rightarrow0^+}{0\le\theta<2\pi}}\ \frac{\cos\theta}{r} 
\end{align*}
which does not exist, since, for example,
\begin{itemize}
\item 
if $\theta=0$, then 
\begin{equation*}
\lim_{\Atop{r\rightarrow0^+}{\theta=0}}\ \frac{\cos\theta}{r}
=\lim_{r\rightarrow0^+}\ \frac{1}{r}
=+\infty
\end{equation*}
\item
while if $\theta=\pi$, then 
\begin{equation*}
\lim_{\Atop{r\rightarrow0^+}{\theta=\pi}}\ \frac{\cos\theta}{r}
=\lim_{r\rightarrow0^+}\ \frac{-1}{r}
=-\infty
\end{equation*}
\end{itemize}

(c) Switching to polar coordinates,
\begin{align*}
\lim_{(x,y)\rightarrow(0,0)}\ \frac{x^2}{x^2+y^2}
  &=\lim_{\Atop{r\rightarrow0^+}{0\le\theta<2\pi}}\ \frac{r^2\cos^2\theta}{r^2}
     =\lim_{\Atop{r\rightarrow0^+}{0\le\theta<2\pi}}\ \cos^2\theta 
\end{align*}
which does not exist, since, for example,
\begin{itemize}
\item 
if $\theta=0$, then 
\begin{equation*}
\lim_{\Atop{r\rightarrow0^+}{\theta=0}}\ \cos^2\theta
=\lim_{r\rightarrow0^+}\ 1
= 1
\end{equation*}
\item
while if $\theta=\frac{\pi}{2}$, then 
\begin{equation*}
\lim_{\Atop{r\rightarrow0^+}{\theta=\pi/2}}\ \cos^2\theta
=\lim_{r\rightarrow0^+}\ 0
=0
\end{equation*}
\end{itemize}

(d) Switching to polar coordinates,
\begin{align*}
\lim_{(x,y)\rightarrow(0,0)}\ \frac{x^3}{x^2+y^2}
   &=\lim_{\Atop{r\rightarrow0^+}{0\le\theta<2\pi}}\ \frac{r^3\cos^3\theta}{r^2}
     =\lim_{\Atop{r\rightarrow0^+}{0\le\theta<2\pi}}\ r\cos^3\theta 
    =0
\end{align*}
since $|\cos\theta|\le 1$ for all $\theta$.

(e) Switching to polar coordinates,
\begin{align*}
    \lim_{(x,y)\rightarrow(0,0)}\ \frac{x^2y^2}{x^2+y^4}
    &=\lim_{\Atop{r\rightarrow0^+}{0\le\theta<2\pi}}\ 
                                \frac{r^2\cos^2\theta\ r^2\sin^2\theta}
                                      {r^2\cos^2\theta+r^4\sin^4\theta}
   =\lim_{\Atop{r\rightarrow0^+}{0\le\theta<2\pi}}\ r^2\sin^2\theta
                  \frac{\cos^2\theta}{\cos^2\theta+r^2\sin^4\theta} \\
   &=  0
\end{align*} 
Here, we used that 
\begin{equation*}
\left|\sin^2\theta\frac{\cos^2\theta} {\cos^2\theta+r^2\sin^4\theta}\right|
\le \frac{\cos^2\theta} {\cos^2\theta+r^2\sin^4\theta}
\le \left.\begin{cases}
             \frac{\cos^2\theta} {\cos^2\theta}&\text{ if }\cos\theta\ne 0 \\
              0 &\text { if }\cos\theta =0
           \end{cases}
     \right\}
\le 1
\end{equation*} 
for all $r>0$.

(f) To start, observe that
\begin{align*}
\lim_{(x,y)\rightarrow (0,0)}\ 
                  \frac{(\sin x)\left(e^y-1\right)}{xy}
          =\left[\lim_{x\rightarrow 0}\ 
                  \frac{\sin x}{x}\right]
            \left[\lim_{y\rightarrow 0}\ 
                  \frac{e^y-1}{y}\right]
\end{align*}
We may evaluate  $\dst\left[\lim_{x\rightarrow 0}\ 
                  \frac{\sin x}{x}\right]$
by l'H\^opital's rule or by using the definition of the derivative to give
\begin{equation*}
\lim_{x\rightarrow 0}\ \frac{\sin x}{x}
=\lim_{x\rightarrow 0}\ \frac{\sin x-\sin 0}{x-0}
=\diff{}{x}\sin x\bigg|_{x=0}
=\cos x\Big|_{x=0}=1
\end{equation*}
Similarly, we may evaluate  $\dst\left[\lim_{y\rightarrow 0}\ 
                  \frac{e^y-1}{y}\right]$
by l'H\^opital's rule or by using the definition of the derivative to give
\begin{equation*}
\lim_{y\rightarrow 0}\ \frac{e^y-1}{y}
=\lim_{y\rightarrow 0}\ \frac{e^y-e^0}{y-0}
=\diff{}{y}e^y\bigg|_{y=0}
=e^y\Big|_{y=0}=1
\end{equation*}
So all together
\begin{align*}
\lim_{(x,y)\rightarrow (0,0)}\ 
                  \frac{(\sin x)\left(e^y-1\right)}{xy}
          =\left[\lim_{x\rightarrow 0}\ 
                  \frac{\sin x}{x}\right]
            \left[\lim_{y\rightarrow 0}\ 
                  \frac{e^y-1}{y}\right]
          =[1]\ [1]=1
\end{align*}
\end{solution}

%%%%%%%%%%%%%%%%%%%%%%%%%%%%%%%%
\begin{question}[M253 2009D] %8
\begin{enumerate}[(a)]
\item
Find the limit: $\dst \lim_{(x,y)\to(0,0)}\frac{x^8+y^8}{x^4+y^4}$.
\item
Prove that the following limit does not exist: 
  $\dst \lim_{(x,y)\to(0,0)}\frac{xy^5}{x^8+y^{10}}$.
\end{enumerate}
\end{question}

\begin{hint}
Switch to polar coordinates.
\end{hint}

\begin{answer}
(a) $0$
\qquad
(b) See the solution.
\end{answer}

\begin{solution}
(a)
In polar coordinates, $x=r\cos\theta$, $y=r\sin\theta$, so that 
\begin{align*}
\frac{x^8+y^8}{x^4+y^4}
&=\frac{r^8\cos^8\theta+r^8\sin^8\theta}{r^4\cos^4\theta+r^4\sin^4\theta}
=r^4\frac{\cos^8\theta+\sin^8\theta}{\cos^4\theta+\sin^4\theta}
\end{align*}
As
\begin{align*}
\frac{\cos^8\theta+\sin^8\theta}{\cos^4\theta+\sin^4\theta}
&\le \frac{\cos^8\theta+2\cos^4\theta\sin^4\theta+\sin^8\theta}
             {\cos^4\theta+\sin^4\theta}
=\frac{\big(\cos^4\theta+\sin^4\theta\big)^2}{\cos^4\theta+\sin^4\theta} \\
&=\cos^4\theta+\sin^4\theta
%\le\cos^2\theta+\sin^2\theta=1
\le 2
\end{align*}
we have
\begin{equation*}
0\le \frac{x^8+y^8}{x^4+y^4}\le 2r^4
\end{equation*}
As $\dst\lim_{(x,y)\to (0,0)}2r^4=0$, the squeeze theorem yields
 $\dst\lim_{(x,y)\to(0,0)}\frac{x^8+y^8}{x^4+y^4}=0$.

(b)  
In polar coordinates
\begin{align*}
\frac{xy^5}{x^8+y^{10}}
&=\frac{r^6\cos\theta\,\sin^5\theta}{r^8\cos^8\theta+r^{10}\sin^{10}\theta}
=\frac{1}{r^2}\frac{\cos\theta\,\sin^5\theta}{\cos^8\theta+r^2\sin^{10}\theta}
\end{align*}
As $(x,y)\to (0,0)$ the first fraction $\frac{1}{r^2}\to\infty$ but the second
factor can take many different values. For example, if we send $(x,y)$
towards the origin along the $y$--axis, i.e. with 
$\theta=\pm\frac{\pi}{2}$,
\begin{align*}
\lim_{\Atop {(x,y)\to(0,0)}{x=0}}\frac{xy^5}{x^8+y^{10}}
=\lim_{y\to 0} \frac{0}{y^{10}}=0
\end{align*} 
but if we send $(x,y)$ towards the origin along the line $y=x$, 
i.e. with  $\theta=\frac{\pi}{4},\frac{5\pi}{4}$,
\begin{align*}
\lim_{\Atop {(x,y)\to(0,0)}{y=x} }\frac{xy^5}{x^8+y^{10}}
=\lim_{x\to 0} \frac{x^6}{x^8+x^{10}}
=\lim_{x\to 0} \frac{1}{x^2}\frac{1}{1+x^2}
=+\infty
\end{align*} 
and if we send $(x,y)$ towards the origin along the line $y=-x$, 
i.e. with  $\theta=-\frac{\pi}{4},\frac{3\pi}{4}$,
\begin{align*}
\lim_{\Atop {(x,y)\to(0,0)}{y=-x} }\frac{xy^5}{x^8+y^{10}}
=\lim_{x\to 0} \frac{-x^6}{x^8+x^{10}}
=\lim_{x\to 0}- \frac{1}{x^2}\frac{1}{1+x^2}
=-\infty
\end{align*} 
So $\frac{xy^5}{x^8+y^{10}}$ does not approach a single value as 
$(x,y)\to(0,0)$ and the limit does not exist.

\end{solution}

%%%%%%%%%%%%%%%%%%%%%%%%%%%%%%%%
\begin{question}[M2226 2009D] %7
Evaluate each of the following limits or show that it does not exist.
\begin{enumerate}[(a)]
\item
$\dst\lim_{(x,y)\rightarrow (0,0)}\frac{x^3-y^3}{x^2+y^2}$
\item
$\dst\lim_{(x,y)\rightarrow (0,0)}\frac{x^2-y^4}{x^2+y^4}$
\end{enumerate}
\end{question}

\begin{hint}
(a) Switch to polar coordinates.

(b) What are the limits when (i) $x=0$ and $y\rightarrow 0$ and when
(ii) $y=0$ and $x\rightarrow 0$?
\end{hint}

\begin{answer}
(a) $0$

(b) The limit does not exist since the limits (i) $x=0$,
$y\rightarrow 0$ and (ii) $y=0$, $x\rightarrow 0$ are different.
\end{answer}

\begin{solution}
(a) In polar coordinates
\begin{equation*}
\frac{x^3-y^3}{x^2+y^2}=\frac{r^3\cos^3\theta-r^3\sin^3\theta}{r^2}
=r\cos^3\theta-r\sin^3\theta
\end{equation*}
Since
\begin{equation*}
\big|r\cos^3\theta-r\sin^3\theta\big|\le 2r
\end{equation*}
and $2r\rightarrow 0$ as $r\rightarrow 0$, the limit exists and is $0$.

(b)
The limit as we approach $(0,0)$ along the $x$-axis is
\begin{align*}
\lim_{t\rightarrow 0}\frac{x^2-y^4}{x^2+y^4}\bigg|_{(x,y)=(t,0)}
=\lim_{t\rightarrow 0}\frac{t^2-0^4}{t^2+0^4}
=1
\end{align*}
On the other hand the limit as we approach $(0,0)$ along the $y$-axis is
\begin{align*}
\lim_{t\rightarrow 0}\frac{x^2-y^4}{x^2+y^4}\bigg|_{(x,y)=(0,t)}
=\lim_{t\rightarrow 0}\frac{0^2-t^4}{0^2+t^4}
=-1
\end{align*}
These are different, so the limit as $(x,y)\rightarrow 0$ does not exist.

We can gain a more detailed understanding of the behaviour of 
$\frac{x^2-y^4}{x^2+y^4}$ near the origin by switching to polar coordinates.
\begin{equation*}
\frac{x^2-y^4}{x^2+y^4}
=\frac{r^2\cos^2\theta-r^4\sin^4\theta}{r^2\cos^2\theta+r^4\sin^4\theta}
=\frac{\cos^2\theta-r^2\sin^4\theta}{\cos^2\theta+r^2\sin^4\theta}
\end{equation*}
Now fix any $\theta$ and let $r\rightarrow 0$ (so that we are approaching the origin along the ray that makes an angle $\theta$ with the positive $x$-axis).
If $\cos\theta\ne 0$ (i.e. the ray is not part of the $y$-axis)
\begin{align*}
\lim_{r\rightarrow 0}
     \frac{\cos^2\theta-r^2\sin^4\theta}{\cos^2\theta+r^2\sin^4\theta}
=\frac{\cos^2\theta}{\cos^2\theta}
=1
\end{align*}
But if $\cos\theta= 0$ (i.e. the ray is part of the $y$-axis)
\begin{align*}
\lim_{r\rightarrow 0}
     \frac{\cos^2\theta-r^2\sin^4\theta}{\cos^2\theta+r^2\sin^4\theta}
=\lim_{r\rightarrow 0}
     \frac{-r^2\sin^4\theta}{r^2\sin^4\theta}
=\frac{-\sin^4\theta}{\sin^4\theta}
=-1
\end{align*}
\end{solution}

%%%%%%%%%%%%%%%%%%
\subsection*{\Application}
%%%%%%%%%%%%%%%%%%

%%%%%%%%%%%%%%%%%%%%%%%%%%%%%%%%
\begin{question}[M226 2010D] %6
Evaluate each of the following limits or show that it does not exist.
\begin{enumerate}[(a)]
\item
$\dst\lim_{(x,y)\rightarrow (0,0)}\frac{2x^2 + x^2y - y^2x + 2y^2}{x^2 + y^2}$
\item
$\dst\lim_{(x,y)\rightarrow(0,1)} \frac{x^2y^2 -2 x^2y + x^2}
                                       {(x^2 + y^2-2y+1)^2}$
\end{enumerate}
\end{question}

\begin{hint}
For part (a) switch to polar coordinates.
For part (b), switch to polar coordinates
centred on $(0,1)$. That is, make the change of variables
$x=r\cos\theta$, $y=1+r\sin\theta$. 
\end{hint}

\begin{answer}
(a) $2$\qquad
(b) The limit does not exist. See the solution.
\end{answer}

\begin{solution}
(a) In polar coordinates $x=r\cos\theta$, $y=r\sin\theta$
\begin{align*}
\frac{2x^2 + x^2y - y^2x + 2y^2}{x^2 + y^2} 
&=\frac{2r^2\cos^2\theta + r^3\cos^2\theta\sin\theta - r^3\cos\theta\sin^2\theta 
                 + 2r^2\sin^2\theta}{r^2} \\ 
&=2+ r\big[\cos^2\theta\sin\theta - \sin^2\theta\cos\theta \big]
\end{align*}
As
\begin{equation*}
r\big|\cos^2\theta\sin\theta - \sin^2\theta\cos\theta \big|
\le 2r
\rightarrow 0\text{ as }r\rightarrow 0
\end{equation*}
we have
\begin{equation*}
\lim_{(x,y)\rightarrow(0,0)} \frac{2x^2 + x^2y - y^2x + 2y^2}{x^2 + y^2}=2
\end{equation*}

(b)
Since 
\begin{align*}
\frac{x^2y^2 -2 x^2y + x^2} {(x^2 + y^2-2y+1)^2} 
=\frac{x^2(y-1)^2} {\big[x^2 + (y-1)^2\big]^2} 
\end{align*}
and, in polar coordinates centred on $(0,1)$,
$x=r\cos\theta$, $y=1+r\sin\theta$,
\begin{equation*}
\frac{x^2(y-1)^2} {\big[x^2 + (y-1)^2\big]^2} 
=\frac{r^4\cos^2\theta\sin^2\theta}{r^4}
=\cos^2\theta\sin^2\theta
\end{equation*}
we have that the limit does not exist. For example, if we send $(x,y)$
to $(0,1)$ along the line $y=1$, so that $\theta=0$, we get the limit $0$,
while if we send $(x,y)$ to $(0,1)$ along the line $y=x+1$, so that 
$\theta=\frac{\pi}{4}$, we get the limit $\frac{1}{4}$.
\end{solution}

%%%%%%%%%%%%%%%%%%%%%%%%%%%%%%%%
\begin{question}
Define, for all $(x,y)\ne(0,0)$, $f(x,y)=\frac{x^2y}{x^4+y^2}$.
\begin{enumerate}[(a)]
\item
Let $0\le \theta<2\pi$. Compute
$\dst\lim_{r\rightarrow 0^+}f(r\cos\theta,r\sin\theta)$.

\item 
Compute $\dst\lim_{x\rightarrow 0}f(x,x^2)$.

\item
Does $\dst\lim_{(x,y)\rightarrow (0,0)}f(x,y)$ exist?
\end{enumerate}
\end{question}

\begin{hint}
For part (c), does there exist a single number, $L$, with the property that
$f(x,y)$ is really close to $L$ for all $(x,y)$ that are really close to 
$(0,0)$?
\end{hint}

\begin{answer}
(a) $0$\qquad
(b) $\frac{1}{2}$ \qquad
(c) No.
\end{answer}

\begin{solution}
(a) We have
\begin{align*}
\lim_{r\rightarrow 0^+}f(r\cos\theta,r\sin\theta)
&=\lim_{r\rightarrow 0^+}
\frac{(r\cos\theta)^2(r\sin\theta)}{(r\cos\theta)^4+(r\sin\theta)^2} \\
&=\lim_{r\rightarrow 0^+}r\ \frac{\cos^2\theta\sin\theta}{r^2\cos^4\theta+\sin^2\theta} \\
&=\lim_{r\rightarrow 0^+}r\ 
\lim_{r\rightarrow 0^+}\frac{\cos^2\theta\sin\theta}
                         {r^2\cos^4\theta+\sin^2\theta}
\end{align*}
Observe that, if $\sin\theta=0$, then
\begin{equation*}
\frac{\cos^2\theta\sin\theta}{r^2\cos^4\theta+\sin^2\theta}=0
\end{equation*}
for all $r\ne 0$. If $\sin\theta\ne 0$,
\begin{align*}
\lim_{r\rightarrow 0^+}   
     \frac{\cos^2\theta\sin\theta}{r^2\cos^4\theta+\sin^2\theta}
&=\frac{\cos^2\theta\sin\theta}{\sin^2\theta}
=\frac{\cos^2\theta}{\sin\theta}
\end{align*}
So the limit 
$\dst\lim_{r\rightarrow 0^+}
\frac{\cos^2\theta\sin\theta}{r^2\cos^4\theta+\sin^2\theta}$
exists (and is finite) for all fixed $\theta$ and 
\begin{equation*}
\lim\limits_{r\rightarrow 0^+}f(r\cos\theta,r\sin\theta)=0
\end{equation*}

(b) We have
\begin{equation*}
\lim_{x\rightarrow 0}f(x,x^2)
=\lim_{x\rightarrow 0}\frac{x^2x^2}{x^4+{(x^2)}^2}
=\lim_{x\rightarrow 0}\frac{x^4}{2x^4}
=\frac{1}{2}
\end{equation*}

(c)
Note that in part (a) we showed that as $(x,y)$ approaches $(0,0)$ along
any straight line, $f(x,y)$ approaches the limit zero.
In part (b) we have just shown that as $(x,y)$ approaches $(0,0)$ along
the parabola $y=x^2$, $f(x,y)$ approaches the limit $\half$, {\bf not} zero.
So $f(x,y)$ takes values very close to $0$, for some $(x,y)$'s 
that are  really near $(0,0)$ and also takes values very close to 
$\frac{1}{2}$, for other $(x,y)$'s that are  really near $(0,0)$.
There is no single number, $L$, with the property that
$f(x,y)$ is really close to $L$ for all $(x,y)$ that are really 
close to $(0,0)$. So the limit does not exist.

\end{solution}

%%%%%%%%%%%%%%%%%%%%%%%%%%%%%%%%
\begin{question}[M226 2007D] %1
Compute the following limits or explain why they do not exist.
\begin{enumerate}[(a)]
\item $\dst\lim_{(x,y)\rightarrow(0,0)}\frac{xy}{x^2+y^2}$
\item $\dst\lim_{(x,y)\rightarrow(0,0)}\frac{\sin(xy)}{x^2+y^2}$
\item $\dst\lim_{(x,y)\rightarrow(-1,1)}\frac{x^2+2xy^2+y^4}{1+y^4}$
\item $\dst\lim_{(x,y)\rightarrow(0,0)}|y|^x$
\end{enumerate}
\end{question}

\begin{hint}
For part (b), consider the ratio of $\frac{\sin(xy)}{x^2+y^2}$
(from part (b)) and $\frac{xy}{x^2+y^2}$ (from part (a)), and recall that
$\dst\lim_{t\rightarrow 0}\tfrac{\sin t}{t}=1$.

For part (d) consider the limits along the positive $x$- and $y$-axes.
\end{hint}

\begin{answer}
(a), (b), (d) Do not exist. See the solutions.\qquad
(c) $0$
\end{answer}

\begin{solution}
(a) Since, in polar coordinates,
\begin{equation*}
\frac{xy}{x^2+y^2}=\frac{r^2\cos\theta\sin\theta}{r^2}
=\cos\theta\sin\theta
\end{equation*}
we have that the limit does not exist. For example, 
\begin{itemize}
\item if we send $(x,y)$
to $(0,0)$ along the positive $x$-axis, so that $\theta=0$, 
we get the limit $\sin\theta\cos\theta\big|_{\theta=0}=0$, 
\item
while if we send $(x,y)$ to $(0,0)$ along the line $y=x$ in the first quadrant, 
so that  $\theta=\frac{\pi}{4}$, we get the limit $\sin\theta\cos\theta\big|_{\theta=\pi/4}=\frac{1}{2}$.
\end{itemize}

(b) This limit does not exist, since if it were to exist the limit
\begin{equation*}
\lim_{(x,y)\rightarrow(0,0)}\frac{xy}{x^2+y^2}
=\lim_{(x,y)\rightarrow(0,0)}\frac{xy}{\sin(xy)}\ \frac{\sin(xy)}{x^2+y^2}
=\lim_{(x,y)\rightarrow(0,0)}\frac{xy}{\sin(xy)}\ 
\lim_{(x,y)\rightarrow(0,0)}\frac{\sin(xy)}{x^2+y^2}
\end{equation*}
would also exist. (Recall that $\dst\lim_{t\rightarrow 0}\tfrac{\sin t}{t}
=1$.)


(c) Since
\begin{align*}
\lim_{(x,y)\rightarrow(-1,1)}\big[x^2+2xy^2+y^4\big]
&=(-1)^2+2(-1)(1)^2+(1)^4=0 \\
\lim_{(x,y)\rightarrow(-1,1)}\big[1+y^4\big]
&=1+(1)^4=2
\end{align*}
and the second limit is nonzero,
\begin{equation*}
\lim_{(x,y)\rightarrow(-1,1)}\frac{x^2+2xy^2+y^4}{1+y^4}=\frac{0}{2}=0
\end{equation*}




(d)  Since the limit along the positive $x$-axis
\begin{equation*}
\lim_{\Atop{t\rightarrow 0}{t>0}}|y|^x\Big|_{(x,y)=(t,0)}
=\lim_{\Atop{t\rightarrow 0}{t>0}}0^t
=\lim_{\Atop{t\rightarrow 0}{t>0}}0
=0
\end{equation*}
and the limit along the $y$-axis
\begin{equation*}
\lim_{t\rightarrow 0}|y|^x\Big|_{(x,y)=(0,t)}
=\lim_{t\rightarrow 0}|t|^0
=\lim_{t\rightarrow 0}1
=1
\end{equation*}
are different, the limit as $(x,y)\rightarrow 0$ does not exist.
\end{solution}

%%%%%%%%%%%%%%%%%%%%%%%%%%%%%%%%
\begin{question}
Evaluate each of the following limits or show that it does not exist.
\begin{enumerate}[(a)]
\item
$\dst\lim_{(x,y)\rightarrow (0,0)}\begin{cases}
                                   \frac{x^2}{y-x} &\text{if $y\ne x$} \\
                                   0 & \text{if $y=x$}
                                   \end{cases}
$
\item
$\dst\lim_{(x,y)\rightarrow(0,0)}\begin{cases}
                                   \frac{x^8}{y-x} &\text{if $y\ne x$} \\
                                   0 & \text{if $y=x$}
                                   \end{cases}
$
\end{enumerate}
\end{question}

\begin{hint}
For part (a), determine what happens as $(x,y)$ tends to $(0,0)$
along the curve $y=x+\frac{x^2}{a}$, where $a$ is any nonzero constant.

\end{hint}

\begin{answer}
(a), (b)  The limit does not exist. See the solution.
\end{answer}

\begin{solution}
(a) Let $a$ be any nonzero constant. When $y=x+\frac{x^2}{a}$ and $x\ne 0$,
\begin{equation*}
\frac{x^2}{y-x} =\frac{x^2}{x^2/a} =a
\end{equation*}
So the limit along the curve $y=x+\frac{x^2}{a}$ is
\begin{equation*}
\lim_{t\rightarrow 0}\frac{x^2}{y-x}\Big|_{(x,y)=(t,t+t^2/a)}
=\lim_{t\rightarrow 0}a
=a
\end{equation*}
In particular, the limit along the curve $y=x+x^2$, which is $1$, and 
the limit along the curve $y=x-x^2$, which is $-1$, are different.
So the limit as $(x,y)\rightarrow 0$ does not exist.

(b) Let $a$ be any nonzero constant. When $y=x+\frac{x^8}{a}$ and $x\ne 0$,
\begin{equation*}
\frac{x^8}{y-x} =\frac{x^8}{x^8/a} =a
\end{equation*}
So the limit along the curve $y=x+\frac{x^8}{a}$ is
\begin{equation*}
\lim_{t\rightarrow 0}\frac{x^8}{y-x}\Big|_{(x,y)=(t,t+t^8/a)}
=\lim_{t\rightarrow 0}a
=a
\end{equation*}
In particular, the limit along the curve $y=x+x^8$, which is $1$, and 
the limit along the curve $y=x-x^8$, which is $-1$, are different.
So the limit as $(x,y)\rightarrow 0$ does not exist.
\end{solution}

