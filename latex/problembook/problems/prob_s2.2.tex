%\documentclass[12pt]{article}

\questionheader{ex:s2.2}


%%%%%%%%%%%%%%%%%%
\subsection*{\Conceptual}
%%%%%%%%%%%%%%%%%%

%%%%%%%%%%%%%%%%%%%%%%%%%%%%%%%%
\begin{question}
Let $f(x,y) = e^x\cos y$. The following table gives some values of $f(x,y)$.

\begin{center}
\renewcommand{\arraystretch}{1.3}
     \begin{tabular}{c|c|c|c|}
       & $x=0$ & $x=0.01$ & $x=0.1$  \\    
    \hline
     $y=-0.1$  & 0.99500 & 1.00500 & 1.09965 \\ \hline
     $y=-0.01$ & 0.99995 & 1.01000 & 1.10512 \\ \hline
     $y=0$     & 1.0     & 1.01005 & 1.10517 \\ \hline
     \end{tabular}
\renewcommand{\arraystretch}{1.0}
\end{center}

\begin{enumerate}[(a)]
\item
Find two different approximate values for $\pdiff{f}{x}(0,0)$ using the data in 
the above table.
\item
Find two different approximate values for $\pdiff{f}{y}(0,0)$ using the data in the above table.
\item
Evaluate $\pdiff{f}{x}(0,0)$ and $\pdiff{f}{y}(0,0)$ exactly.
\end{enumerate}
\end{question}

\begin{hint}
Review Definition \eref{CLP200}{def partials}  in the CLP-3 text.
\end{hint}

\begin{answer}
\begin{enumerate}[(a)]
\item
\begin{align*}
\pdiff{f}{x}(0,0)
\approx \left.\frac{f(h,0)-f(0,0)}{h}\right|_{h=0.1}
=\frac{1.10517-1}{0.1}
=1.0517
\end{align*}
and
\begin{align*}
\pdiff{f}{x}(0,0)
\approx \left.\frac{f(h,0)-f(0,0)}{h}\right|_{h=0.01}
=\frac{1.01005-1}{0.01}
=1.005
\end{align*}
\item
\begin{align*}
\pdiff{f}{y}(0,0)
\approx \left.\frac{f(0,h)-f(0,0)}{h}\right|_{h=-0.1}
=\frac{0.99500-1}{-0.1}
=0.0500
\end{align*}
and
\begin{align*}
\pdiff{f}{y}(0,0)
\approx \left.\frac{f(0,h)-f(0,0)}{h}\right|_{h=-0.01}
=\frac{0.99995-1}{-0.01}
=.0050
\end{align*}
\item
$
\pdiff{f}{x}(0,0) =1 
$
and
$\pdiff{f}{y}(0,0) = 0
$
\end{enumerate}
\end{answer}

\begin{solution}
\begin{enumerate}[(a)]
\item
By definition
\begin{equation*}
\pdiff{f}{x}(0,0)
=\lim_{h\rightarrow 0}\frac{f(h,0)-f(0,0)}{h}
\end{equation*}
One approximation to this is
\begin{align*}
\pdiff{f}{x}(0,0)
\approx \left.\frac{f(h,0)-f(0,0)}{h}\right|_{h=0.1}
=\frac{1.10517-1}{0.1}
=1.0517
\end{align*}
Another approximation to this is
\begin{align*}
\pdiff{f}{x}(0,0)
\approx \left.\frac{f(h,0)-f(0,0)}{h}\right|_{h=0.01}
=\frac{1.01005-1}{0.01}
=1.005
\end{align*}
\item
By definition
\begin{equation*}
\pdiff{f}{y}(0,0)
=\lim_{h\rightarrow 0}\frac{f(0,h)-f(0,0)}{h}
\end{equation*}
One approximation to this is
\begin{align*}
\pdiff{f}{y}(0,0)
\approx \left.\frac{f(0,h)-f(0,0)}{h}\right|_{h=-0.1}
=\frac{0.99500-1}{-0.1}
=0.0500
\end{align*}
Another approximation to this is
\begin{align*}
\pdiff{f}{y}(0,0)
\approx \left.\frac{f(0,h)-f(0,0)}{h}\right|_{h=-0.01}
=\frac{0.99995-1}{-0.01}
=.0050
\end{align*}
\item
To take the partial derivative with respect to $x$ at $(0,0)$,
we set $y=0$, differentiate with respect to $x$ and then set $x=0$. So
\begin{align*}
\pdiff{f}{x}(0,0) = \left.\diff{}{x} e^x\cos 0\right|_{x=0}
=\left.e^x\right|_{x=0}=1
\end{align*}
To take the partial derivative with respect to $y$ at $(0,0)$,
we set $x=0$, differentiate with respect to $y$ and then set $y=0$. So
\begin{align*}
\pdiff{f}{y}(0,0) = \left.\diff{}{y} e^0\cos y\right|_{y=0}
=\left.\sin y\right|_{y=0}
=0
\end{align*}
\end{enumerate}
\end{solution}

%%%%%%%%%%%%%%%%%%%
\begin{question}
You are traversing an undulating landscape. Take the $z$-axis to be straight up towards the sky, the positive $x$-axis to be due south, and the positive $y$-axis to be due east. Then the landscape near you is described by the equation $z=f(x,y)$, with you at the point $(0,0,f(0,0))$. The function $f(x,y)$ is differentiable.

Suppose $f_y(0,0)<0$. Is it possible that you are at a summit? Explain.

\end{question}
\begin{hint}
What happens if you move ``backwards," in the negative $y$ direction?
\end{hint}
\begin{answer}
No: you can go higher by moving in the negative $y$ direction.
\end{answer}
\begin{solution}

If $f_y(0,0)<0$, then $f(0,y)$ decreases as $y$ increases from $0$.
Thus moving in the positive $y$ direction takes you downhill. This means 
you aren't at the lowest point in a valley, since you can still move downhill.
On the other hand, as $f_y(0,0)<0$, $f(0,y)$ also decreases as $y$ 
increases towards $0$ from slightly negative values. Thus if you move in the \emph{negative} $y$-direction from $y=0$, your height $z$ will \emph{increase}. 
So you are not at a locally highest point---you're not at a summit.

\end{solution}
%%%%%%%%%%%%%%%%%%%

%%%%%%%%%%%%%%%%%%%%%%%%%%%%%%%%
\begin{question}[M226 2009D] %1
Let
\begin{equation*}
f(x,y)=\begin{cases}\frac{x^2y}{x^2+y^2}& \text{if $(x,y)\ne (0,0)$} \\
                \noalign{\vskip0.05in}
                0 & \text{if $(x,y)=(0,0)$}
       \end{cases}
\end{equation*}
Compute, directly from the definitions,
\begin{enumerate}[(a)]
\item
$\pdiff{f}{x}(0,0)$ 
\item
$\pdiff{f}{y}(0,0)$
\item
$\diff{}{t} f(t,t)\Big|_{t=0}$
\end{enumerate}
\end{question}

\begin{hint}
For (a) and (b), remember $\pdiff{f}{x}(x,y)=\lim\limits_{h\to0}\frac{f(x+h,y)-f(x,y)}{h}$ and
$\pdiff{f}{y}(x,y)=\lim\limits_{h\to0}\frac{f(x,y+h)-f(x,y)}{h}$. For (c), you're finding the derivative of a function of one variable, say $g(t)$, where \begin{equation*}
g(t)=f(t,t)=\begin{cases}
                  \frac{t^2t}{t^2+t^2} & \text{if $t\ne 0$} \\
                  0                    & \text{if $t= 0$}
                 \end{cases}
\end{equation*}
\end{hint}

\begin{answer}
(a) $0$\qquad
(b) $0$\qquad
(c) $\frac{1}{2}$
\end{answer}

\begin{solution}
(a) By definition
\begin{align*}
\pdiff{f}{x}(0,0)
&=\lim_{\De x\rightarrow 0}\frac{f(\De x,0)-f(0,0)}{\De x} \\
&=\lim_{\De x\rightarrow 0}\frac{\frac{(\De x^2)(0)}{\De x^2+0^2}-0}{\De x} \\
&=0
\end{align*}

(b) By definition
\begin{align*}
\pdiff{f}{y}(0,0)
&=\lim_{\De y\rightarrow 0}\frac{f(0,\De y)-f(0,0)}{\De y} \\
&=\lim_{\De y\rightarrow 0}\frac{\frac{(0^2)(\De y)}{0^2+\De y^2}-0}{\De y} \\
&=0
\end{align*}

(c) By definition
\begin{align*}
\diff{}{t} f(t,t)\Big|_{t=0}
&=\lim_{t\rightarrow 0}\frac{f(t,t)-f(0,0)}{t} \\
&=\lim_{h\rightarrow 0}\frac{\frac{(t^2)(t)}{t^2+t^2}-0}{t} \\
&=\lim_{t\rightarrow 0}\frac{t/2}{t} \\
&=\frac{1}{2}
\end{align*}

\end{solution}



%%%%%%%%%%%%%%%%%%
\subsection*{\Procedural}
%%%%%%%%%%%%%%%%%%


%%%%%%%%%%%%%%%%%%%%%%%%%%%%%%%%
\begin{question}
Find all first partial derivatives of the following functions 
and evaluate them at the given point.
\begin{enumerate}[(a)]
\item
$f(x,y,z)=x^3y^4z^5\qquad (0,-1,-1)$
\item
$w(x,y,z)=\ln\left(1+e^{xyz}\right)\qquad (2,0,-1)$
\item
$f(x,y)=\frac{1}{\sqrt{x^2+y^2}}\qquad (-3,4)$
\end{enumerate}
\end{question}

%\begin{hint}
%
%\end{hint}

\begin{answer}
(a)
\begin{align*}
f_x(x,y,z)&=3x^2y^4z^5 & f_x(0,-1,-1)&=0\\ 
f_y(x,y,z)&=4x^3y^3z^5 & f_y(0,-1,-1)&=0\\
f_z(x,y,z)&=5x^3y^4z^4 & f_z(0,-1,-1)&=0
\end{align*}

(b)
\begin{align*}
w_x(x,y,z)&=\frac{yz e^{xyz}}{1+e^{xyz}} & w_x(2,0,-1)&=0\\
w_y(x,y,z)&=\frac{xz e^{xyz}}{1+e^{xyz}} & w_y(2,0,-1)&=-1\\
w_z(x,y,z)&=\frac{xy e^{xyz}}{1+e^{xyz}} & w_z(2,0,-1)&=0
\end{align*}

(c)
\begin{align*}
f_x(x,y)&=-\frac{x}{(x^2+y^2)^{3/2}} & f_x(-3,4)&=\frac{3}{125}\\
f_y(x,y)&=-\frac{y}{(x^2+y^2)^{3/2}} & f_y(-3,4)&=-\frac{4}{125} 
\end{align*}
\end{answer}

\begin{solution}
(a)
\begin{align*}
f_x(x,y,z)&=3x^2y^4z^5 & f_x(0,-1,-1)&=0\\ 
f_y(x,y,z)&=4x^3y^3z^5 & f_y(0,-1,-1)&=0\\
f_z(x,y,z)&=5x^3y^4z^4 & f_z(0,-1,-1)&=0
\end{align*}

(b)
\begin{align*}
w_x(x,y,z)&=\frac{yz e^{xyz}}{1+e^{xyz}} & w_x(2,0,-1)&=0\\
w_y(x,y,z)&=\frac{xz e^{xyz}}{1+e^{xyz}} & w_y(2,0,-1)&=-1\\
w_z(x,y,z)&=\frac{xy e^{xyz}}{1+e^{xyz}} & w_z(2,0,-1)&=0
\end{align*}

(c)
\begin{align*}
f_x(x,y)&=-\frac{x}{(x^2+y^2)^{3/2}} & f_x(-3,4)&=\frac{3}{125}\\
f_y(x,y)&=-\frac{y}{(x^2+y^2)^{3/2}} & f_y(-3,4)&=-\frac{4}{125} 
\end{align*}

\end{solution}


%%%%%%%%%%%%%%%%%%%%%%%%%%%%%%%%
\begin{question}
Show that the function $z(x,y)=\frac{x+y}{x-y}$ obeys
\begin{equation*}
x\pdiff{z}{x}(x,y)+y\pdiff{z}{y}(x,y) = 0
\end{equation*}
\end{question}

\begin{hint}
Just evaluate $x\pdiff{z}{x}(x,y)+y\pdiff{z}{y}(x,y)$.
\end{hint}

\begin{answer}
See the solution.
\end{answer}

\begin{solution}
By the quotient rule
\begin{alignat*}{3}
\pdiff{z}{x}(x,y)
&=\frac{(1)(x-y)-(x+y)(1)}{(x-y)^2}
&&=\frac{-2y}{(x-y)^2}\\
\pdiff{z}{y}(x,y)
&=\frac{(1)(x-y)-(x+y)(-1)}{(x-y)^2}
&&=\frac{2x}{(x-y)^2}
\end{alignat*}
Hence
\begin{equation*}
x\pdiff{z}{x}(x,y)+y\pdiff{z}{y}(x,y) 
=\frac{-2xy+2yx}{(x-y)^2}
=0
\end{equation*}
\end{solution}

%%%%%%%%%%%%%%%%%%%%%%%%%%%%%%%%
\begin{question}[M200 2010A] %1
A surface $z(x, y)$ is defined by $zy - y + x = \ln(xyz)$.
\begin{enumerate}[(a)]
\item
Compute $\pdiff{z}{x}$, $\pdiff{z}{y}$ in terms of $x$, $y$, $z$.
\item
Evaluate $\pdiff{z}{x}$ and $\pdiff{z}{y}$ at $(x, y, z) = (-1, -2, 1/2)$.
\end{enumerate}
\end{question}

%\begin{hint}
%
%\end{hint}

\begin{answer}
(a) 
$\pdiff{z}{x} = \frac{z(1-x)}{x(yz-1)}$,\quad
$\pdiff{z}{y} = \frac{z(1+y-yz)}{y(yz-1)}$

(b) 
$\pdiff{z}{x}(-1,-2) =\frac{1}{2}$,\quad
$\pdiff{z}{y}(-1,-2) =0$.
\end{answer}

\begin{solution}
(a) We are told that $z(x,y)$ obeys
\begin{align*}
z(x,y)\, y - y + x = \ln\big(xy\,z(x,y)\big)
\tag{$*$}\end{align*}
for all $(x,y)$ (near $(-1,-2)$).  Differentiating $(*)$ with respect to $x$ 
gives
\begin{align*}
y\,\pdiff{z}{x}(x,y) + 1 = \frac{1}{x} + \frac{\pdiff{z}{x}(x,y)}{z(x,y)}
\implies \pdiff{z}{x}(x,y) = \frac{\frac{1}{x}-1}{y-\frac{1}{z(x,y)}}
\end{align*}
or, dropping the arguments $(x,y)$ and multiplying both the numerator and denominator by $xz$,
\begin{align*}
\pdiff{z}{x} = \frac{z-xz}{xyz-x} = \frac{z(1-x)}{x(yz-1)}
\end{align*}
Differentiating $(*)$ with respect to $y$ 
gives
\begin{align*}
z(x,y)+y\,\pdiff{z}{y}(x,y) - 1 = \frac{1}{y} + \frac{\pdiff{z}{y}(x,y)}{z(x,y)}
\implies \pdiff{z}{y}(x,y) = \frac{\frac{1}{y}+1-z(x,y)}{y-\frac{1}{z(x,y)}}
\end{align*}
or, dropping the arguments $(x,y)$ and multiplying both the numerator and denominator by $yz$,
\begin{align*}
\pdiff{z}{y} = \frac{z+yz-yz^2}{y^2z-y} = \frac{z(1+y-yz)}{y(yz-1)}
\end{align*}

(b)  When $(x,y,z) = (-1, -2, 1/2)$,
\begin{align*}
\pdiff{z}{x}(-1,-2) 
  &= \left.\frac{\frac{1}{x}-1}{y-\frac{1}{z}}
                         \right|_{(x,y,z) = (-1, -2, 1/2)}
   =\frac{\frac{1}{-1}-1}{-2-2}
   =\frac{1}{2} \\
\pdiff{z}{y}(-1,-2)  &= \left.\frac{\frac{1}{y}+1-z}{y-\frac{1}{z}}
                         \right|_{(x,y,z) = (-1, -2, 1/2)}
         =\frac{\frac{1}{-2}+1-\frac{1}{2}}{-2-2}
         =0
\end{align*}
\end{solution}

%%%%%%%%%%%%%%%%%%%%%%%%%%%%%%%%
\begin{question}[M200 2010D] %3
Find  $\pdiff{U}{T}$ and $\pdiff{T}{V}$ at $(1, 1, 2, 4)$ if $(T, U, V, W)$ are related by
\begin{equation*}
(TU-V)^2 \ln(W-UV) = \ln 2
\end{equation*}
\end{question}

%\begin{hint}
%
%\end{hint}

\begin{answer}
$\pdiff{U}{T}(1,2,4) = -\frac{2\ln(2)}{1+2\ln(2)}$\qquad
$\pdiff{T}{V}(1,2,4) = 1 -\frac{1}{4\ln(2)}$
\end{answer}

\begin{solution}
We are told that the four variables $T$, $U$, $V$, $W$ obey the
the single equation $(TU-V)^2 \ln(W-UV) = \ln 2$. So they are not all
independent variables. Roughly speaking, we can treat any three of them
as independent variables and solve the given equation for the fourth 
as a function of the three chosen independent variables.


We are first asked to find $\pdiff{U}{T}$. This implicitly tells to
treat $T$, $V$ and $W$ as independent variables and to view $U$
as a function $U(T,V,W)$ that obeys 
\begin{equation*}
\big(T\, U(T,V,W)-V\big)^2 \ln\big(W-U(T,V,W)\,V\big) = \ln 2
\tag{E1}\end{equation*}
for all $(T, U, V, W)$ sufficiently near $(1, 1, 2, 4)$.
Differentiating (E1) with respect to $T$ gives
\begin{align*}
&2\big(T\, U(T,V,W)-V\big)
         \left[ U(T,V,W) +T\ \pdiff{U}{T}(T,V,W)\right] 
            \ln\big(W-U(T,V,W)\,V\big) \\
&\hskip1in
  -\big(T\, U(T,V,W)-V\big)^2 \frac{1}{W-U(T,V,W)\,V}\pdiff{U}{T}(T,V,W)\,V = 0
\end{align*}
In particular, for $(T, U, V, W)=(1, 1, 2, 4)$,
\begin{align*}
&2\big((1)(1)-2\big)
         \left[ 1 +(1)\pdiff{U}{T}(1,2,4)\right] 
            \ln\big(4-(1)(2)\big) \\
&\hskip1in
  -\big((1)(1)-2\big)^2 \frac{1}{4-(1)(2)}\pdiff{U}{T}(1,2,4)\,(2) = 0
\end{align*}
This simplifies to
\begin{align*}
-2\left[ 1 +\pdiff{U}{T}(1,2,4)\right] \ln(2)
                        -\pdiff{U}{T}(1,2,4)=0 
\implies
\pdiff{U}{T}(1,2,4) = -\frac{2\ln(2)}{1+2\ln(2)}
\end{align*}

\medskip

We are then asked to find $\pdiff{T}{V}$. This implicitly tells to
treat $U$, $V$ and $W$ as independent variables and to view $T$
as a function $T(U,V,W)$ that obeys 
\begin{equation*}
\big(T(U,V,W)\, U-V\big)^2 \ln\big(W-U\,V\big) = \ln 2
\tag{E2}\end{equation*}
for all $(T, U, V, W)$ sufficiently near $(1, 1, 2, 4)$.
Differentiating (E2) with respect to $V$ gives
\begin{align*}
&2\big(T(U,V,W)\, U-V\big)\ \left[\pdiff{T}{V}(U,V,W)\ U-1\right]
            \ln\big(W-U\,V\big) \\
&\hskip2.5in
  -\big(T(U,V,W)\, U-V\big)^2 \frac{U}{W-U\,V} = 0
\end{align*}
In particular, for $(T, U, V, W)=(1, 1, 2, 4)$,
\begin{align*}
&2\big((1)(1)-2\big)
         \left[ (1)\pdiff{T}{V}(1,2,4)-1\right] 
            \ln\big(4-(1)(2)\big) \\
&\hskip2.5in
  -\big((1)(1)-2\big)^2 \frac{1}{4-(1)(2)} = 0
\end{align*}
This simplifies to
\begin{align*}
-2\left[\pdiff{T}{V}(1,2,4)-1\right] \ln(2) -\frac{1}{2}=0 
\implies
\pdiff{T}{V}(1,2,4) = 1 -\frac{1}{4\ln(2)} %\approx 0.639
\end{align*}
\end{solution}

\begin{question}[M200 2013D] %1c
Suppose that $u = x^2 + yz$, $x = \rho r \cos(\theta)$, 
$y = \rho r \sin(\theta)$ and $z = \rho r$. Find $\pdiff{u}{r}$
at the point $(\rho_0 , r_0 , \theta_0) = (2, 3, \pi/2)$.
\end{question}

%\begin{hint}
%
%\end{hint}

\begin{answer}
$24$
\end{answer}

\begin{solution}
The function
\begin{align*}
u(\rho, r,\theta) &= \big[\rho r\cos\theta\big]^2 
                      +\big[\rho r\sin\theta\big] \rho r \\
&=\rho^2 r^2\cos^2\theta +\rho^2 r^2\sin\theta
\end{align*}
So 
\begin{align*}
\pdiff{u}{r}(\rho, r,\theta) 
&=2 \rho^2 r\cos^2\theta +2 \rho^2 r\sin\theta
\end{align*}
and
\begin{align*}
\pdiff{u}{r}(2, 3,\pi/2) 
&=2 (2^2) (3) (0)^2 +2 (2^2) (3) (1)
=24
\end{align*}
\end{solution}

%%%%%%%%%%%%%%%%%%%%%%%%%%%%%%%%
\begin{question}
Use the definition of the derivative to evaluate $f_x(0,0)$ and $f_y(0,0)$ for
\begin{equation*}
f(x,y)=\begin{cases}
                \frac{x^2-2y^2}{x-y}&\text{if $x\ne y$}\\ 
                0&\text{if $x=y$} \end{cases}
\end{equation*}
\end{question}

%\begin{hint}
%
%\end{hint}

\begin{answer}
$f_x(0,0)=1$,\qquad
$f_y(0,0)=2$
\end{answer}

\begin{solution}
By definition
\begin{equation*}
f_x(x_0,y_0)=\lim_{\De x\rightarrow 0}\frac{f(x_0+\De x,y_0)-f(x_0,y_0)}{\De x}
\qquad
f_y(x_0,y_0)=\lim_{\De y\rightarrow 0}\frac{f(x_0,y_0+\De y)-f(x_0,y_0)}{\De y}
\end{equation*}
Setting $x_0=y_0=0$,
\begin{alignat*}{5}
f_x(0,0)&=\lim_{\De x\rightarrow 0}\frac{f(\De x,0)-f(0,0)}{\De x}
&=\lim_{\De x\rightarrow 0}\frac{f(\De x,0)}{\De x}
&=\lim_{\De x\rightarrow 0}\frac{((\De x)^2-2\times0^2)/(\De x-0)}{\De x} \\
&=\lim_{\De x\rightarrow 0}1
=1\\
f_y(0,0)&=\lim_{\De y\rightarrow 0}\frac{f(0,\De y)-f(0,0)}{\De y}
&=\lim_{\De y\rightarrow 0}\frac{f(0,\De y)}{\De y}
&=\lim_{\De y\rightarrow 0}\frac{(0^2-2(\De y)^2)/(0-\De y)}{\De y} \\
&=\lim_{\De y\rightarrow 0}2
=2
\end{alignat*}

\end{solution}


%%%%%%%%%%%%%%%%%%
\subsection*{\Application}
%%%%%%%%%%%%%%%%%%

%%%%%%%%%%%%%%%%%%%%%%%%%%%%%%%%
\begin{question}
Let $f$ be any differentiable function of one variable. Define 
$z(x,y)=f(x^2+y^2)$. Is the equation 
\begin{equation*}
y\pdiff{z}{x}(x,y)-x\pdiff{z}{y}(x,y) = 0
\end{equation*}
necessarily satisfied? 
\end{question}

\begin{hint}
Just evaluate $y\pdiff{z}{x}(x,y)$ and $x\pdiff{z}{y}(x,y)$.
\end{hint}

\begin{answer}
Yes.
\end{answer}

\begin{solution}
As $z(x,y)=f(x^2+y^2)$
\begin{align*}
\pdiff{z}{x}(x,y)&=2xf'(x^2+y^2) \\
\pdiff{z}{y}(x,y)&=2yf'(x^2+y^2)
\end{align*}
by the (ordinary single variable) chain rule.
So
\begin{equation*}
y\pdiff{z}{x}-x\pdiff{z}{y}
=y(2x)f'(x^2+y^2)-x(2y)f'(x^2+y^2)=0
\end{equation*}
and the differential equation is always satisfied, assuming that
$f$ is differentiable, so that the chain rule applies.
\end{solution}

\begin{question}
Define the function
\begin{equation*}
f(x,y)=\begin{cases}\frac{(x+2y)^2}{x+y}& \text{if $x+y\ne 0$} \\
                     \noalign{\vskip.05in}
                        0 &\text{if $x+y=0$}
       \end{cases}
\end{equation*}
\begin{enumerate}[(a)]
\item
Evaluate, if possible, $\pdiff{f}{x}(0,0)$ and 
$\pdiff{f}{y}(0,0)$.
\item
 Is $f(x,y)$ continuous at $(0,0)$? 
\end{enumerate}
\end{question}

%\begin{hint}
%
%\end{hint}

\begin{answer}
(a) $\pdiff{f}{x}(0,0)=1$, $\pdiff{f}{y}(0,0)=4$\qquad
(b) Nope.
\end{answer}

\begin{solution}
By definition
\begin{align*}
\pdiff{f}{x}(0,0)
&=\lim_{\De x\rightarrow 0}\frac{f(\De x,0)-f(0,0)}{\De x} \\
&=\lim_{\De x\rightarrow 0}\frac{\frac{(\De x+2\times 0)^2}{\De x+0}-0}{\De x}   \\
&=\lim_{\De x\rightarrow 0}\frac{\De x}{\De x} \\
&=1
\end{align*}
and
\begin{align*}
\pdiff{f}{y}(0,0)
&=\lim_{\De y\rightarrow 0}\frac{f(0,\De y)-f(0,0)}{\De y} \\
&=\lim_{\De y\rightarrow 0}\frac{\frac{(0+2\De y)^2}{0+\De y}-0}{\De y} \\
&=\lim_{\De y\rightarrow 0}\frac{4\De y}{\De y} \\
&=4
\end{align*}

(b) $f(x,y)$ is \emph{not continuous} at $(0,0)$, even though both partial
derivatives exist there. To see this, make a change of coordinates from
$(x,y)$ to $(X,y)$ with $X=x+y$ (the denominator). Of course,
$(x,y)\rightarrow (0,0)$ if and only if $(X,y)\rightarrow (0,0)$.
Now watch what happens when $(X,y)\rightarrow(0,0)$ with $X$ a lot smaller
than $y$. For example, $X=ay^2$. Then
\begin{align*}
\frac{(x+2y)^2}{x+y}=\frac{(X+y)^2}{X}=\frac{(ay^2+y)^2}{ay^2}
=\frac{(1+ay)^2}{a}\rightarrow\frac{1}{a}
\end{align*}
This depends on $a$. So approaching $(0,0)$ along different paths gives
different limits. (You can see the same effect without changing coordinates
by sending $(x,y)\rightarrow (0,0)$ with $x=-y+ay^2$.) Even more dramatically,
watch what happens when $(X,y)\rightarrow(0,0)$ with  $X=y^3$. Then
\begin{equation*}
\frac{(x+2y)^2}{x+y}=\frac{(X+y)^2}{X}=\frac{(y^3+y)^2}{y^3}
=\frac{{(1+y^2)}^2}{y}\rightarrow\pm\infty
\end{equation*}

\end{solution}

%%%%%%%%%%%%%%%%%%%
\begin{question}
Consider the cylinder whose base is the radius-1 circle in the $xy$-plane 
centred at $(0,0)$, and which slopes parallel to the line in the $yz$-plane
given by $z=y$.
\begin{center}
\begin{tikzpicture}
\draw[thin, gray] (0,0)--(1,0) (0,0)--(0,1) (0,0)--(-.2,-.2);
\draw (0,1)--(0,2) (1,0)--(3,0) (-.2,-.2)--(-1,-1);
\draw (2,2) node[shape=ellipse, minimum width=2cm, minimum height=.4cm,draw,fill=gray,fill opacity=0.3]{};
\draw[thick,blue] (-3,-2)--(1,2) (-1,-2)--(3,2);
\draw[thick,blue] (-1,0) arc (180:360:1cm and .2cm);
\draw[dotted,thick](-1,0) arc (180:0:1cm and .2cm);
\draw[thick,blue] (-3,-2) arc (180:360:1cm and .2cm);
\draw[dotted,thick](-3,-2) arc (180:0:1cm and .2cm);
\end{tikzpicture}
\end{center}
When you stand at the point $(0,-1,0)$, what is the slope of the surface if you look in the positive $y$ direction? The positive $x$ direction?
\end{question}
\begin{hint}
You can find an equation for the surface, or just look at the diagram.
\end{hint}
\begin{answer}
1 resp. 0
\end{answer}
\begin{solution}
\textbf{Solution 1}\\
Let's start by finding an equation for this surface. Every level curve 
is a horizontal circle of radius one, so the equation should be of the form
\begin{equation*}
(x-f_1)^2+(y-f_2)^2=1
\end{equation*}
where $f_1$ and $f_2$ are functions depending only on $z$. Since the centre of the circle at height $z$ is at position $x=0$, $y=z$, we see that the equation of our surface is
\begin{equation*}
x^2+(y-z)^2=1
\end{equation*}
The height of the surface at the point $(x,y)$ is the $z(x,y)$ found by solving
that equation. That is, 
\begin{equation*}
x^2+\big(y-z(x,y)\big)^2=1
\tag{$*$}
\end{equation*}
We differentiate this equation implicitly to find $z_x(x,y)$ and $z_y(x,y)$ at the desired point $(x,y)= (0,-1)$. First, differentiating $(*)$ with respect 
to $y$ gives
\begin{align*}
0+2\big(y-z(x,y)\big)\big(1- z_y(x,y)\big)&=0 \\
%2\big(y-z(x,y)\big)\big(1-z_y(x,y)\big)&=0\\
2(-1-0)\big(1-z_y(0,-1)\big)&=0& & \mbox{ at } (0,-1,0)
\end{align*}
so that the slope looking in the positive $y$ direction is $z_y(0,-1)=1$.
Similarly, differentiating $(*)$ with respect to $x$ gives
\begin{align*}
2x+2\big(y-z(x,y)\big)\cdot\big(0-z_x(x,y)\big)&=0 \\
2x&=2\big(y-z(x,y)\big)\cdot z_x(x,y)\\
z_x(x,y)&=\frac{x}{y-z(x,y)}\\
z_x(0,-1)&=0 &\mbox{ at } (0,-1,0)
\end{align*}
The slope looking in the positive $x$ direction is $z_x(0,-1)=0$.

\textbf{Solution 2}\\
Standing at $(0,-1,0)$ and looking in the positive $y$ direction, 
the surface follows the straight line that 
\begin{itemize}
\item 
passes through the point $(0,-1,0)$, and  
\item
is parallel to the central line $z=y, x=0$ of the cylinder.
\end{itemize}
Shifting the central line one unit in the $y$-direction, we get the line $z=y+1$, $x=0$. (As a check, notice that $(0,-1,0)$ is indeed on $z=y+1$,  
$x=0$.) The slope of this line is 1.

Standing at $(0,-1,0)$ and looking in the positive $x$ direction, 
the surface follows the circle $x^2+y^2=1$, $z=0$, which is the intersection 
of the cylinder with the $xy$-plane. As we move along that circle our $z$ coordinate stays fixed at $0$. So the slope in that direction is 0.

%%At the point $(0,-1)$ on the unit circle in the $xy$-plane, the vector 
% $\llt 0,1 \rgt$ is tangent to the unit circle. At this point, 
% $\frac{dy}{dx}=0$. So, if we move an infinitesimal amount in the positive 
% $x$-direction, our $y$ coordinate stays the same. That means our $z$ 
% coordinate stays the same as well. On our surface, at the point $(0,-1,0)$, 
% moving in the $x$-direction momentarily keeps us on the level curve $z=0$, so % the slope in that direction is 0.

\end{solution}
  

