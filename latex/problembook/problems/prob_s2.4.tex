%\documentclass[12pt]{article}

\questionheader{ex:s2.4}


%%%%%%%%%%%%%%%%%%
\subsection*{\Conceptual}
%%%%%%%%%%%%%%%%%%

%%%%%%%%%%%%%%%%%%%%%%%%%%%%%%%%
\begin{question}
Write out the chain rule
for each of the following functions.
\begin{enumerate}[(a)]
\item
   $\pdiff{h}{x}$ for $h(x,y)=f\big(x,u(x,y)\big)$
\item
   $\diff{h}{x}$ for $h(x)=f\big(x,u(x),v(x)\big)$
\item
   $\pdiff{h}{x}$ for $h(x,y,z)=f\big(u(x,y,z),v(x,y),w(x)\big)$
\end{enumerate}
\end{question}

\begin{hint}
 Review   \S\eref{CLP200}{subsec memory aid} in the CLP-3 text.
\end{hint}

\begin{answer}
(a) $\pdiff{h}{x}(x,y)
=\pdiff{f}{x}\big(x,u(x,y)\big)
+\pdiff{f}{u}\big(x,u(x,y)\big)
\pdiff{u}{x}(x,y)$

(b) $\diff{h}{x}(x)
=\pdiff{f}{x}\big(x,u(x),v(x)\big)
+\pdiff{f}{u}\big(x,u(x),v(x)\big)
\diff{u}{x}(x)
+\pdiff{f}{v}\big(x,u(x),v(x)\big)
\diff{v}{x}(x)$

(c) $\pdiff{h}{x}(x,y,z)
=\pdiff{f}{u}\big(u(x,y,z),v(x,y),w(x)\big)
\pdiff{u}{x}(x,y,z)
+\pdiff{f}{v}\big(u(x,y,z),v(x,y),w(x)\big)
\pdiff{v}{x}(x,y)$ 
\null$\ \ \ \ \ \ \ \ +\pdiff{f}{w}\big(u(x,y,z),v(x,y),w(x)\big)
\diff{w}{x}(x)$


\end{answer}

\begin{solution}

(c) We'll start with part (c) and follow the procedure given in 
\S\eref{CLP200}{subsec memory aid} in the CLP-3 text. We are to compute the
derivative of $h(x,y,z)=f\big(u(x,y,z),v(x,y),w(x)\big)$ with respect to 
$x$. For this function, the template of Step 2 in 
\S\eref{CLP200}{subsec memory aid} is
\begin{equation*}
\pdiff{h}{x}=\frac{\partial f}{ }\frac{ }{\partial x}
\end{equation*}
Note that 
\begin{itemize}
\item
The function $h$ appears once in the numerator on the left.
The function $f$, from which $h$ is constructed by a change of variables,
appears once in the numerator on the right.

\item
The variable, $x$, in the denominator on the left appears once in the
denominator on the right. 
\end{itemize}
Now we fill in the blanks with every variable that makes sense. In 
particular, since $f$ is a function of $u$, $v$ and $w$, it may only be 
differentiated with respect to $u$, $v$ and $w$. So we add together
three copies of our template --- one for each of $u$, $v$ and $w$:
\begin{align*}
\pdiff{h}{x}=\pdiff{f}{u}\pdiff{u}{x}
             +\pdiff{f}{v}\pdiff{v}{x}
             +\pdiff{f}{w}\diff{w}{x}
\end{align*}
Since $w$ is a function of only one variable, we use the ordinary 
derivative symbol $\diff{w}{x}$, rather than the partial
derivative symbol $\pdiff{w}{x}$ in the third copy.
Finally we put in the only functional dependence that makes sense. 
The left hand side is a function of $x$, $y$ and $z$, because $h$
is a function of $x$, $y$ and $z$. Hence the right hand side must 
also be a function of $x$, $y$ and $z$. As $f$ is a
function of $u$, $v$ and $w$, this is achieved by evaluating $f$ at 
$u=u(x,y,z)$, $v=v(x,y)$ and $w=w(x)$.
\begin{align*}
\pdiff{h}{x}(x,y,z)
&=\pdiff{f}{u}\big(u(x,y,z),v(x,y),w(x)\big)
\pdiff{u}{x}(x,y,z)
+\pdiff{f}{v}\big(u(x,y,z),v(x,y),w(x)\big)
\pdiff{v}{x}(x,y) \\&\hskip2in
+\pdiff{f}{w}\big(u(x,y,z),v(x,y),w(x)\big)
\diff{w}{x}(x)
\end{align*}

(a)
We again follow the procedure given in 
\S\eref{CLP200}{subsec memory aid} in the CLP-3 text. We are to compute the
derivative of $h(x,y)=f\big(x,u(x,y)\big)$ with respect to 
$x$. For this function, the template of Step 2 in 
\S\eref{CLP200}{subsec memory aid} is
\begin{equation*}
\pdiff{h}{x}=\frac{\partial f}{ }\frac{ }{\partial x}
\end{equation*}
Now we fill in the blanks with every variable that makes sense. In 
particular, since $f$ is a function of $x$ and $u$, it may only be 
differentiated with respect to $x$, and $u$. So we add together
two copies of our template --- one for $x$ and one for $u$:
\begin{align*}
\pdiff{h}{x}=\pdiff{f}{x}\diff{x}{x}
             +\pdiff{f}{u}\pdiff{u}{x}
\end{align*}
In $\diff{x}{x}$ we are to differentiate the (explicit) function $x$
(i.e. the function $F(x)=x$) with respect to $x$. The answer is of 
course $1$. So 
\begin{align*}
\pdiff{h}{x}=\pdiff{f}{x}
             +\pdiff{f}{u}\pdiff{u}{x}
\end{align*} 
Finally we put in the only functional depedence that makes sense. 
The left hand side is a function of $x$, and $y$, because $h$
is a function of $x$ and $y$. Hence the right hand side must 
also be a function of $x$ and $y$. As $f$ is a
function of $x$, $u$, this is achieved by evaluating $f$ at 
$u=u(x,y)$.
\begin{align*}
\pdiff{h}{x}(x,y)
=\pdiff{f}{x}\big(x,u(x,y)\big)
+\pdiff{f}{u}\big(x,u(x,y)\big)
\pdiff{u}{x}(x,y)
\end{align*}


(b) Yet again we follow the procedure given in 
\S\eref{CLP200}{subsec memory aid} in the CLP-3 text. We are to compute the
derivative of $h(x)=f\big(x,u(x),v(x)\big)$ with respect to 
$x$. For this function, the template of Step 2 in 
\S\eref{CLP200}{subsec memory aid} is
\begin{equation*}
\diff{h}{x}=\frac{\partial f}{ }\frac{ }{\partial x}
\end{equation*}
(As $h$ is function of only one variable, we use the ordinary derivative
symbol $\diff{h}{x}$ on the left hand side.)
Now we fill in the blanks with every variable that makes sense. In 
particular, since $f$ is a function of $x$, $u$ and $v$, it may only be 
differentiated with respect to $x$, $u$ and $v$. So we add together
three copies of our template --- one for each of $x$, $u$ and $v$:
\begin{align*}
\diff{h}{x}&=\pdiff{f}{x}\diff{x}{x}
             +\pdiff{f}{u}\diff{u}{x}
             +\pdiff{f}{v}\diff{v}{x} \\
            &=\pdiff{f}{x}
             +\pdiff{f}{u}\diff{u}{x}
             +\pdiff{f}{v}\diff{v}{x}
\end{align*}
Finally we put in the only functional depedence that makes sense. 
\begin{align*}
\diff{h}{x}(x)
=\pdiff{f}{x}\big(x,u(x),v(x)\big)
+\pdiff{f}{u}\big(x,u(x),v(x)\big)
\diff{u}{x}(x)
+\pdiff{f}{v}\big(x,u(x),v(x)\big)
\diff{v}{x}(x)
\end{align*}

\end{solution}
%%%%%%%%%%%%%%%%%%%%%%%%%%%%%%%%
\begin{question}A piece of the surface $z=f(x,y)$ is shown below for some continuously differentiable function $f(x,y)$. The level curve $f(x,y)=z_1$ is marked with a blue line. The three points $P_0$, $P_1$, and $P_2$ lie on the surface.
	
\includegraphics{partialTotalX.pdf}
	
	On the level curve $z=z_1$, we can think of $y$ as a function of $x$. Let $w(x)=f(x,y(x))=z_1$. We approximate, at $P_0$, $f_x(x,y) \approx \frac{\Delta f}{\Delta x}$ and $\frac{dw}{dx}(x)\approx\frac{\Delta w}{\Delta x}$. Identify the quantities $\Delta f$, $\Delta w$, and $\Delta x$ from the diagram.% Describe also when $\Delta y=0$ or $\Delta z =0$.
	
\end{question}
\begin{hint}
%text ex 2.4.10
This is a visualization, in a simplified setting, of Example~\eref{CLP3}{eg:chainRuleE} in CLP3.
\end{hint}
\begin{answer}
	
\textit{To visualize, in a simplified setting, the situation from Example~\eref{CLP3}{eg:chainRuleE} in CLP3, note that $w'(x)$ is the rate of change of $z$ as we slide along the blue line, while $f_x(x,y)$ is the change of $z$ as we slide along the orange line.}
	
In the approximation $f_x(x,y)\approx \frac{\Delta f}{\Delta x}$, starting at the point $P_0$, $\Delta x=x_2-x_1$ and $\Delta f=z_2-z_1$.

In the approximation $\diff{w}{x}\approx \frac{\Delta w}{\Delta x}$, starting at the point $P_0$,  $\Delta x=x_2-x_1$ again, and $\Delta w=z_1-z_1=0$.

\end{answer}
\begin{solution}
	
\textit{To visualize, in a simplified setting, the situation from Example~\eref{CLP3}{eg:chainRuleE} in CLP3, note that $w'(x)$ is the rate of change of $z$ as we slide along the blue line, while $f_x(x,y)$ is the change of $z$ as we slide along the orange line.}

In the partial derivative $f_x(x,y)\approx \frac{\Delta f}{\Delta x}$, we let $x$ change, while $y$ stays the same. Necessarily, that forces $f$ to change as well. Starting at point $P_0$, if we move $x$ but keep $y$ fixed, we end up at $P_2$. According to the labels on the diagram, $\Delta x$ is $x_2-x_1$, and $\Delta f$ is $z_2-z_1$.


The function $w(x)$ is a constant function, so we expect $w'(x)=0$.
In the approximation $\diff{w}{x}\approx \frac{\Delta w}{\Delta x}$, we let $x$ change, but $w$ stays the same. Necessarily, to stay on the surface, this
forces $y$ to change. Starting at point $P_0$, if we move $x$ but keep 
$z=f(x,y)$ fixed, we end up at $P_1$. According to the labels on the diagram, $\Delta x$ is $x_2-x_1$ again, and $\Delta w=z_1-z_1=0$.

To compare the two situations, note the first case has $\Delta y =0$ while 
the second case has $\Delta f =0$.
\end{solution}
%%%%%%%%%%%%%%%%%%%%%%%%%%%%%%%%
\begin{question}[M200 2000D] %8
Let $w=f(x,y,t)$ with $x$ and $y$ depending on $t$. Suppose
that at some point $(x,y)$ and at some time $t$, the partial derivatives
$f_x$, $f_y$ and $f_t$ are equal to $2$, $-3$ and $5$ respectively, while
$\diff{x}{t}=1$ and $\diff{y}{t}=2$. Find and explain the difference
between $\diff{w}{t}$ and $f_t$.
\end{question}

\begin{hint}
Pay attention to which variables change, and which are held fixed, in each context.
\end{hint}

\begin{answer}
$\diff{w}{t}=1$ and $f_t=5$. 
$f_t$ gives the rate of change of $f(x,y,t)$ as $t$ varies 
while $x$ and $y$ are held fixed. 
$\diff{w}{t}$ gives the rate of change of $f\big(x(t),y(t),t\big)$.
For the latter all of $x=x(t)$, $y=y(t)$ and $t$ are 
changing at once.
\end{answer}

\begin{solution}
We are told in the statement of the question that 
$w(t)= f\big(x(t),y(t),t\big)$.
Applying the chain rule to
$
w(t)= f\big(x(t),y(t),t\big),
$
by following the procedure given in 
\S\eref{CLP200}{subsec memory aid} in the CLP-3 tex, gives
\begin{align*}
\diff{w}{t}(t)
    &=\pdiff{f}{x}\big(x(t),y(t),t\big)\diff{x}{t}(t)
    +\pdiff{f}{y}\big(x(t),y(t),t\big)\diff{y}{t}(t)
    +\pdiff{f}{t}\big(x(t),y(t),t\big)\diff{t}{t} \\
   &=\pdiff{f}{x}\big(x(t),y(t),t\big)\diff{x}{t}(t)
    +\pdiff{f}{y}\big(x(t),y(t),t\big)\diff{y}{t}(t)
    +\pdiff{f}{t}\big(x(t),y(t),t\big)
\end{align*}
Substituting in the values given in the question
\begin{equation*}
\diff{w}{t}
    =2\times 1
    -3\times 2
    +5
    =1
\end{equation*}
On the other hand, we are told explicitly in the question that $f_t$ is $5$.
The reason that $f_t$ and $\diff{w}{t}$ are different is that
\begin{itemize}\itemsep1pt \parskip0pt \parsep0pt %\itemindent-15p
\item
$f_t$ gives the rate of change of $f(x,y,t)$ as $t$ varies 
while $x$ and $y$ are held fixed, but
\item
$\diff{w}{t}$ gives the rate of change of $f\big(x(t),y(t),t\big)$.
For the latter all of $x=x(t)$, $y=y(t)$ and $t$ are 
changing at once.
\end{itemize}
\end{solution}

%%%%%%%%%%%%%%%%%%%%%%%%%%%%%%%%
\begin{question}
Thermodynamics texts
use the relationship 
\begin{equation*}
\left(\pdiff{y}{x}\right)
\left(\pdiff{z}{y}\right)
\left(\pdiff{x}{z}\right)=-1
\end{equation*}
Explain the meaning of this equation and prove that it is true.
\end{question}

\begin{hint}
The basic assumption is that the three quantites $x$, $y$
and $z$ are not independent. Given any two of them, the third is uniquely
determined. They are assumed to satisfy a relationship
$F(x,y,z)=0$, which can be solved to
\begin{itemize}\itemsep1pt \parskip0pt \parsep0pt %\itemindent-15p
\item
determine $x$ as a function of $y$ and $z$ (say $x=f(y,z)$) and 
can alternatively be solved to
\item
determine $y$ as a function of $x$ and $z$ (say $y=g(x,z)$) and 
can alternatively be solved to
\item
determine $z$ as a function of $x$ and $y$
(say $z=h(x,y)$). 
\end{itemize}
For example, saying that  $F(x,y,z)=0$ determines $x=f(y,z)$ means that 
\begin{equation*}
F\big(f(y,z),y,z\big)=0
\tag{$*$}
\end{equation*}
for all $y$ and $z$. The equation 
\begin{equation*}
\left(\pdiff{y}{x}\right)
\left(\pdiff{z}{y}\right)
\left(\pdiff{x}{z}\right)=-1
\end{equation*}
really means
\begin{equation*}
\left(\pdiff{g}{x}\right)
\left(\pdiff{h}{y}\right)
\left(\pdiff{f}{z}\right)=-1
\end{equation*}
So use $(*)$ to compute $\pdiff{f}{z}$.  Use other equations similar to $(*)$
to compute $\pdiff{g}{x}$ and $\pdiff{h}{y}$.
\end{hint}

\begin{answer}
See the solution.
\end{answer}

\begin{solution}
 The basic assumption is that the three quantites $x$, $y$
and $z$ are not independent. Given any two of them, the third is uniquely
determined. They are assumed to satisfy a relationship
$F(x,y,z)=0$, which can be solved to
\begin{itemize}\itemsep1pt \parskip0pt \parsep0pt %\itemindent-15p
\item
determine $x$ as a function of $y$ and $z$ (say $x=f(y,z)$) and 
can alternatively be solved to
\item
determine $y$ as a function of $x$ and $z$ (say $y=g(x,z)$) and 
can alternatively be solved to
\item
determine $z$ as a function of $x$ and $y$
(say $z=h(x,y)$). 
\end{itemize}
As an example, if $F(x,y,z) = xyz-1$, then 
\begin{itemize}\itemsep1pt \parskip0pt \parsep0pt %\itemindent-15p
\item
$F(x,y,z)=xyz-1=0$ implies that $x=\frac{1}{yz}=f(y,z)$ and
\item
$F(x,y,z)=xyz-1=0$ implies that $y=\frac{1}{xz}=g(x,z)$ and
\item
$F(x,y,z)=xyz-1=0$ implies that $z=\frac{1}{xy}=h(x,y)$
\end{itemize}
In general, saying that  $F(x,y,z)=0$ determines $x=f(y,z)$ means that 
\begin{equation*}
F\big(f(y,z),y,z\big)=0
\tag{$*$}
\end{equation*}
for all $y$ and $z$. 
Set $\mathcal{F}(y,z)=F\big(f(y,z),y,z\big)$. Applying the chain rule to
$\mathcal{F}(y,z)=F\big(f(y,z),y,z\big)$ (with $y$ and $z$ independent variables) 
gives
\begin{align*}
\pdiff{\mathcal{F}}{z}(y,z) 
%&= \pdiff{F}{x}\big(f(y,z),y,z\big)\pdiff{f}{z}(y,z)
%+\pdiff{F}{z}\big(f(y,z),y,z\big)\diff{z}{z} \\
&= 
\pdiff{F}{x}\big(f(y,z),y,z\big)
\pdiff{f}{z}(y,z)
+\pdiff{F}{z}\big(f(y,z),y,z\big)
\end{align*}
The equation $(*)$ says that $\mathcal{F}(y,z)=F\big(f(y,z),y,z\big)=0$ 
for all $y$ and $z$.
So differentiating the equation $(*)$ with respect to 
$z$ gives
\begin{align*}
\pdiff{\mathcal{F}}{z}(y,z)&=\pdiff{F}{x}\big(f(y,z),y,z\big)
\pdiff{f}{z}(y,z)
+\pdiff{F}{z}\big(f(y,z),y,z\big)=0 \\
&\implies
\pdiff{f}{z}(y,z)
=-\frac{\pdiff{F}{z}\big(f(y,z),y,z\big)}
{\pdiff{F}{x}\big(f(y,z),y,z\big)}
\end{align*}
for all $y$ and $z$. Similarly, differentiating 
$F\big(x,g(x,z),z\big)=0$ with respect to $x$ and
$F\big(x,y,h(x,y)\big)=0$ with respect to $y$ gives
\begin{align*}
\pdiff{g}{x}(x,z)
=-\frac{\pdiff{F}{x}\big(x,g(x,z),z\big)}
{\pdiff{F}{y}\big(x,g(x,z),z\big)}\qquad
\pdiff{h}{y}(x,y)
=-\frac{\pdiff{F}{y}\big(x,y,h(x,y)\big)}
{\pdiff{F}{z}\big(x,y,h(x,y)\big)}
\end{align*}
If $(x,y,z)$ is any point satisfying $F(x,y,z)=0$ 
(so that $x=f(y,z)$ and $y=g(x,z)$ and  $z=h(x,y)$), then
\begin{align*}
\pdiff{f}{z}(y,z)
=-\frac{\pdiff{F}{z}\big(x,y,z\big)}
{\pdiff{F}{x}\big(x,y,z\big)}\qquad
\pdiff{g}{x}(x,z)
=-\frac{\pdiff{F}{x}\big(x,y,z\big)}
{\pdiff{F}{y}\big(x,y,z\big)}\qquad
\pdiff{h}{y}(x,y)
=-\frac{\pdiff{F}{y}\big(x,y,z\big)}
{\pdiff{F}{z}\big(x,y,z\big)}
\end{align*}
and
\begin{align*}
\pdiff{f}{z}(y,z)\ 
\pdiff{g}{x}(x,z)\ 
\pdiff{h}{y}(x,y)
&=-\frac{\pdiff{F}{z}\big(x,y,z\big)}
{\pdiff{F}{x}\big(x,y,z\big)}\ 
\frac{\pdiff{F}{x}\big(x,y,z\big)}
{\pdiff{F}{y}\big(x,y,z\big)}\ 
\frac{\pdiff{F}{y}\big(x,y,z\big)} 
{\pdiff{F}{z}\big(x,y,z\big)} \\
&=-1
\end{align*}
\end{solution}


%%%%%%%%%%%%%%%%%%%%%%%%%%%%%%%%
\begin{question}
What is wrong with the following argument? Suppose that $w=f(x,y,z)$
and $z=g(x,y)$. By the chain rule,
\begin{align*}
\pdiff{w}{x}
=\pdiff{w}{x}\pdiff{x}{x}
+\pdiff{w}{y}\pdiff{y}{x}
+\pdiff{w}{z}\pdiff{z}{x}
=\pdiff{w}{x}
+\pdiff{w}{z}\pdiff{z}{x}
\end{align*}
Hence $0=\pdiff{w}{z}\pdiff{z}{x}$
and so $\pdiff{w}{z}=0$
or $\pdiff{z}{x}=0$.
\end{question}

\begin{hint}
Is the $\pdiff{w}{x}$ on the left hand side really
the same as the $\pdiff{w}{x}$ on the right hand side?
\end{hint}

\begin{answer}
The problem is that $\pdiff{w}{x}$ is used to represent
two completely different functions in the same equation.
See the solution for more details.
\end{answer}

\begin{solution}
The problem is that $\pdiff{w}{x}$ is used
to represent
two completely different functions in the same equation. The careful way
to write the equation is the following. Let $f(x,y,z)$
and $g(x,y)$ be continuously differentiable functions and define
$w(x,y)=f\big(x,y,g(x,y)\big)$. By the chain rule,
\begin{align*}
\pdiff{w}{x}(x,y)
&=\pdiff{f}{x}\big(x,y,g(x,y)\big)
\pdiff{x}{x}
+\pdiff{f}{y}\big(x,y,g(x,y)\big)
\pdiff{y}{x}
+\pdiff{f}{z}\big(x,y,g(x,y)\big)
\pdiff{g}{x}(x,y)\cr
&=\pdiff{f}{x}\big(x,y,g(x,y)\big)
+\pdiff{f}{z}\big(x,y,g(x,y)\big)
\pdiff{g}{x}(x,y)
\end{align*}
While $w(x,y)=f\big(x,y,g(x,y)\big)$, it is not true that
$\pdiff{w}{x}(x,y)
=\pdiff{f}{x}\big(x,y,g(x,y)\big)$.
For example, take $f(x,y,z)=x-z$ and $g(x,y)=x$. Then 
$w(x,y)=f\big(x,y,g(x,y)\big)=x-g(x,y)=0$ for all $(x,y)$, so that 
$\pdiff{w}{x}(x,y)=0$
while $\pdiff{f}{x}(x,y,z)=1$ for all $(x,y,z)$.
\end{solution}


%%%%%%%%%%%%%%%%%%%%%%%%%%%%
%\Instructions{Questions~\ref{prob_s1.0first} through \ref{prob_s1.0last} provide practice with.}
%%%%%%%%%%%%%%%%%%%%

%%%%%%%%%%%%%%%%%%
\subsection*{\Procedural}
%%%%%%%%%%%%%%%%%%

%%%%%%%%%%%%%%%%%%%%%%%%%%%%%%%%
\begin{question}
Use two methods (one using the chain rule) to evaluate 
$\pdiff{w}{s}$ and $\pdiff{w}{t}$ 
given that the function $w=x^2+y^2+z^2$, with 
$x=st,\ y=s\cos t$ and $z=s\sin t$.
\end{question}

\begin{hint}
To avoid the chain rule, write $w$ explicitly as a function of $s$ and $t$.
\end{hint}

\begin{answer}
$w_s(s,t)=2s(t^2+1)$\qquad
$w_t(s,t)=s^2(2t)$
\end{answer}

\begin{solution}
\emph{Method 1:}
Since $w(s,t)=x(s,t)^2+y(s,t)^2+z(s,t)^2$ with $x(s,t)=st$, 
$y(s,t)=s\cos t$ and $z(s,t)=s\sin t$ we can write out $w(s,t)$
explicitly:
\begin{align*}
w(s,t)&=(st)^2+(s\cos t)^2+(s\sin t)^2
      =s^2(t^2+1)\\
\implies w_s(s,t)&=2s(t^2+1)\qquad\text{and}\qquad
  w_t(s,t)=s^2(2t)
\end{align*}

\emph{ Method 2:}
The question specifies that $w(s,t)=x(s,t)^2+y(s,t)^2+z(s,t)^2$ with 
$x(s,t)=st$, $y(s,t)=s\cos t$ and $z(s,t)=s\sin t$. That is, 
$w(s,t) = W\big(x(s,t), y(s,t), z(s,t)\big)$
with $W(x,y,z)= x^2+y^2+z^2$. Applying the chain rule to $w(s,t) = W\big(x(s,t), y(s,t), z(s,t)\big)$ and noting that $\pdiff{W}{x}=2x$,
$\pdiff{W}{y}=2y$, $\pdiff{W}{z}=2z$, gives
\begin{align*}
 \pdiff{w}{s}(s,t)
 &= \pdiff{W}{x}\big(x(s,t), y(s,t), z(s,t)\big)\pdiff{x}{s}(s,t)
   + \pdiff{W}{y}\big(x(s,t), y(s,t), z(s,t)\big)\pdiff{y}{s}(s,t)\\&\hskip1in
   + \pdiff{W}{z}\big(x(s,t), y(s,t), z(s,t)\big)\pdiff{z}{s}(s,t) \\
 &=2 x(s,t)\ x_s(s,t)+2 y(s,t)\ y_s(s,t)+2 z(s,t)\ z_s(s,t) \\
                  &=2 (st)\ t+2 (s\cos t)\ \cos t+2 (s\sin t)\ \sin t\\
                 &=2 st^2+2 s\\
  \pdiff{w}{t}(s,t)
 &= \pdiff{W}{x}\big(x(s,t), y(s,t), z(s,t)\big)\pdiff{x}{t}(s,t)
   + \pdiff{W}{y}\big(x(s,t), y(s,t), z(s,t)\big)\pdiff{y}{t}(s,t)\\&\hskip1in
   + \pdiff{W}{z}\big(x(s,t), y(s,t), z(s,t)\big)\pdiff{z}{t}(s,t) \\
 &=2 x(s,t)\ x_t(s,t)+2 y(s,t)\ y_t(s,t)+2 z(s,t)\ z_t(s,t)\\
                 &=2 (st)\ s+2 (s\cos t)\ (-s\sin t)+2 (s\sin t)\ (s\cos t)\\
                 &=2 s^2t
\end{align*}
\end{solution}

%%%%%%%%%%%%%%%%%%%%%%%%%%%%%%%%
\begin{question}
Evaluate
 $\frac{\partial^3}{\partial x\partial y^2}f(2x+3y,xy)$
in terms of partial derivatives of $f$. 
You may assume that $f$ is a smooth function so that the 
Chain Rule and Clairaut's Theorem on the equality of the mixed partial derivatives apply.
\end{question}

\begin{hint}
Start by setting $F(x,y)=f(2x+3y,xy)$. It might also help to define $g(x,y)=2x+3y$ and $h(x,y)=xy$.
\end{hint}

\begin{answer}
We have
\begin{equation*}
\frac{\partial^3}{\partial x\partial y^2}f(2x+3y,xy)
=6f_{12}+2x\,f_{22}+18\,f_{111}+(9y+12x)\,f_{112}
+(6xy+2x^2)\,f_{122}+x^2y\,f_{222}
\end{equation*}
All functions on the right hand side have arguments $(2x+3y,xy)$.
The notation $f_{21}$, for example, means first differentiate with 
respect to the second argument and then differentiate with respect 
to the first argument.
\end{answer}

\begin{solution}
By definition,
\begin{equation*}
\frac{\partial^3}{\partial x\partial y^2}f(2x+3y,xy)
=\pdiff{}{x}\left[\pdiff{}{y}\left(\pdiff{}{y}f(2x+3y,xy)\right)
\right]
\end{equation*}
We'll compute the derivatives from the inside out.
Let's call $F(x,y)=f(2x+3y,xy)$ so that the innermost derivative is $G(x,y)=\pdiff{}{y}f(2x+3y,xy)=\pdiff{}{y}F(x,y)$. By the chain rule
\begin{align*}
G(x,y)=\pdiff{}{y}F(x,y)
      &=f_1(2x+3y,xy)\pdiff{}{y}(2x+3y)+f_2(2x+3y,xy)\pdiff{}{y}(xy) \\
      &= 3f_1(2x+3y,xy)+xf_2(2x+3y,xy)
\end{align*}
Here the subscript $1$ means take the partial derivative of $f$ with respect to
the first argument while holding the second argument fixed, and
 the subscript $2$ means take the partial derivative of $f$ with respect to
the second argument while holding the first argument fixed. 
Next call the middle derivative 
$H(x,y)=\pdiff{}{y}\left(\pdiff{}{y}f(2x+3y,xy)\right)$ so that
\begin{align*}
H(x,y) &= \pdiff{}{y} G(x,y) \\
       &=\pdiff{}{y}\Big(3f_1(2x+3y,xy)+xf_2(2x+3y,xy)\Big) \\
       &=3\pdiff{}{y}\Big(f_1(2x+3y,xy)\Big)
         +x\pdiff{}{y}\Big(f_2(2x+3y,xy)\Big)
\end{align*}
By the chain rule (twice),
\begin{align*}
\pdiff{}{y}\Big(f_1(2x+3y,xy)\Big)
      &=f_{11}(2x+3y,xy)\pdiff{}{y}(2x+3y)+f_{12}(2x+3y,xy)\pdiff{}{y}(xy) \\
      &= 3f_{11}(2x+3y,xy)+xf_{12}(2x+3y,xy) \\
\pdiff{}{y}\Big(f_2(2x+3y,xy)\Big)
      &=f_{21}(2x+3y,xy)\pdiff{}{y}(2x+3y)+f_{22}(2x+3y,xy)\pdiff{}{y}(xy) \\
      &= 3f_{21}(2x+3y,xy)+xf_{22}(2x+3y,xy) 
\end{align*}
so that
\begin{align*}
H(x,y)&=3\Big(3f_{11}(2x+3y,xy)+xf_{12}(2x+3y,xy) \Big) \\[-0.05in]
         &\hskip1.5in  +x\Big(3f_{21}(2x+3y,xy)+xf_{22}(2x+3y,xy)\Big) \\
      &=9f_{11}(2x+3y,xy)+6xf_{12}(2x+3y,xy)+x^2f_{22}(2x+3y,xy)
\end{align*}
In the last equality we used that $f_{21}(2x+3y,xy)=f_{12}(2x+3y,xy)$.
The notation
$f_{21}$ means first differentiate with respect to the second argument
and then differentiate with respect to the first argument. For example,
if $f(x,y)=e^{2y}\sin x$, then
\begin{align*}
f_{21}(x,y)
=\pdiff{}{x}
   \Big[\pdiff{}{y}\big(e^{2y}\sin x\big)\Big]
=\pdiff{}{x}
   \Big[2e^{2y}\sin x\Big]
=2e^y\cos x
\end{align*}
Finally, we get to 
\begin{align*}
\frac{\partial^3}{\partial x\partial y^2}f(2x+3y,xy)
&=\pdiff{}{x} H(x,y) \\
&=\pdiff{}{x}\Big(9f_{11}(2x+3y,xy)+6xf_{12}(2x+3y,xy)+x^2f_{22}(2x+3y,xy)\Big)\\
&=9\pdiff{}{x}\Big(f_{11}(2x+3y,xy)\Big)\\&\hskip0.1in 
  +6f_{12}(2x+3y,xy) +6x\pdiff{}{x}\Big(f_{12}(2x+3y,xy)\Big) \\&\hskip0.1in
  +2xf_{22}(2x+3y,xy)
  +x^2\pdiff{}{x}\Big(f_{22}(2x+3y,xy)\Big)
\end{align*}
By three applications of the chain rule
\begin{align*}
\frac{\partial^3}{\partial x\partial y^2}f(2x+3y,xy)
%&=\frac{\partial^2}{\partial x\partial y}
%                  \Big[3f_1(2x+3y,xy)+xf_2(2x+3y,xy)\Big]\cr
%&\hskip-.8in=\pdiff{}{x}
% \Big[9f_{11}(2x+3y,xy)+6xf_{12}(2x+3y,xy)+x^2f_{22}(2x+3y,xy)\Big]\cr
&=9\Big(2f_{111}+yf_{112}\Big)\\&\hskip0.1in
+6f_{12}+6x\Big(2f_{121}+yf_{122}\Big)\\&\hskip0.1in
+2xf_{22}+x^2\Big(2f_{221}+yf_{222}\Big)\\
&=6\,f_{12}+2x\,f_{22}+18\,f_{111}+(9y+12x)\,f_{112}
+(6xy+2x^2)\,f_{122}+x^2y\,f_{222}
\end{align*}
All functions on the right hand side have arguments $(2x+3y,xy)$.

\end{solution}



%%%%%%%%%%%%%%%%%%%%%%%%%%%%%%%%
\begin{question}
 Find all second order derivatives of $g(s,t)=f(2s+3t,3s-2t)$.
You may assume that $f(x,y)$ is a smooth function so that the 
Chain Rule and Clairaut's Theorem on the equality of the mixed partial derivatives apply.
\end{question}

%\begin{hint}
%\end{hint}

\begin{answer}
\begin{align*}
g_{ss}(s,t)&=4f_{11}(2s+3t,3s-2t)+12f_{12}(2s+3t,3s-2t)+9f_{22}(2s+3t,3s-2t)\\
g_{st}(s,t)&=6f_{11}(2s+3t,3s-2t)+5f_{12}(2s+3t,3s-2t)-6f_{22}(2s+3t,3s-2t)\\
g_{tt}(s,t)&=9f_{11}(2s+3t,3s-2t)-12f_{12}(2s+3t,3s-2t)+4f_{22}(2s+3t,3s-2t)
\end{align*}
Here $f_1$ denotes the partial derivative of $f$ with respect to its first
argument, $f_{12}$ is the result of first taking one partial derivative
of $f$ with respect to its first argument and then taking a partial derivative
with respect to its second argument, and so on.
\end{answer}

\begin{solution}
The given function is
\begin{equation*}
g(s,t)=f(2s+3t,3s-2t)
\end{equation*}
The first order derivatives are
\begin{align*}
g_s(s,t)&=2f_1(2s+3t,3s-2t)+3f_2(2s+3t,3s-2t)\\
g_t(s,t)&=3f_1(2s+3t,3s-2t)-2f_2(2s+3t,3s-2t)
\end{align*}
The second order derivatives are
\begin{align*}
g_{ss}(s,t)
      &=\pdiff{}{s}\Big(2f_1(2s+3t,3s-2t)+3f_2(2s+3t,3s-2t)\Big)\\
      &=2\Big(2f_{11}+3f_{12}\Big)
        +3\Big(2f_{21}+3f_{22}\Big)\\
      &=4f_{11}+6f_{12}+6f_{21}+9f_{22}\\
      &=4f_{11}(2s+3t,3s-2t)+12f_{12}(2s+3t,3s-2t)+9f_{22}(2s+3t,3s-2t)\\
g_{st}(s,t)&=\pdiff{}{t}\Big(2f_1(2s+3t,3s-2t)+3f_2(2s+3t,3s-2t)\Big)\\
     &=2\Big(3f_{11}-2f_{12}\Big)
        +3\Big(3f_{21}-2f_{22}\Big)\\
%     &=6f_{11}-4f_{12}+9f_{21}-6f_{22}\\
     &=6f_{11}(2s+3t,3s-2t)+5f_{12}(2s+3t,3s-2t)-6f_{22}(2s+3t,3s-2t)\\
g_{tt}(s,t)&=\pdiff{}{t}\Big(3f_1(2s+3t,3s-2t)-2f_2(2s+3t,3s-2t)\Big)\\
     &=3\Big(3f_{11}-2f_{12}\Big)
        -2\Big(3f_{21}-2f_{22}\Big)\\
%     &=9f_{11}-6f_{12}-6f_{21}+4f_{22}\\
     &=9f_{11}(2s+3t,3s-2t)-12f_{12}(2s+3t,3s-2t)+4f_{22}(2s+3t,3s-2t)
\end{align*}
Here $f_1$ denotes the partial derivative of $f$ with respect to its first
argument, $f_{12}$ is the result of first taking one partial derivative
of $f$ with respect to its first argument and then taking a partial derivative
with respect to its second argument, and so on.  

\end{solution}


%%%%%%%%%%%%%%%%%%%%%%%%%%%%%%%%
\begin{question}[M200 2005D] %2
Assume that $f(x,y)$ satisfies Laplace's equation 
$\frac{\partial^2 f}{\partial x^2}+\frac{\partial^2 f}{\partial y^2}=0$.
Show that this is also the case for the composite function 
$g(s,t) = f (s - t, s + t)$. That is, show that
$\frac{\partial^2 g}{\partial s^2}+\frac{\partial^2 g}{\partial t^2}=0$.
You may assume that $f(x,y)$ is a smooth function so that the 
Chain Rule and Clairaut's Theorem on the equality of the mixed partial derivatives apply.
\end{question}

\begin{hint}
Start by showing that, because
$\frac{\partial^2 f}{\partial x^2}+\frac{\partial^2 f}{\partial y^2}=0$,
the second derivative
$\frac{\partial^2g}{\partial s^2} = 2\frac{\partial^2f}{\partial x \partial y}$.
\end{hint}

\begin{answer}
See the solutions.
\end{answer}

\begin{solution}
By the chain rule,
\begin{align*}
\pdiff{g}{s}(s,t)
&=\pdiff{}{s} f (s - t, s + t) \\
&= \pdiff{f}{x}\big(s-t\,,\,s+t\big)\pdiff{}{s}\big(s-t\big) 
                 +\pdiff{f}{y}\big(s-t\,,\,s+t\big)\pdiff{}{s}\big(s+t\big) \\
&= \pdiff{f}{x}\big(s-t\,,\,s+t\big) 
                    +\pdiff{f}{y}\big(s-t\,,\,s+t\big) \\
%%
\frac{\partial^2 g}{\partial s^2}(s,t)
  &=\textcolor{blue}{\pdiff{}{s}\left[\pdiff{f}{x}\big(s-t\,,\,s+t\big)\right]}
  +\textcolor{red}{\pdiff{}{s}\left[\pdiff{f}{y}\big(s-t\,,\,s+t\big)\right]} \\
  &=\textcolor{blue}{ \frac{\partial^2 f}{\partial x^2}\big(s-t\,,\,s+t\big)
    +\frac{\partial^2 f}{\partial y\partial x}\big(s-t\,,\,s+t\big)} \\
  &\hskip0.2in
 +\textcolor{red}{\frac{\partial^2 f}{\partial x\partial y}\big(s-t\,,\,s+t\big)
    +\frac{\partial^2 f}{\partial y^2}\big(s-t\,,\,s+t\big)} \\
&=\left\{\frac{\partial^2 f}{\partial x^2}\big(s-t\,,\,s+t\big)
 + 2\frac{\partial^2 f}{\partial x\partial y}\big(s-t\,,\,s+t\big)
 +\frac{\partial^2 f}{\partial y^2}\big(s-t\,,\,s+t\big)\right\}
\end{align*}
and
\begin{align*}
\pdiff{g}{t}(s,t)&=\pdiff{}{t} f (s - t, s + t) \\
&= \pdiff{f}{x}\big(s-t\,,\,s+t\big)\pdiff{}{t}\big(s-t\big) 
                 +\pdiff{f}{y}\big(s-t\,,\,s+t\big)\pdiff{}{t}\big(s+t\big) \\
&= -\pdiff{f}{x}\big(s-t\,,\,s+t\big) 
                    +\pdiff{f}{y}\big(s-t\,,\,s+t\big) \\
%%
\frac{\partial^2 g}{\partial t^2}(s,t)
 &=-\textcolor{blue}{\pdiff{}{t}\left[\pdiff{f}{x}\big(s-t\,,\,s+t\big)\right]} 
 +\textcolor{red}{\pdiff{}{t}\left[\pdiff{f}{y}\big(s-t\,,\,s+t\big)\right]} \\
  &= -\textcolor{blue}{\Big[-\frac{\partial^2 f}{\partial x^2} 
                                              \big(s-t\,,\,s+t\big)
     +\frac{\partial^2 f}{\partial y\partial x}\big(s-t\,,\,s+t\big)\Big]} \\
  &\hskip0.2in
  +\textcolor{red}{\Big[-\frac{\partial^2 f}{\partial x\partial y}                                                       \big(s-t\,,\,s+t\big)
    +\frac{\partial^2 f}{\partial y^2}\big(s-t\,,\,s+t\big) \Big]} \\
&=\left\{\frac{\partial^2 f}{\partial x^2}\big(s-t\,,\,s+t\big)
 - 2\frac{\partial^2 f}{\partial x\partial y}\big(s-t\,,\,s+t\big)
 +\frac{\partial^2 f}{\partial y^2}\big(s-t\,,\,s+t\big)\right\}
\end{align*}
Suppressing the arguments
\begin{align*}
\frac{\partial^2 g}{\partial s^2} + \frac{\partial^2 g}{\partial t^2}
&=\left\{\frac{\partial^2 f}{\partial x^2}
 + 2\frac{\partial^2 f}{\partial x\partial y}
 +\frac{\partial^2 f}{\partial y^2}\right\}
 +\left\{\frac{\partial^2 f}{\partial x^2}
 - 2\frac{\partial^2 f}{\partial x\partial y}
 +\frac{\partial^2 f}{\partial y^2}\right\} \\
&=2\left[\frac{\partial^2 f}{\partial x^2}
 +\frac{\partial^2 f}{\partial y^2}\right] \\
&=0
\end{align*}
as desired.
\end{solution}

%%%%%%%%%%%%%%%%%%%%%%%%%%%%%%%%
\begin{question}[M200 2006A] %5
Let $z = f(x,y)$ where $x = 2s + t$ and $y = s - t$. Find the values 
of the constants $a$, $b$ and $c$ such that
\begin{equation*}
a\frac{\partial^2 z}{\partial x^2}
+b\frac{\partial^2 z}{\partial x\,\partial y}
+c\frac{\partial^2 z}{\partial y^2}
=\frac{\partial^2 z}{\partial s^2}
+\frac{\partial^2 z}{\partial t^2}
\end{equation*}
You may assume that $z = f(x,y)$ is a smooth function so that the 
Chain Rule and Clairaut's Theorem on the equality of the 
mixed partial derivatives apply.
\end{question}

\begin{hint}
The notation in the statement of this question is horrendous --- 
the symbol $z$ is used with two different meanings in one equation.
On the left hand side, it is a function of $x$ and $y$, 
and on the right hand side, it is a function of $s$ and $t$.  
Unfortunately that abuse of notation is also very common. 
Until you get used to it, undo this notation conflict by renaming the 
function of $s$ and $t$ to $F(s,t)$. That is,
$
F(s,t) = f\big(2s+t\,,\,s-t\big)
$.

Then, evaluate each term on the right-hand side of the equation.
\end{hint}

\begin{answer}
$a=5$ and $b=c=2$.
\end{answer}

\begin{solution}
The notation in the statement of this question is horrendous --- 
the symbol $z$ is used with two different meanings in one equation.
On the left hand side, it is a function of $x$ and $y$, 
and on the right hand side, it is a function of $s$ and $t$.  
Unfortunately that abuse of notation is also very common. 
Let us undo the notation conflict by renaming the function of $s$ and $t$ 
to $F(s,t)$. That is,
\begin{equation*}
F(s,t) = f\big(2s+t\,,\,s-t\big)
\end{equation*}
In this new notation, we are being asked to find $a$, $b$ and $c$ so that 
\begin{align*}
a\frac{\partial^2 f}{\partial x^2}
 +b\frac{\partial^2 f}{\partial x\partial y}
 +c\frac{\partial^2 f}{\partial y^2}
&=\frac{\partial^2 F}{\partial s^2} + \frac{\partial^2 F}{\partial t^2}
\end{align*}
with the arguments on the right hand side being $(s,t)$ and the
arguments on the left hand side being $\big(2s+t\,,\,s-t\big)$.

By the chain rule,
\begin{align*}
\pdiff{F}{s}(s,t)&= \pdiff{f}{x}\big(2s+t\,,\,s-t\big)\pdiff{}{s}(2s+t) 
                    +\pdiff{f}{y}\big(2s+t\,,\,s-t\big)\pdiff{}{s}(s-t) \\
&= 2\pdiff{f}{x}\big(2s+t\,,\,s-t\big) 
                    +\pdiff{f}{y}\big(2s+t\,,\,s-t\big) \displaybreak[0]\\
%%
\frac{\partial^2 F}{\partial s^2}(s,t)
&=\textcolor{blue}{2\pdiff{}{s}\left[\pdiff{f}{x}\big(2s+t\,,\,s-t\big)\right]}
  +\textcolor{red}{\pdiff{}{s}\left[\pdiff{f}{y}\big(2s+t\,,\,s-t\big)\right]}\\
&= \textcolor{blue}{4\frac{\partial^2 f}{\partial x^2}\big(2s+t\,,\,s-t\big)
 +         2\frac{\partial^2 f}{\partial y\partial x} \big(2s+t\,,\,s-t\big)} \\
  &\hskip0.2in
+\textcolor{red}{2\frac{\partial^2 f}{\partial x\partial y}
                                   \big(2s+t\,,\,s-t\big)
    +\frac{\partial^2 f}{\partial y^2}\big(2s+t\,,\,s-t\big)}
\end{align*}
and
\begin{align*}
\pdiff{F}{t}(s,t)&= \pdiff{f}{x}\big(2s+t\,,\,s-t\big)\pdiff{}{t}(2s+t) 
           +\pdiff{f}{y}\big(2s+t\,,\,s-t\big)\pdiff{}{t}(s-t)\\
&= \pdiff{f}{x}\big(2s+t\,,\,s-t\big) 
                    -\pdiff{f}{y}\big(2s+t\,,\,s-t\big) \displaybreak[0] \\
%%
\frac{\partial^2 F}{\partial t^2}(s,t)
 &=\textcolor{blue}{\pdiff{}{t}\left[\pdiff{f}{x}\big(2s+t\,,\,s-t\big)\right]} 
  \textcolor{red}{-\pdiff{}{t}\left[\pdiff{f}{y}\big(2s+t\,,\,s-t\big)\right]} \\
  &= \textcolor{blue}{\frac{\partial^2 f}{\partial x^2}\big(2s+t\,,\,s-t\big)
     -\frac{\partial^2 f}{\partial y\partial x}\big(2s+t\,,\,s-t\big)} \\
  &\hskip0.2in
     \textcolor{red}{-\frac{\partial^2 f}{\partial x\partial y}
                                          \big(2s+t\,,\,s-t\big)
    +\frac{\partial^2 f}{\partial y^2}\big(2s+t\,,\,s-t\big) }
\end{align*}
Suppressing the arguments
\begin{align*}
\frac{\partial^2 F}{\partial s^2} + \frac{\partial^2 F}{\partial t^2}
&=5\frac{\partial^2 f}{\partial x^2}
 +2\frac{\partial^2 f}{\partial x\partial y}
 +2\frac{\partial^2 f}{\partial y^2}
\end{align*}
Finally, translating back into the (horrendous) notation of the question
\begin{align*}
\frac{\partial^2 z}{\partial s^2} + \frac{\partial^2 z}{\partial t^2}
&=5\frac{\partial^2 z}{\partial x^2}
 +2\frac{\partial^2 z}{\partial x\partial y}
 +2\frac{\partial^2 z}{\partial y^2}
\end{align*}
so that $a=5$ and $b=c=2$.
\end{solution}

%%%%%%%%%%%%%%%%%%%%%%%%%%%%%%%%
\begin{question}[M200 2007A] %4
Let $F$ be a function on $\bbbr^2$. Denote points in $\bbbr^2$ by $(u, v)$ 
and the corresponding partial derivatives of $F$ by $F_u(u, v)$, 
$F_v (u, v)$, $F_{uu}(u, v)$, $F_{uv}(u, v)$, etc.. 
Assume those derivatives are all continuous. Express
\begin{align*}
\frac{\partial^2}{\partial x\, \partial y} F(x^2 - y^2 , 2xy)
\end{align*}
in terms of partial derivatives of the function $F$.
\end{question}

\begin{hint}
Let $u(x,y) = x^2 - y^2$ , and $v(x,y) = 2xy$. Then $F(x^2 - y^2 , 2xy)
=F\big(u(x,y),v(x,y)\big)$.
\end{hint}

\begin{answer}
\begin{align*}
\frac{\partial^2}{\partial x\, \partial y} F(x^2 - y^2 , 2xy)
&= 2\,  F_v(x^2 - y^2 , 2xy) -4xy\, F_{uu}(x^2 - y^2 , 2xy) \\&\hskip0.5in
  +4(x^2-y^2)\, F_{uv}(x^2 - y^2 , 2xy) \\&\hskip0.5in
  +4xy\,  F_{vv}(x^2 - y^2 , 2xy)
\end{align*}
\end{answer}

\begin{solution}
Let $u(x,y) = x^2 - y^2$ , and $v(x,y) = 2xy$. Then 
$F(x^2 - y^2 , 2xy) =F\big(u(x,y),v(x,y)\big)$.
By the chain rule
\begin{align*}
\pdiff{}{y}F(x^2 - y^2 , 2xy)
&=\pdiff{}{y}F(u(x,y) , v(xy)) \\
&=F_u(u(x,y) , v(xy))\pdiff{u}{y}(x,y) +
          F_v(u(x,y) , v(xy))\pdiff{v}{y}(x,y) \\
&=  F_u(x^2 - y^2 , 2xy)\ (-2y) + F_v(x^2 - y^2 , 2xy)\, (2x) \\
\frac{\partial^2}{\partial x\, \partial y} F(x^2 - y^2 , 2xy)
&=\pdiff{}{x}\left\{-2y F_u(x^2 - y^2 , 2xy)
                    + 2x  F_v(x^2 - y^2 , 2xy)
             \right\} \\
&=\textcolor{blue}{-2y\pdiff{}{x}\left[ F_u(x^2 - y^2 , 2xy)\right]}
   +2 F_v(x^2 - y^2 , 2xy) \\&\hskip0.5in
   +\textcolor{red}{2x \pdiff{}{x}\left[F_v(x^2 - y^2 , 2xy)\right]} \\
&= \textcolor{blue}{-4xy\,F_{uu}(x^2 - y^2 , 2xy) -4y^2 F_{uv}(x^2 - y^2 , 2xy)}
  %\\&\hskip0.5in
  +2  F_v(x^2 - y^2 , 2xy) \\&\hskip0.5in
 + \textcolor{red}{4x^2  F_{vu}(x^2 - y^2 , 2xy)  
   +4xy\,F_{vv}(x^2 - y^2 , 2xy)} \\
&= 2 \, F_v(x^2 - y^2 , 2xy) -4xy\, F_{uu}(x^2 - y^2 , 2xy) \\&\hskip0.5in
  +4(x^2-y^2)\, F_{uv}(x^2 - y^2 , 2xy) \\&\hskip0.5in
  +4xy\,  F_{vv}(x^2 - y^2 , 2xy)
\end{align*}
\end{solution}


%%%%%%%%%%%%%%%%%%%%%%%%%%%%%%%%
\begin{question}[M200 2008D] %3
$u(x,y)$ is defined as
\begin{equation*}
u(x,y) = e^y\, F\big(xe^{-y^2}\big)
\end{equation*}
for an arbitrary function $F(z)$.

\begin{enumerate}[(a)]
\item
If $F(z) = \ln(z)$, find $\pdiff{u}{x}$ and $\pdiff{u}{y}$.

\item
For an arbitrary $F(z)$ show that $u(x,y)$ satisfies
\begin{equation*}
2xy\pdiff{u}{x} + \pdiff{u}{y} = u
\end{equation*}
\end{enumerate}
\end{question}

\begin{hint}
(b)  Since $F$ is a function of only one variable, the chain rule for (say) $\frac{\partial}{\partial x} F\big(xe^{-y^2}\big)$  has only one term.
\end{hint}

\begin{answer}
(a) $\pdiff{u}{x}(x,y) = \frac{e^y}{x}$,
    $\pdiff{u}{y}(x,y) = e^y\,\ln (x) -y^2\,e^y -2y e^y$

(b) See the solution.
\end{answer}

\begin{solution}
For any (differentiable) function $F$, we have, by the chain and product rules,
\begin{align*}
\pdiff{u}{x}(x,y)&= \pdiff{}{x}\Big[e^y\,F\big(xe^{-y^2}\big)\Big]
                =e^y\,\pdiff{}{x}\Big[F\big(xe^{-y^2}\big)\Big] \\
               &=e^y\,F'\big(xe^{-y^2}\big)\pdiff{}{x}\Big(xe^{-y^2}\Big) \\ 
                  &= e^y\, F'\big(xe^{-y^2}\big)\ e^{-y^2} 
\displaybreak[0]\\ 
\pdiff{u}{y}(x,y)&= \pdiff{}{y}\Big[e^y\,F\big(xe^{-y^2}\big)\Big]\\
                &=e^y\,F\big(xe^{-y^2}\big)
                    +e^y\,\pdiff{}{y}\Big[F\big(xe^{-y^2}\big)\Big] \\
  &= e^y\, F\big(xe^{-y^2}\big)
         + e^y\,F'\big(xe^{-y^2}\big)\ \pdiff{}{y}\Big(xe^{-y^2}\Big) \\
  &= e^y\, F\big(xe^{-y^2}\big)
                  + e^y\, F'\big(xe^{-y^2}\big)\ (-2xy)e^{-y^2} 
\end{align*}

(a) In particular, when $F(z)=\ln(z)$, $F'(z)=\frac{1}{z}$ and
\begin{align*}
\pdiff{u}{x}(x,y)&= e^y\, \frac{1}{xe^{-y^2}}\ e^{-y^2} 
                  = \frac{e^y}{x}\\ 
\pdiff{u}{y}(x,y)&= e^y\, \ln\big(xe^{-y^2}\big)
                  + e^y\, \frac{1}{xe^{-y^2}}\ (-2xy)e^{-y^2}
                  = e^y\, \ln\big(xe^{-y^2}\big)
                  -2y e^y \\
                &= e^y\,\ln (x) -y^2\,e^y -2y e^y
\end{align*}

(b) In general
\begin{align*}
2xy\pdiff{u}{x} + \pdiff{u}{y}
&=2xy\ e^y\, F'\big(xe^{-y^2}\big)\ e^{-y^2}
           +e^y\, F\big(xe^{-y^2}\big)
                  + e^y\, F'\big(xe^{-y^2}\big)\ (-2xy)e^{-y^2} \\
&=e^y\, F\big(xe^{-y^2}\big) \\
&=u
\end{align*}
\end{solution}

%%%%%%%%%%%%%%%%%%%%%%%%%%%%%%%%
\begin{question}[M200 2009A] %2
Let $f(x)$ and $g(x)$ be two functions of $x$ satisfying $f''(7) = -2$ 
and $g''(-4) = -1$. If $z = h(s,t) = f(2s + 3t) + g(s - 6t)$ is a function 
of $s$ and $t$, find the value of $\frac{\partial^2 z}{\partial t^2}$ 
when $s = 2$ and $t = 1$.
\end{question}

\begin{hint}
At some point, you'll be using the chain rule you learned in first-semester calculus.
\end{hint}

\begin{answer}
$-54$
\end{answer}

\begin{solution}
By the chain rule,
\begin{align*}
\pdiff{h}{t}(s,t) 
&=\pdiff{}{t}\big[f(2s + 3t)\big] + \pdiff{}{t}\big[g(s - 6t)\big] \\
&=f'(2s + 3t)\pdiff{}{t}(2s+3t) + g(s - 6t)\pdiff{}{t}(s-6t) \\
&= 3f'(2s + 3t) - 6g'(s - 6t) \displaybreak[0]\\
\frac{\partial^2 h}{\partial t^2}(s,t)
&=\textcolor{blue}{3\pdiff{}{t}\big[f'(2s + 3t)\big]} 
  \textcolor{red}{- 6\pdiff{}{t}\big[ g'(s - 6t) \big]} \\
&=\textcolor{blue}{3\,f''(2s + 3t)\pdiff{}{t}(2s+3t)} 
  \textcolor{red}{- 6\,g''(s - 6t) \pdiff{}{t}(s-6t)} \\
&=\textcolor{blue}{9f''(2s+3t)} +\textcolor{red}{36g''(s-6t)}
\end{align*}
In particular
\begin{align*}
\frac{\partial^2 h}{\partial t^2}(2,1)
                 =9f''(7) +36g''(-4)
                 =9(-2) +36(-1)
                 =-54
\end{align*}
\end{solution}

%%%%%%%%%%%%%%%%%%%%%%%%%%%%%%%%
\begin{question}[M200 2011A] %3
Suppose that $w = f (xz, yz)$, where $f$ is a differentiable function. 
Show that
\begin{equation*}
x\pdiff{w}{x} + y\pdiff{w}{y} = z\pdiff{w}{z}
\end{equation*}
\end{question}

\begin{hint}
Just compute the first order partial derivatives of $w(x,y,z)$.
\end{hint}

\begin{answer}
See the solution.
\end{answer}

\begin{solution}
We'll first compute the first order partial derivatives of $w(x,y,z)$.
Write $u(x,y,z)=xz$ and $v(x,y,z)=yz$ so that 
$w(x,y,z) = f\big(u(x,y,z), v(x,y,z)\big)$.
By the chain rule,
\begin{align*}
\pdiff{w}{x}(x,y,z) &=  \pdiff{}{x}\big[f \big(u(x,y,z), v(x,y,z)\big)\big]\\
                   &=\pdiff{f}{u}\big(u(x,y,z), v(x,y,z)\big)\pdiff{u}{x}(x,y,z)
                +\pdiff{f}{v}\big(u(x,y,z), v(x,y,z)\big)\pdiff{v}{x}(x,y,z) \\
                   &=z\pdiff{f}{u}(xz, yz) \displaybreak[0]\\
\pdiff{w}{y}(x,y,z) &=  \pdiff{}{y}\big[f \big(u(x,y,z), v(x,y,z)\big)\big]\\
                   &=\pdiff{f}{u}\big(u(x,y,z), v(x,y,z)\big)\pdiff{u}{y}(x,y,z)
                +\pdiff{f}{v}\big(u(x,y,z), v(x,y,z)\big)\pdiff{v}{y}(x,y,z) \\
                   &=z\pdiff{f}{v}(xz, yz) \displaybreak[0]\\
\pdiff{w}{z}(x,y,z) &=  \pdiff{}{z}\big[f \big(u(x,y,z), v(x,y,z)\big)\big]\\
                   &=\pdiff{f}{u}\big(u(x,y,z), v(x,y,z)\big)\pdiff{u}{z}(x,y,z)
                +\pdiff{f}{v}\big(u(x,y,z), v(x,y,z)\big)\pdiff{v}{z}(x,y,z) \\
                   &=x\pdiff{f}{u}(xz, yz) + y\pdiff{f}{v}(xz, yz) 
\end{align*}
So
\begin{align*}
x\pdiff{w}{x} + y\pdiff{w}{y}
=xz\pdiff{f}{u}(xz, yz) + yz\pdiff{f}{v}(xz, yz)
=z\left[x\pdiff{f}{u}(xz, yz) + y\pdiff{f}{v}(xz, yz)\right]
=z\pdiff{w}{z}
\end{align*}
as desired.
\end{solution}

%%%%%%%%%%%%%%%%%%%%%%%%%%%%%%%%
\begin{question}[M200 2011D] %2
Suppose $z = f (x, y)$ has continuous second order partial derivatives, and $x = r \cos t$,
$y = r \sin t$. Express the following partial derivatives in terms $r$, $t$, and partial derivatives
of $f$.
\begin{enumerate}[(a)]
\item
$\pdiff{z}{t}$
\item
$\frac{\partial^2 z}{\partial t^2}$
\end{enumerate}
\end{question}

\begin{hint}
The function $\pdiff{f}{x}$ depends on both $x$ and $y$, so don't forget to account for both of these when you take its partial derivative.
\end{hint}

\begin{answer}
(a) \begin{align*}
\pdiff{z}{t}(r,t) &= -r\sin t\  \pdiff{f}{x}(r\cos t\,,\,r\sin t)
                     +r\cos t\  \pdiff{f}{y}(r\cos t\,,\,r\sin t)
\end{align*}

(b)
\begin{align*}
\frac{\partial^2 z}{\partial t^2}(r,t) 
&= -r\cos t \  \pdiff{f}{x}
   -r\sin t \  \pdiff{f}{y} \\
&\hskip0.5in
  +r^2\sin^2 t\ \frac{\partial^2 f}{\partial x^2}
  -2r^2\sin t\cos t\ 
       \frac{\partial^2\ f}{\partial x\partial y}
  +r^2\cos^2 t\ \frac{\partial^2 f}{\partial y^2}
\end{align*}
with all of the partial derivatives of $f$ evaluated at 
$(r\cos t\,,\,r\sin t)$.
\end{answer}

\begin{solution}
 By definition $z(r,t) = f(r\cos t\,,\,r\sin t)$.

(a) By the chain rule
\begin{align*}
\pdiff{z}{t}(r,t) 
   &= \pdiff{}{t}\Big[f(r\cos t\,,\,r\sin t)\Big] \\
   &= \pdiff{f}{x}(r\cos t\,,\,r\sin t)\pdiff{}{t}(r\cos t)
       +\pdiff{f}{y}(r\cos t\,,\,r\sin t)\pdiff{}{t}(r\sin t) \\
   &= -r\sin t\  \pdiff{f}{x}(r\cos t\,,\,r\sin t)
                     +r\cos t\  \pdiff{f}{y}(r\cos t\,,\,r\sin t)
\end{align*}

(b) By linearity, the product rule and the chain rule
\begin{align*}
\frac{\partial^2 z}{\partial t^2}(r,t) 
&= -\pdiff{}{t}\left[r\sin t\  \pdiff{f}{x}(r\cos t\,,\,r\sin t)\right]
   +\pdiff{}{t}\left[r\cos t\  \pdiff{f}{y}(r\cos t\,,\,r\sin t)\right] \\
&= -r\cos t\  \pdiff{f}{x}(r\cos t\,,\,r\sin t)
   \textcolor{blue}{
     -r\sin t\  \pdiff{}{t}\left[\pdiff{f}{x}(r\cos t\,,\,r\sin t)\right]} \\
    &\hskip0.5in
   -r\sin t\  \pdiff{f}{y}(r\cos t\,,\,r\sin t) 
   \textcolor{red}{
   +r\cos t\  \pdiff{}{t}\left[\pdiff{f}{y}(r\cos t\,,\,r\sin t)\right]} 
\displaybreak[0]\\
&= -r\cos t \  \pdiff{f}{x}(r\cos t\,,\,r\sin t) \\
   &\hskip0.5in 
  \textcolor{blue}{
    +r^2\sin^2 t\ \frac{\partial^2 f}{\partial x^2}(r\cos t\,,\,r\sin t)
   -r^2\sin t\cos t\ 
       \frac{\partial^2\ f}{\partial y\partial x}(r\cos t\,,\,r\sin t)}
\\
&\phantom{=}-r\sin t \  \pdiff{f}{y}(r\cos t\,,\,r\sin t) 
\\
&\hskip0.5in 
 \textcolor{red}{-r^2\sin t\cos t\ 
       \frac{\partial^2\ f}{\partial x\partial y}(r\cos t\,,\,r\sin t)
  +r^2\cos^2 t\ \frac{\partial^2 f}{\partial y^2}(r\cos t\,,\,r\sin t)}
\displaybreak[0]\\
&= -r\cos t \  \pdiff{f}{x}
   -r\sin t \  \pdiff{f}{y} \\
&\hskip0.5in
  +r^2\sin^2 t\ \frac{\partial^2 f}{\partial x^2}
  -2r^2\sin t\cos t\ 
       \frac{\partial^2\ f}{\partial x\partial y}
  +r^2\cos^2 t\ \frac{\partial^2 f}{\partial y^2}
\end{align*}
with all of the partial derivatives of $f$ evaluated at 
$(r\cos t\,,\,r\sin t)$.
\end{solution}

%%%%%%%%%%%%%%%%%%%%%%%%%%%%%%%%
\begin{question}[M200 2012a] %3
Let $z = f(x, y)$, where $f(x, y)$ has continuous second-order partial derivatives, 
and
\begin{equation*}
f_x (2, 1) = 5, \qquad
f_y(2, 1) =-2, \qquad
f_{xx}(2, 1) = 2,\qquad
f_{xy}(2, 1) = 1, \qquad
f_{yy}(2, 1) = -4
\end{equation*}
Find 
$
\difftwo{}{t} z\big(x(t),y(t)\big)
$
when $x(t)=2t^2$, $y(t)=t^3$ and $t=1$.
\end{question}

%\begin{hint}
%
%\end{hint}

\begin{answer}
$28$
\end{answer}

\begin{solution}
Write $w(t) = z\big(x(t),y(t)\big) = f\big(x(t),y(t)\big)$ with
$x(t)=2t^2$, $y(t)=t^3$.
We are to compute $\difftwo{w}{t}(1)$.
By the chain rule
\begin{align*}
\diff{w}{t}(t)&= \diff{}{t} f\big(x(t),y(t)\big) \\
  &= f_x(x(t)\,,\,y(t))\,\diff{x}{t}(t)
                     +f_y(x(t)\,,\,y(t))\,\diff{y}{t}(t) \\
  &= 4t\,f_x(x(t)\,,\,y(t))
                     +3t^2\,f_y(x(t)\,,\,y(t)) 
\end{align*}
By linearity, the product rule, and the chain rule,
\begin{align*}
\difftwo{}{t} f\big(x(t),y(t)\big) 
&= \diff{}{t}\left[4t\,f_x\big(x(t),y(t)\big) \right]
   +\diff{}{t}\left[3t^2\,f_y\big(x(t),y(t)\big) \right] \\
&= 4\,f_x\big(x(t),y(t)\big)  
   + \textcolor{blue}{4t \diff{}{t}\left[f_x\big(x(t),y(t)\big) \right] }
         \\&\hskip0.1in
   +6t\,f_y\big(x(t),y(t)\big) 
   +\textcolor{red}{3t^2\diff{}{t}\left[f_y\big(x(t),y(t)\big) \right]} \\
&= 4\,f_x(2t^2\,,\,t^3) 
          + \textcolor{blue}{4t\Big[f_{xx}\big(x(t),y(t)\big)\,\diff{x}{t}(t)
          + f_{xy}\big(x(t),y(t)\big)\,\diff{y}{t}(t)\Big]} \\
&\phantom{=} + 6t\,f_y(2t^2\,,\,t^3) 
          + \textcolor{red}{3t^2\Big[f_{yx}\big(x(t),y(t)\big)\,\diff{x}{t}(t) 
          + f_{yy}\big(x(t),y(t)\big)\,\diff{y}{t}(t)\Big]} \\
&= 4\,f_x(2t^2\,,\,t^3) 
          + \textcolor{blue}{16t^2\,f_{xx}(2t^2\,,\,t^3)
          + 12t^3\,f_{xy}(2t^2\,,\,t^3)} \\
&\phantom{=} + 6t\,f_y(2t^2\,,\,t^3) 
          + \textcolor{red}{12t^3\,f_{yx}(2t^2\,,\,t^3) 
          + 9t^4\,f_{yy}(2t^2\,,\,t^3)} 
\end{align*}
In particular, when $t=1$, and since $f_{xy}(2, 1)=f_{yx}(2, 1)$,
\begin{align*}
\left.\difftwo{}{t} f\big(x(t),y(t)\big) \right|_{t=1}
&= 4\,(5) 
          +\textcolor{blue}{ 16\,(2)
          + 12\,(1)} \\
&\phantom{=} + 6\,(-2) 
          + \textcolor{red}{12\,(1)
          + 9\,(-4)} \\
&=28 
\end{align*}
\end{solution}

%%%%%%%%%%%%%%%%%%%%%%%%%%%%%%%%
\begin{question}[M200 2012D] %2
Assume that the function $F(x,y,z)$ satisfies the equation
$\frac{\partial F}{\partial z} = \frac{\partial^2 F}{\partial x^2}
+ \frac{\partial^2 F}{\partial y^2}$ and the mixed partial derivatives
$\frac{\partial^2 F}{\partial x \partial y}$ and
$\frac{\partial^2 F}{\partial y \partial x}$ are equal. Let $A$ be some 
constant and let $G(\gamma, s, t) = F (\gamma + s, \gamma-s, At)$. 
Find the value of $A$ such that
$\frac{\partial G}{\partial t} = \frac{\partial^2 G}{\partial \gamma^2}
+ \frac{\partial^2 G}{\partial s^2}$.
\end{question}

\begin{hint}
Just compute $\frac{\partial G}{\partial t}$, 
$\frac{\partial^2 G}{\partial \gamma^2}$ and
$\frac{\partial^2 G}{\partial s^2}$.
\end{hint}

\begin{answer}
$A=2$.
\end{answer}

\begin{solution}
By the chain rule
\begin{align*}
\pdiff{G}{t}(\ga,s,t) 
  &= \pdiff{}{t} \Big[F (\gamma + s, \gamma-s, At)\Big] \\
  &=\pdiff{F}{x}(\gamma + s, \gamma-s, At)\, \pdiff{}{t}(\gamma + s)
   +\pdiff{F}{y}(\gamma + s, \gamma-s, At)\, \pdiff{}{t}(\gamma - s)
   \\&\hskip1in
   +\pdiff{F}{z}(\gamma + s, \gamma-s, At)\,\pdiff{}{t}(At) \\
  &= A\,\pdiff{F}{z}(\gamma + s, \gamma-s, At) 
\end{align*}
and
\begin{align*}
\pdiff{G}{\ga}(\ga,s,t) 
  &= \pdiff{}{\ga} \Big[F (\gamma + s, \gamma-s, At)\Big] \\
  &=\pdiff{F}{x}(\gamma + s, \gamma-s, At)\, \pdiff{}{\ga}(\gamma + s)
   +\pdiff{F}{y}(\gamma + s, \gamma-s, At)\, \pdiff{}{\ga}(\gamma - s)
   \\&\hskip1in
   +\pdiff{F}{z}(\gamma + s, \gamma-s, At)\,\pdiff{}{\ga}(At) \\
  &=  \pdiff{F}{x}(\gamma + s, \gamma-s, At)
                           +\pdiff{F}{y}(\gamma + s, \gamma-s, At)
\tag{E1} 
\end{align*}
and
\begin{align*}
\pdiff{G}{s}(\ga,s,t) 
  &= \pdiff{}{s} \Big[F (\gamma + s, \gamma-s, At)\Big] \\
  &=\pdiff{F}{x}(\gamma + s, \gamma-s, At)\, \pdiff{}{s}(\gamma + s)
   +\pdiff{F}{y}(\gamma + s, \gamma-s, At)\, \pdiff{}{s}(\gamma - s)
   \\&\hskip1in
   +\pdiff{F}{z}(\gamma + s, \gamma-s, At)\,\pdiff{}{s}(At) \\
&=  \pdiff{F}{x}(\gamma + s, \gamma-s, At)
                           -\pdiff{F}{y}(\gamma + s, \gamma-s, At)
\tag{E2} 
\end{align*}
We can evaluate the second derivatives by applying the chain rule
to the four terms on the right hand sides of
\begin{align*}
\frac{\partial^2G}{\partial\ga^2}(\ga,s,t)
&=\pdiff{}{\ga}\Big[\pdiff{G}{\ga}(\ga, s, t)\Big]
=\textcolor{red}{\pdiff{}{\ga}\Big[\pdiff{F}{x}(\gamma + s, \gamma-s, At)\Big]}
+\textcolor{orange}{\pdiff{}{\ga}\Big[\pdiff{F}{y}(\gamma + s, \gamma-s, At) 
                                                                   \Big]}
\\
\frac{\partial^2G}{\partial s^2}(\ga,s,t)
&=\pdiff{}{s}\Big[\pdiff{G}{s}(\ga, s, t)\Big]
=\textcolor{blue}{\pdiff{}{s}\Big[\pdiff{F}{x}(\gamma + s, \gamma-s, At)\Big]}
  -\textcolor{violet}{\pdiff{}{s}\Big[\pdiff{F}{y}(\gamma + s, \gamma-s, At)
                                                            \Big]}
\end{align*}
Alternatively, we can observe that replacing $F$ by $\pdiff{F}{x}$ in
(E1) and (E2) gives
\begin{align*}
\textcolor{red}{\pdiff{}{\ga}\Big[\pdiff{F}{x}(\gamma + s, \gamma-s, At)\Big]}
&=\textcolor{red}{\frac{\partial^2 F}{\partial x^2}(\gamma + s, \gamma-s, At)
  +\frac{\partial^2 F}{\partial y\partial x}(\gamma + s,  \gamma-s, At)}
\\
\textcolor{blue}{\pdiff{}{s}\Big[\pdiff{F}{x}(\gamma + s, \gamma-s, At) \Big]}
&=\textcolor{blue}{\frac{\partial^2 F}{\partial x^2}(\gamma + s, \gamma-s, At)
         -\frac{\partial^2 F}{\partial y\partial x}(\gamma + s, \gamma-s, At)}
\end{align*}
replacing $F$ by $\pdiff{F}{y}$ in
(E1) and (E2) gives
\begin{align*}
\textcolor{orange}{\pdiff{}{\ga}\Big[\pdiff{F}{y}(\gamma + s, \gamma-s,At)\Big]}
&=\textcolor{orange}{\frac{\partial^2 F}{\partial x\partial y}(\gamma + s, 
                                                                \gamma-s, At)
 +\frac{\partial^2 F}{\partial y^2}(\gamma + s, \gamma-s,At)}
\\
\textcolor{violet}{\pdiff{}{s}\Big[\pdiff{F}{y}(\gamma + s, \gamma-s, At) \Big]}
&=\textcolor{violet}{\frac{\partial^2 F}{\partial x\partial y}(\gamma + s, 
                                                                 \gamma-s, At)
         -\frac{\partial^2 F}{\partial y^2}(\gamma + s, \gamma-s, At)}
\end{align*}
Consequently
\begin{align*}
\frac{\partial^2G}{\partial\ga^2}(\ga,s,t) 
 &=  \textcolor{red}{\frac{\partial^2 F}{\partial x^2}(\gamma + s, \gamma-s, At)
   +\frac{\partial^2 F}{\partial y\partial x}(\gamma + s,  \gamma-s, At)} \\
                 &\phantom{=}+ 
 \textcolor{orange}{\frac{\partial^2 F}{\partial x\partial y}(\gamma + s, 
                                                               \gamma-s, At)
  +\frac{\partial^2 F}{\partial y^2}(\gamma + s, \gamma-s,  At)} \\
  &= \frac{\partial^2 F}{\partial x^2}(\gamma + s, \gamma-s, At)
    +2\frac{\partial^2 F}{\partial y\partial x}(\gamma + s, \gamma-s, At)
    +\frac{\partial^2 F}{\partial y^2}(\gamma + s, \gamma-s, At) 
\end{align*}
and
\begin{align*}
\frac{\partial^2G}{\partial s^2}(\ga,s,t) 
&= \textcolor{blue}{\frac{\partial^2 F}{\partial x^2}(\gamma + s, \gamma-s, At)
   -\frac{\partial^2 F}{\partial y\partial x}(\gamma + s, \gamma-s, At)} \\
                 &\phantom{=}-\Big[ 
  \textcolor{violet}{\frac{\partial^2 F}{\partial x\partial y}(\gamma + s, 
                                                              \gamma-s, At)
         -\frac{\partial^2 F}{\partial y^2}(\gamma + s, \gamma-s, At)}\Big] \\
  &= \frac{\partial^2 F}{\partial x^2}(\gamma + s, \gamma-s, At)
    -2\frac{\partial^2 F}{\partial y\partial x}(\gamma + s, \gamma-s, At)
    +\frac{\partial^2 F}{\partial y^2}(\gamma + s, \gamma-s, At) 
\end{align*}
So, suppressing the arguments,
\begin{align*}
\frac{\partial^2 G}{\partial \gamma^2}
+ \frac{\partial^2 G}{\partial s^2}
-\frac{\partial G}{\partial t}
&=2\frac{\partial^2 F}{\partial x^2} + 2\frac{\partial^2 F}{\partial y^2}
  -A \pdiff{F}{z}
=2\pdiff{F}{z}-A \pdiff{F}{z}
=0
\end{align*}
if $A=2$.
\end{solution}

\begin{question}[M200 2013D] %1d
Let $f(x)$ be a differentiable function, and suppose it is given that 
$f'(0) = 10$.
Let $g(s,t) = f (as - bt)$, where $a$ and $b$ are constants. Evaluate 
$\pdiff{g}{s}$ at the point $(s,t) = (b,a)$, that is, find 
$\pdiff{g}{s}\big|_{(b,a)}$.
\end{question}

%\begin{hint}
%
%\end{hint}

\begin{answer}
$10a$
\end{answer}

\begin{solution}
By the chain rule
\begin{align*}
\pdiff{g}{s}(s,t) = \pdiff{}{s}\big[f(as-bt)\big] 
                  =f'(as-bt)\pdiff{}{s}(as-bt)
                  = af'(as-bt)
\end{align*}
In particular
\begin{align*}
\pdiff{g}{s}(b,a) = af'(ab-ba) =af'(0) =10 a 
\end{align*}
\end{solution}

\begin{question}[M200 2014D] %2
Let $f(u,v)$ be a differentiable function of two variables, and let $z$ 
be a differentiable function of $x$ and $y$ defined implicitly by 
$f(xz,yz) = 0$. Show that
\begin{equation*}
x\pdiff{z}{x}+y\pdiff{z}{y} = -z
\end{equation*}
\end{question}

\begin{hint}
Use implicit differentiation to find $\pdiff{z}{x}(x,y)$ and 
$\pdiff{z}{y}(x,y)$.
\end{hint}

\begin{answer}
See the solution.
\end{answer}

\begin{solution}
We are told that the function $z(x,y)$ obeys
\begin{equation*}
f\big(x\,z(x,y)\,,\, y\,z(x,y)\big) =0 
\tag{$*$}
\end{equation*}
for all $x$ and $y$. By the chain rule,
\begin{align*}
&\pdiff{}{x}\big[f\big(x\,z(x,y)\,,\, y\,z(x,y)\big)\big] \\[-0.05in]
&\hskip0.3in=f_u\big(x\,z(x,y)\,,\, y\,z(x,y)\big)\pdiff{}{x}\big[x\,z(x,y)\big]
 + f_v\big(x\,z(x,y)\,,\, y\,z(x,y)\big)\pdiff{}{x}\big[y\,z(x,y)\big] \\
&\hskip0.3in=f_u\big(x\,z(x,y)\,,\, y\,z(x,y)\big)\,\big[z(x,y)+x\,z_x(x,y)\big]
            + f_v\big(x\,z(x,y)\,,\, y\,z(x,y)\big)\,y\,z_x(x,y)
\displaybreak[0]\\
&\pdiff{}{y}\big[f\big(x\,z(x,y)\,,\, y\,z(x,y)\big)\big] \\[-0.05in]
&\hskip0.3in=f_u\big(x\,z(x,y)\,,\, y\,z(x,y)\big)\pdiff{}{y}\big[x\,z(x,y)\big]
 + f_v\big(x\,z(x,y)\,,\, y\,z(x,y)\big)\pdiff{}{y}\big[y\,z(x,y)\big] \\
&\hskip0.3in=f_u\big(x\,z(x,y)\,,\, y\,z(x,y)\big)\, x\,z_y(x,y)
            + f_v\big(x\,z(x,y)\,,\, y\,z(x,y)\big)\,
                    \big[z(x,y)+y\,z_y(x,y)\big]
\end{align*}
so differentiating $(*)$ with respect to $x$
and with respect to $y$ gives
\begin{align*}
f_u\big(x\,z(x,y)\,,\, y\,z(x,y)\big)\,\big[z(x,y)+x\,z_x(x,y)\big]
            + f_v\big(x\,z(x,y)\,,\, y\,z(x,y)\big)\,y z_x(x,y) &=0  \\
f_u\big(x\,z(x,y)\,,\, y\,z(x,y)\big)\, x\,z_y(x,y)
            + f_v\big(x\,z(x,y)\,,\, y\,z(x,y)\big)\,
                    \big[z(x,y)+y\,z_y(x,y)\big]  &=0  
\end{align*}
or, leaving out the arguments,
\begin{align*}
f_u\,\big[z+x\,z_x\big] + f_v\,y z_x &=0  \\
f_u\, x\,z_y + f_v\,\big[z+y\,z_y\big]  &=0  
\end{align*}
Solving the first equation for $z_x$ and the second for $z_y$  gives
\begin{align*}
z_x & = -\frac{z\,f_u}{x\,f_u+y\,f_v} \\
z_y & = -\frac{z\,f_v}{x\,f_u+y\,f_v} 
\end{align*}
so that
\begin{align*}
x\pdiff{z}{x}+y\pdiff{z}{y}
= -\frac{xz\,f_u}{x\,f_u+y\,f_v} -\frac{yz\,f_v}{x\,f_u+y\,f_v}
=-\frac{z\,(x\,f_u+y\,f_v)}{x\,f_u+y\,f_v}  
 = -z
\end{align*}
as desired. 

\emph{Remark:}\ \ \ 
This is of course under the assumption that 
$x\,f_u+y\,f_v$ is nonzero. That is equivalent, by the chain rule,
to the assumption that $\pdiff{}{z}\big[f(xz,yz)\big]$ is non zero.
That, in turn, is almost, but not quite, equivalent to the statement
that $f(xz,yz)=0$ is can be solved for $z$ as a function of $x$ and $y$.
\end{solution}

%%%%%%%%%%%%%%%%%%%%%%%%%%%%%%%%
\begin{question}[M200 2015D] %3
Let $w(s,t) = u(2s + 3t, 3s - 2t)$ for some twice differentiable function 
$u = u(x, y)$.
\begin{enumerate}[(a)]
\item
Find $w_{ss}$ in terms of $u_{xx}$, $u_{xy}$ , and $u_{yy}$ 
(you can assume that $u_{xy} = u_{yx}$).
\item
Suppose $u_{xx} + u_{yy} = 0$. For what constant $A$ will 
$w_{ss} = Aw_{tt}$?
\end{enumerate}
\end{question}

%\begin{hint}
%
%\end{hint}

\begin{answer}
(a) $w_{ss}(s,t) =4\,u_{xx}(2s + 3t, 3s - 2t) 
                 +12\,u_{xy}(2s + 3t, 3s - 2t) 
                   +9\,u_{yy}(2s + 3t, 3s - 2t) $

(b) $A=-1$
\end{answer}

\begin{solution}
(a) 
By the chain rule
\begin{align*}
w_s(s,t) &=\pdiff{}{s}\big[u(2s + 3t, 3s - 2t)\big] \\
         &= u_x(2s + 3t, 3s - 2t)\pdiff{}{s}\big(2s + 3t\big)
           +u_y(2s + 3t, 3s - 2t)\pdiff{}{s}\big(3s - 2t\big) \\
         &= 2\,u_x(2s + 3t, 3s - 2t) +3\,u_y(2s + 3t, 3s - 2t)
\end{align*}
and
\begin{align*}
w_{ss}(s,t) &=\textcolor{blue}{2\pdiff{}{s}\big[u_x(2s + 3t, 3s - 2t)\big]}
            + \textcolor{red}{3\pdiff{}{s}\big[u_y(2s + 3t, 3s - 2t)\big]} \\
&=\textcolor{blue}{\big[4u_{xx}(2s + 3t, 3s - 2t) 
     +6u_{xy}(2s + 3t, 3s - 2t) \big]} \\&\hskip1.8in
  +\textcolor{red}{\big[6u_{yx}(2s + 3t, 3s - 2t) 
     +9u_{yy}(2s + 3t, 3s - 2t) \big]} \\
&=4\,u_{xx}(2s + 3t, 3s - 2t) 
     +12\,u_{xy}(2s + 3t, 3s - 2t) 
     +9\,u_{yy}(2s + 3t, 3s - 2t) 
\end{align*}


(b)
Again by the chain rule
\begin{align*}
w_t(s,t) &=\pdiff{}{t}\big[u(2s + 3t, 3s - 2t)\big] \\
         &= u_x(2s + 3t, 3s - 2t)\pdiff{}{t}\big(2s + 3t\big)
           +u_y(2s + 3t, 3s - 2t)\pdiff{}{t}\big(3s - 2t\big) \\
         &= 3\,u_x(2s + 3t, 3s - 2t) -2\, u_y(2s + 3t, 3s - 2t)
\end{align*}
and
\begin{align*}
w_{tt}(s,t) &=\textcolor{blue}{3\pdiff{}{t}\big[u_x(2s + 3t, 3s - 2t)\big]}
             -\textcolor{red}{2\pdiff{}{t}\big[u_y(2s + 3t, 3s - 2t)\big]} \\
&=\textcolor{blue}{\big[9u_{xx}(2s + 3t, 3s - 2t) 
     -6u_{xy}(2s + 3t, 3s - 2t) \big]} \\&\hskip1.8in
  -\textcolor{red}{\big[6u_{yx}(2s + 3t, 3s - 2t) 
     -4u_{yy}(2s + 3t, 3s - 2t) \big]} \\
&=9\,u_{xx}(2s + 3t, 3s - 2t) 
     -12\,u_{xy}(2s + 3t, 3s - 2t) 
     +4\,u_{yy}(2s + 3t, 3s - 2t) 
\end{align*}
Consquently, for any constant $A$,
\begin{align*}
w_{ss} - Aw_{tt}
&= (4-9A) u_{xx}   +(12+12A) u_{xy} + (9-4A) u_{yy} 
\end{align*}
Given that $u_{xx} + u_{yy}=0$, this will be zero, as desired,
if $A=-1$. (Then $(4-9A)=(9-4A)=13$.)
\end{solution}

%%%%%%%%%%%%%%%%%%%%%%%%%%%%%%%%
\begin{question}[M200 2016D] %4
Suppose that $f(x,y)$ is twice differentiable (with $f_{xy}=f_{yx}$),
and $x=r\cos\theta$ and $y=r\sin\theta$.
\begin{enumerate}[(a)]
\item
Evaluate $f_\theta$, $f_r$ and $f_{r\theta}$ in terms of $r$, $\theta$
and partial derivatives of $f$ with respect to $x$ and $y$.

\item
Let $g(x,y)$ be another function satisfying $g_x=f_y$ and $g_y=-f_x$.
Express $f_r$ and $f_\theta$ in terms of $r$, $\theta$ and $g_r$, $g_\theta$.
\end{enumerate}
\end{question}

\begin{hint}
This question uses bad (but standard) notation, in that the one symbol $f$
is used for two different functions, namely $f(x,y)$ and
$f(r,\theta)=f(x,y)\big|_{x=r\cos\theta,\,y=r\sin\theta}$.
Until you get used to it, undo this notation conflict by renaming the 
function of $r$ and $\theta$ to $F(r,\theta)$. That is,
$
F(r,\theta) = f\big(r\cos\theta\,,\,r\sin\theta\big)
$.
Similarly, rename $g$, viewed as a 
function of $r$ and $\theta$, to $G(r,\theta)$. That is,
$
G(r,\theta) = g\big(r\cos\theta\,,\,r\sin\theta\big)
$.

\end{hint}

\begin{answer}
(a)
\begin{align*}
\pdiff{}{\theta}\big[f\big(r\cos\theta\,,\,r\sin\theta\big)\big]
&=-r\sin\theta\, f_x
  +r\cos\theta\, f_y \\
\pdiff{}{r}\big[f\big(r\cos\theta\,,\,r\sin\theta\big)\big]
&= \cos\theta\, f_x
  +\sin\theta\, f_y \\
\frac{\partial^2}{\partial \theta\,\partial r}
         \big[f\big(r\cos\theta\,,\,r\sin\theta\big)\big] 
&=-\sin\theta\, f_x
  +\cos\theta\, f_y \\&\hskip0.5in
  -r\sin\theta\cos\theta\, f_{xx}
  +r[\cos^2\theta-\sin^2\theta]\, f_{xy}
  +r\sin\theta\cos\theta\, f_{yy}
\end{align*}
with the arguments of $f_x$, $f_y$, $f_{xx}$, $f_{xy}$ and $f_{yy}$
all being $\big(r\cos\theta\,,\,r\sin\theta\big)$.

(b)
\begin{align*}
\pdiff{}{r}\big[f\big(r\cos\theta\,,\,r\sin\theta\big)\big]
&=-\frac{1}{r} \pdiff{}{\theta}\big[g\big(r\cos\theta\,,\,r\sin\theta\big)\big]
\\
\pdiff{}{\theta}\big[f\big(r\cos\theta\,,\,r\sin\theta\big)\big]
&=r\pdiff{}{r}\big[g\big(r\cos\theta\,,\,r\sin\theta\big)\big]
\end{align*}
\end{answer}

\begin{solution}
This question uses bad (but standard) notation, in that the one symbol $f$
is used for two different functions, namely $f(x,y)$ and
$f(r,\theta)=f(x,y)\big|_{x=r\cos\theta,\,y=r\sin\theta}$.
Let us undo this notation conflict by renaming the 
function of $r$ and $\theta$ to $F(r,\theta)$. That is,
\begin{equation*}
F(r,\theta) = f\big(r\cos\theta\,,\,r\sin\theta\big)
\end{equation*}
Similarly, rename $g$, viewed as a 
function of $r$ and $\theta$, to $G(r,\theta)$. That is,
\begin{equation*}
G(r,\theta) = g\big(r\cos\theta\,,\,r\sin\theta\big)
\end{equation*}
In this new notation, we are being asked 
\begin{itemize}
\item 
in part (a) to find $F_\theta$, $F_r$ and $F_{r\theta}$ in terms of
$r$, $\theta$, $f_x$ and $f_y$, and 
\item 
in part (b) to express $F_r$ and $F_\theta$ in terms of $r$, $\theta$ and $G_r$, $G_\theta$.
\end{itemize}


(a) By the chain rule
\begin{align*}
F_\theta(r,\theta)
&=\pdiff{}{\theta}\big[f\big(r\cos\theta\,,\,r\sin\theta\big)\big] \\
&= f_x\big(r\cos\theta\,,\,r\sin\theta\big)
           \ \pdiff{}{\theta}\big(r\cos\theta\big)
  +f_y\big(r\cos\theta\,,\,r\sin\theta\big)
           \ \pdiff{}{\theta}\big(r\sin\theta\big)
   \\
&=-r\sin\theta\, f_x\big(r\cos\theta\,,\,r\sin\theta\big)
  +r\cos\theta\, f_y\big(r\cos\theta\,,\,r\sin\theta\big) 
\tag{E1}\\ %%%%%%%%%%%%
F_r(r,\theta)
&=\pdiff{}{r}\big[f\big(r\cos\theta\,,\,r\sin\theta\big)\big] \\
&= f_x\big(r\cos\theta\,,\,r\sin\theta\big)
           \ \pdiff{}{r}\big(r\cos\theta\big)
  +f_y\big(r\cos\theta\,,\,r\sin\theta\big)
           \ \pdiff{}{r}\big(r\sin\theta\big)
   \\
&= \cos\theta\, f_x\big(r\cos\theta\,,\,r\sin\theta\big)
  +\sin\theta\, f_y\big(r\cos\theta\,,\,r\sin\theta\big) 
\tag{E2}\\ %%%%%%%%%%%%%%%%%%%%%%%%%%%%%
F_{r\theta}(r,\theta) &= \pdiff{}{\theta}\big[F_r(r,\theta)\big] \\
%&= \frac{\partial^2}{\partial r\,\partial \theta}
%         \big[f\big(r\cos\theta\,,\,r\sin\theta\big)\big] 
&=\pdiff{}{\theta}\Big[\cos\theta\, f_x\big(r\cos\theta\,,\,r\sin\theta\big)
  +\sin\theta\, f_y\big(r\cos\theta\,,\,r\sin\theta\big)\Big] \\
&=-\sin\theta\  f_x\big(r\cos\theta\,,\,r\sin\theta\big)
  +\cos\theta\, \textcolor{blue}{
          \pdiff{}{\theta} \Big[f_x\big(r\cos\theta\,,\,r\sin\theta\big)\Big]}
   \\&\hskip0.2in
  +\cos\theta\  f_y\big(r\cos\theta\,,\,r\sin\theta\big)
  +\sin\theta\,  \textcolor{red}{ 
       \pdiff{}{\theta}\Big[f_y\big(r\cos\theta\,,\,r\sin\theta\big)\Big]} \\
&=-\sin\theta\  f_x\big(r\cos\theta\,,\,r\sin\theta\big) \\&\hskip1in
    +\cos\theta \textcolor{blue}{
          \Big[f_{xx}\big(r\cos\theta\,,\,r\sin\theta\big)\,(-r\sin\theta)
              +f_{xy}\big(r\cos\theta\,,\,r\sin\theta\big)\,(r\cos\theta)\Big]}
          \\&\hskip0.1in
  +\cos\theta\  f_y\big(r\cos\theta\,,\,r\sin\theta\big) \\&\hskip1in
    +\sin\theta \textcolor{red}{
          \Big[f_{yx}\big(r\cos\theta\,,\,r\sin\theta\big)\,(-r\sin\theta)
              +f_{yy}\big(r\cos\theta\,,\,r\sin\theta\big)\,(r\cos\theta)\Big]}
          \\
&=-\sin\theta\, f_x
  +\cos\theta\, f_y \\&\hskip1in
  -r\sin\theta\cos\theta\, f_{xx}
  +r[\cos^2\theta-\sin^2\theta]\, f_{xy}
  +r\sin\theta\cos\theta\, f_{yy}
\end{align*}
with the arguments of $f_x$, $f_y$, $f_{xx}$, $f_{xy}$ and $f_{yy}$
all being $\big(r\cos\theta\,,\,r\sin\theta\big)$.

(b) Replacing $f$ by $g$ in (E1) gives
\begin{align*}
G_\theta(r,\theta)
&=\pdiff{}{\theta}\big[g\big(r\cos\theta\,,\,r\sin\theta\big)\big] \\
&=-r\sin\theta\, g_x\big(r\cos\theta\,,\,r\sin\theta\big)
  +r\cos\theta\, g_y\big(r\cos\theta\,,\,r\sin\theta\big) \\
&=-r\sin\theta\, f_y\big(r\cos\theta\,,\,r\sin\theta\big)
  -r\cos\theta\, f_x\big(r\cos\theta\,,\,r\sin\theta\big) \\
&=-r \pdiff{}{r}\big[f\big(r\cos\theta\,,\,r\sin\theta\big)\big] 
\qquad\text{by (E2)}
\end{align*}
Replacing $f$ by $g$ in (E2) gives
\begin{align*}
G_r(r,\theta)
&=\pdiff{}{r}\big[g\big(r\cos\theta\,,\,r\sin\theta\big)\big] \\
&= \cos\theta\, g_x\big(r\cos\theta\,,\,r\sin\theta\big)
  +\sin\theta\, g_y\big(r\cos\theta\,,\,r\sin\theta\big) \\
&= \cos\theta\, f_y\big(r\cos\theta\,,\,r\sin\theta\big)
  -\sin\theta\, f_x\big(r\cos\theta\,,\,r\sin\theta\big) \\
&=\frac{1}{r} \pdiff{}{\theta}\big[f\big(r\cos\theta\,,\,r\sin\theta\big)\big]
\qquad\text{by (E1)}
\end{align*}
or
\begin{align*}
\pdiff{}{r}\big[f\big(r\cos\theta\,,\,r\sin\theta\big)\big]
&=-\frac{1}{r} \pdiff{}{\theta}\big[g\big(r\cos\theta\,,\,r\sin\theta\big)\big]
\\
\pdiff{}{\theta}\big[f\big(r\cos\theta\,,\,r\sin\theta\big)\big]
&=r\pdiff{}{r}\big[g\big(r\cos\theta\,,\,r\sin\theta\big)\big]
\end{align*}
\end{solution}

%%%%%%%%%%%%%%%%%%%%%%%%%%%%%%%%%%%%%
\begin{question}[M200 2003A] %2
By definition, the gradient of the differentiable function $f(x,y)$ at the point 
$\big( x_0\,,\,y_0\big)$ is
\begin{equation*}
\vnabla f(x_0,y_0)
=\llt \pdiff{f}{x}\big( x_0\,,\,y_0\big)\,,\,
     \pdiff{f}{y}\big( x_0\,,\,y_0\big)\rgt
\end{equation*}
Suppose that we know
\begin{equation*}
\vnabla f(3,6)=\llt 7,8\rgt
\end{equation*}
Suppose also that
\begin{equation*}
\vnabla g(1,2)=\llt -1,4\rgt,
\end{equation*}
and
\begin{equation*}
\vnabla h(1,2)=\llt -5,10\rgt.
\end{equation*}
Assuming $g(1,2)=3$, $h(1,2)=6$, and $z(s,t)=f\big(g(s,t),h(s,t)\big)$,
find
\begin{equation*}
\vnabla z(1,2)
\end{equation*}
\end{question}

%\begin{hint}
%
%\end{hint}

\begin{answer}
$\vnabla z(1,2)=\llt -47,108\rgt$
\end{answer}

\begin{solution}
By the chain rule
\begin{alignat*}{5}
\pdiff{z}{s}(s,t)
&=\pdiff{}{s}f\big(g(s,t),h(s,t)\big)
&&=\pdiff{f}{x}\big(g(s,t),h(s,t)\big)
            \pdiff{g}{s}(s,t)
+\pdiff{f}{y}\big(g(s,t),h(s,t)\big)
            \pdiff{h}{s}(s,t) \\
\pdiff{z}{t}(s,t)
&=\pdiff{}{t}f\big(g(s,t),h(s,t)\big)
&&=\pdiff{f}{x}\big(g(s,t),h(s,t)\big)
            \pdiff{g}{t}(s,t)
+\pdiff{f}{y}\big(g(s,t),h(s,t)\big)
            \pdiff{h}{t}(s,t)
\end{alignat*}
In particular
\begin{align*}
\pdiff{z}{s}(1,2)
&=\pdiff{f}{x}\big(g(1,2),h(1,2)\big)
            \pdiff{g}{s}(1,2)
+\pdiff{f}{y}\big(g(1,2),h(1,2)\big)
            \pdiff{h}{s}(1,2) \\
&=\pdiff{f}{x}(3,6)
            \pdiff{g}{s}(1,2)
+\pdiff{f}{y}(3,6))
            \pdiff{h}{s}(1,2) \\
&=7\times(-1)+8\times(-5)
=-47\\
\pdiff{z}{t}(1,2)
&=\pdiff{f}{x}\big(g(1,2),h(1,2)\big)
            \pdiff{g}{t}(1,2)
+\pdiff{f}{y}\big(g(1,2),h(1,2)\big)
            \pdiff{h}{t}(1,2) \\
&=7\times4+8\times 10
=108
\end{align*}
Hence $\vnabla z(1,2)=\llt -47,108\rgt$.
\end{solution}

%%%%%%%%%%%%%%%%%%%%%%%%%%%%%%%%%%%%%%%%%
\begin{question} [M200 2002D] %2
\begin{enumerate}[(a)]
\item
Let $f$ be an arbitrary differentiable function defined on the entire 
real line. Show that the function $w$ defined on the entire plane as
\begin{equation*}
w(x,y)=e^{-y}f(x-y)
\end{equation*}
satisfies the partial differential equation:
\begin{equation*}
w+\pdiff{w}{x}+\pdiff{w}{y}=0
\end{equation*}
\item
The equations $x=u^3-3uv^2$, $y=3u^2v-v^3$ and $z=u^2-v^2$
define $z$ as a function of $x$ and $y$. Determine 
$\pdiff{z}{x}$ at the point $(u,v)=(2,1)$ which 
corresponds to the point $(x,y)=(2,11)$.
\end{enumerate}
\end{question}

\begin{hint}
(b) Think of $x=u^3-3uv^2$, $y=3u^2v-v^3$ as two equations in the two 
    unknowns $u$, $v$ with $x$, $y$ just being given parameters. The
    question implicitly tells us that those two equations can be solved
    for $u$, $v$ in terms of $x$, $y$, at least near
    $(u,v)=(2,1)$, $(x,y)=(2,11)$. That is, the question implicitly
    tells us that the functions $u(x,y)$ and $v(x,y)$ are determined by
    $x=u(x,y)^3-3u(x,y)\,v(x,y)^2$, $y=3u(x,y)^2v(x,y)-v(x,y)^3$.
    Then $z(x,y)$ is determined by $z(x,y)=u(x,y)^2-v(x,y)^2$.
\end{hint}

\begin{answer}
(a) See the solution.\qquad
(b) $\frac{4}{15}$
\end{answer}

\begin{solution}
(a) By the product and chain rules
\begin{align*}
w_x(x,y) 
   &= \pdiff{}{x}\big[e^{-y}f(x-y)\big]
    = e^{-y} \pdiff{}{x}\big[f(x-y)\big]
    = e^{-y} f'(x-y)\pdiff{}{x}(x-y) \\
   &= e^{-y}f'(x-y) 
\\%%%%%%%%%%%%%%%%%
w_y(x,y) &= \pdiff{}{y}\big[e^{-y}f(x-y)\big]
    = -e^{-y}f(x-y) + e^{-y} \pdiff{}{y}\big[f(x-y)\big] \\
   &= -e^{-y}f(x-y) + e^{-y} f'(x-y)\pdiff{}{y}(x-y) \\
   &= -e^{-y}f(x-y)-e^{-y}f'(x-y)
\end{align*}
Hence
\begin{equation*}
w+\pdiff{w}{x}+\pdiff{w}{y}
=e^{-y}f(x-y)+e^{-y}f'(x-y)-e^{-y}f(x-y)-e^{-y}f'(x-y)
=0
\end{equation*}
as desired.

(b) 
Think of $x=u^3-3uv^2$, $y=3u^2v-v^3$ as two equations in the two 
    unknowns $u$, $v$ with $x$, $y$ just being given parameters. The
    question implicitly tells us that those two equations can be solved
    for $u$, $v$ in terms of $x$, $y$, at least near
    $(u,v)=(2,1)$, $(x,y)=(2,11)$. That is, the question implicitly
    tells us that the functions $u(x,y)$ and $v(x,y)$ are determined by
\begin{equation*}
x=u(x,y)^3-3u(x,y)\,v(x,y)^2\qquad
y=3u(x,y)^2v(x,y)-v(x,y)^3
\end{equation*}
Applying $\pdiff{}{x}$ to both sides of the equation
$x=u(x,y)^3-3u(x,y)v(x,y)^2$ gives
\begin{align*}
1&=3\,u(x,y)^2\,\pdiff{u}{x}(x,y)
       -3\pdiff{u}{x}(x,y)\,v(x,y)^2
       -6\,u(x,y)\,v(x,y)\,\pdiff{v}{x}(x,y) \\[0.05in]
\end{align*}
Then applying $\pdiff{}{x}$ to both sides of $y=3u(x,y)^2v(x,y)-v(x,y)^3$ gives
\begin{align*}
0&=6\,u(x,y)\,\pdiff{u}{x}(x,y)\,v(x,y)
   +3\,u(x,y)^2\,\pdiff{v}{x}(x,y)
   -3\,v(x,y)^2\,\pdiff{v}{x}(x,y)
\end{align*}
Substituting in $x=2$, $y=11$, $u=2$, $v=1$ gives
\begin{alignat*}{3}
1&=12\pdiff{u}{x}(2,11)
       -3\pdiff{u}{x}(2,11)
       -12\pdiff{v}{x}(2,11)
&&=9\pdiff{u}{x}(2,11)
       -12\pdiff{v}{x}(2,11)\\[0.05in]
0&=12\pdiff{u}{x}(2,11)
   +12\pdiff{v}{x}(2,11)
   -3\pdiff{v}{x}(2,11)
&&=12\pdiff{u}{x}(2,11)
   +9\pdiff{v}{x}(2,11)
\end{alignat*}
From the second equation $\pdiff{v}{x}(2,11)
=-\frac{4}{3}\pdiff{u}{x}(2,11)$. Substituting into the
first equation gives 
\begin{equation*}
1=9\pdiff{u}{x}(2,11)
       -12\left[-\frac{4}{3}\pdiff{u}{x}(2,11)\right]
= 25\pdiff{u}{x}(2,11)
\end{equation*}
so that $\pdiff{u}{x}(2,11)=\frac{1}{25}$
and $\pdiff{v}{x}(2,11)=-\frac{4}{75}$.
The question also tells us that $z(x,y)=u(x,y)^2-v(x,y)^2$.
Hence
\begin{align*}
&\pdiff{z}{x}(x,y)
=2u(x,y)\pdiff{u}{x}(x,y)
    -2v(x,y)\pdiff{v}{x}(x,y)\\
\implies &
\pdiff{z}{x}(2,11)
=4\pdiff{u}{x}(2,11)
    -2\pdiff{v}{x}(2,11)
=4\frac{1}{25}+2\frac{4}{75}=\frac{20}{75}
=\frac{4}{15}
\end{align*}
\end{solution}

%%%%%%%%%%%%%%%%%%%%%%%%%%%%%%%%
\begin{question}[M200 2001D] %2
The equations
\begin{align*}
x^2-y\cos(uv)&=v\\
x^2+y^2-\sin(uv)&=\frac{4}{\pi}u
\end{align*}
define $x$ and $y$ implicitly as functions of $u$ and $v$ (i.e. $x=x(u,v)$,
and $y=y(u,v)$) near the point $(x,y)=(1,1)$ at which 
$(u,v)=\big(\frac{\pi}{2},0\big)$.
\begin{enumerate}[(a)]
\item
Find
\begin{equation*}
\pdiff{x}{u}\text{ and }
\pdiff{y}{u}
\end{equation*}
at $(u,v)=\big(\frac{\pi}{2},0\big)$.

\item 
If $z=x^4+y^4$, determine $\pdiff{z}{u}$
at the point $(u,v)=\big(\frac{\pi}{2},0\big)$.
\end{enumerate}
\end{question}

\begin{hint}
The question tells us that $x(u,v)$ and $y=y(u,v)$ ar eimplicitly determined by
\begin{equation*}
x(u,v)^2-y(u,v)\cos(uv)=v\qquad
x(u,v)^2+y(u,v)^2-\sin(uv)=\frac{4}{\pi}u
\end{equation*}
at least near $(x,y)=(1,1)$, $(u,v)=\big(\frac{\pi}{2},0\big)$.
Then, in part (b), $z=x^4+y^4$ really means $z(u,v)=x(u,v)^4+y(u,v)^4$.
\end{hint}

\begin{answer}
(a) $\pdiff{x}{u}\big(\frac{\pi}{2},0\big)=\frac{2}{3\pi}$,
$\pdiff{y}{u}\big(\frac{\pi}{2},0\big)=\frac{4}{3\pi}$\qquad
(b) $\frac{8}{\pi}$
\end{answer}

\begin{solution}
(a) We are told that
\begin{equation*}
x(u,v)^2-y(u,v)\cos(uv)=v\qquad
x(u,v)^2+y(u,v)^2-\sin(uv)=\frac{4}{\pi}u
\end{equation*}
Applying $\pdiff{}{u}$ to both equations gives
\begin{align*}
2x(u,v)\pdiff{x}{u}(u,v)
-\pdiff{y}{u}(u,v)\cos(uv)+v\,y(u,v)\sin(uv)&=0 \\
2x(u,v)\pdiff{x}{u}(u,v)
   +2y(u,v)\pdiff{y}{u}(u,v)
   -v\cos(uv)&=\frac{4}{\pi}
\end{align*}
Setting $u=\frac{\pi}{2}$, $v=0$, $x\big(\frac{\pi}{2},0\big)=1$, 
$y\big(\frac{\pi}{2},0\big)=1$ gives
\begin{align*}
2\pdiff{x}{u}\left(\frac{\pi}{2},0\right)
-\pdiff{y}{u}\left(\frac{\pi}{2},0\right)&=0 \\[0.1in]
2\pdiff{x}{u}\left(\frac{\pi}{2},0\right)
   +2\pdiff{y}{u}\left(\frac{\pi}{2},0\right)
   &=\frac{4}{\pi}
\end{align*}
Substituting $\pdiff{y}{u}\big(\frac{\pi}{2},0\big)=2\pdiff{x}{u}\big(\frac{\pi}{2},0\big)$, 
from the first equation, into the second equation gives 
$6\pdiff{x}{u}\big(\frac{\pi}{2},0\big)=\frac{4}{\pi}$
so that
  $\pdiff{x}{u}\big(\frac{\pi}{2},0\big)=\frac{2}{3\pi}$ and
$\pdiff{y}{u}\big(\frac{\pi}{2},0\big)=\frac{4}{3\pi}$.

(b) 
We are told that $z(u,v)=x(u,v)^4+y(u,v)^4$. So
\begin{align*}
\pdiff{z}{u}(u,v)
&=4x(u,v)^3\ \pdiff{x}{u}(u,v)
+4y(u,v)^3\ \pdiff{y}{u}(u,v) 
\end{align*}
Substituting in $u=\frac{\pi}{2}$, $v=0$, $x\big(\frac{\pi}{2},0\big)=1$, 
$y\big(\frac{\pi}{2},0\big)=1$ and using the results of part (a),
\begin{align*}
\pdiff{z}{u}\left(\frac{\pi}{2},0\right)
&=4\,x\!\left(\frac{\pi}{2},0\right)^3\ 
       \pdiff{x}{u}\!\left(\frac{\pi}{2},0\right)
+4\,y\!\left(\frac{\pi}{2},0\right)^3\ 
       \pdiff{y}{u}\!\left(\frac{\pi}{2},0\right) \\
&=4\left(\frac{2}{3\pi}\right)
+4\left(\frac{4}{3\pi}\right) \\
&=\frac{8}{\pi}
\end{align*}
\end{solution}

%%%%%%%%%%%%%%%%%%%%%%%%%%%%%%%%%%%%%%%%%
\begin{question} [M200 2001A] % 5
Let $f(u,v)$ be a differentiable function, and let $u=x+y$
and $v=x-y$. Find a constant, $\al$, such that
\begin{align*}
(f_x)^2+(f_y)^2=\al\big((f_u)^2+(f_v)^2\big)
\end{align*}
\end{question}

\begin{hint}
This question uses bad (but standard) notation, in that the one symbol $f$
is used for two different functions, namely $f(u,v)$ and
$f(x,y)=f(u,v)\big|_{u=x+y,v=x-y}$. A better wording is
\begin{itemize}
\item[]
Let $f(u,v)$ and $F(x,y)$ be differentiable functions such that
$F(x,y)=f(x+y,x-y)$.  Find a constant, $\al$, such that
\begin{equation*}
F_x(x,y)^2+F_y(x,y)^2=\al\big\{f_u(x+y,x-y)^2+f_v(x+y,x-y)^2\big\}
\end{equation*}
\end{itemize}
\end{hint}

\begin{answer}
$\al=2$
\end{answer}

\begin{solution}
This question uses bad (but standard) notation, in that the one symbol $f$
is used for two different functions, namely $f(u,v)$ and
$f(x,y)=f(u,v)\big|_{u=x+y,v=x-y}$. A better wording is
\begin{itemize}
\item[]
Let $f(u,v)$ and $F(x,y)$ be differentiable functions such that
$F(x,y)=f(x+y,x-y)$.  Find a constant, $\al$, such that
\begin{equation*}
F_x(x,y)^2+F_y(x,y)^2=\al\big\{f_u(x+y,x-y)^2+f_v(x+y,x-y)^2\big\}
\end{equation*}
\end{itemize}
By the chain rule
\begin{align*}
\pdiff{F}{x}(x,y)
&=f_u(x+y,x-y)\pdiff{}{x}(x+y)
+f_v(x+y,x-y)\pdiff{}{x}(x-y) \\
&=f_u(x+y,x-y)+f_v(x+y,x-y)\\
\pdiff{F}{y}(x,y)
&=f_u(x+y,x-y)\pdiff{}{y}(x+y)
+f_v(x+y,x-y)\pdiff{}{y}(x-y) \\
&=f_u(x+y,x-y)-f_v(x+y,x-y)
\end{align*}
Hence
\begin{align*}
F_x(x,y)^2+F_y(x,y)^2
&=\Big[f_u(x+y,x-y)+f_v(x+y,x-y)\Big]^2 \\
&\hskip1in +\Big[f_u(x+y,x-y)-f_v(x+y,x-y)\Big]^2\\
&=2f_u(x+y,x-y)^2+2f_v(x+y,x-y)^2
\end{align*}
So $\al=2$ does the job.
\end{solution}

%%%%%%%%%%%%%%%%%%
\subsection*{\Application}
%%%%%%%%%%%%%%%%%%

%%%%%%%%%%%%%%%%%%%%%%%%%%%%%%%%
\begin{question}
 The wave equation
\begin{equation*}
\frac{\partial^2u}{\partial x^2}
   -\frac{1}{c^2}\,\frac{\partial^2u}{\partial t^2}
=0
\end{equation*}
arises in many models involving wave-like phenomena. Let $u(x,t)$ and 
$v(\xi,\eta)$ be related by the change of variables
\begin{align*}
u(x,t)&=v\big(\xi(x,t),\eta(x,t)\big)\cr
\xi(x,t)&=x-ct\cr
\eta(x,t)&=x+ct
\end{align*}


\begin{enumerate}[(a)]
\item
Show that 
$\ \frac{\partial^2u}{\partial x^2}
       -\frac{1}{c^2}\frac{\partial^2u}{\partial t^2}
=0\ $ 
if and only if
$\ \frac{\partial^2\hfil v\hfil}{\partial\xi\partial\eta}=0$.
\item
Show that 
$\ \frac{\partial^2u}{\partial x^2}
       -\frac{1}{c^2}\frac{\partial^2u}{\partial t^2}
=0\ $ 
if and only if 
$\ u(x,t)=F(x-ct)+G(x+ct)\ $ for some functions $F$ and $G$.
\item
Interpret $\ F(x-ct)+G(x+ct)\ $ in terms of travelling waves. 
Think of $u(x,t)$ as the height, at position $x$ and time $t$, of a wave 
    that is travelling along the $x$-axis. 
\end{enumerate}

\textit{Remark:} Don't be thrown by the strange symbols $\xi$ and $\eta$.
          They are just two harmless letters from the Greek alphabet, called
          ``xi'' and ``eta'' respectively.
\end{question}

\begin{hint}
Use the chain rule to show that
$\frac{\partial^2 u}{\partial x^2}(x,t)
-\frac{1}{c^2}\frac{\partial^2 u}{\partial t^2}(x,t)
=4\frac{\partial^2\hfil v\hfil\,}{\partial\xi\partial\eta}
      \big(\xi(x,t),\eta(x,t)\big)$.
\end{hint}

\begin{answer}
See the solutions.
\end{answer}

\begin{solution}
Recall that $u(x,t)=v\big(\xi(x,t),\eta(x,t)\big)$.
By the chain rule
\begin{align*}
\pdiff{u}{x}(x,t)
&= \pdiff{v}{\xi}\big(\xi(x,t),\eta(x,t)\big)
   \pdiff{\xi}{x}
+\pdiff{v}{\eta}\big(\xi(x,t),\eta(x,t)\big)
   \pdiff{\eta}{x} \\
&= \pdiff{v}{\xi}\big(\xi(x,t),\eta(x,t)\big)
+\pdiff{v}{\eta}\big(\xi(x,t),\eta(x,t)\big)
\\
%%
\pdiff{u}{t}(x,t)
&= \pdiff{v}{\xi}\big(\xi(x,t),\eta(x,t)\big)
   \pdiff{\xi}{t}
+\pdiff{v}{\eta}\big(\xi(x,t),\eta(x,t)\big)
   \pdiff{\eta}{t} \\
&= -c\pdiff{v}{\xi}\big(\xi(x,t),\eta(x,t)\big)
+c\pdiff{v}{\eta}\big(\xi(x,t),\eta(x,t)\big)
\end{align*}
Again by the chain rule
\begin{align*}
\frac{\partial^2 u}{\partial x^2}(x,t)
&=\textcolor{blue}{
     \pdiff{}{x} \Big[\pdiff{v}{\xi}\big(\xi(x,t),\eta(x,t)\big)\Big]}
+\textcolor{red}{\pdiff{}{x}
       \Big[\pdiff{v}{\eta}\big(\xi(x,t),\eta(x,t)\big)\Big]}\cr
&=\textcolor{blue}{
     \frac{\partial^2 v}{\partial \xi^2}\big(\xi(x,t),\eta(x,t)\big)
                                                         \pdiff{\xi}{x}
  +\frac{\partial^2\hfil v\hfil\,}{\partial\eta\partial \xi}
                                        \big(\xi(x,t),\eta(x,t)\big)
                                        \pdiff{\eta}{x}}\\&\hskip0.2in
+\textcolor{red}{\frac{\partial^2\hfil v\hfil\,}{\partial\xi\partial \eta}
                                      \big(\xi(x,t),\eta(x,t)\big)
                                                     \pdiff{\xi}{x}
+\frac{\partial^2 v}{\partial \eta^2}\big(\xi(x,t),\eta(x,t)\big)
                                                    \pdiff{\eta}{x}}\\
&=\frac{\partial^2 v}{\partial \xi^2}\big(\xi(x,t),\eta(x,t)\big)
+2\frac{\partial^2\hfil v\hfil\,}{\partial\xi\partial \eta}
                                       \big(\xi(x,t),\eta(x,t)\big)
+\frac{\partial^2 v}{\partial \eta^2}\big(\xi(x,t),\eta(x,t)\big)
\end{align*}
and
\begin{align*}
\frac{\partial^2 u}{\partial t^2}(x,t)
&=-c \textcolor{blue}{\pdiff{}{t}
              \Big[\pdiff{v}{\xi}\big(\xi(x,t),\eta(x,t)\big)\Big]}
+c\textcolor{red}{\pdiff{}{t}
              \Big[ \frac{\partial  v}{\partial\eta}
                          \big(\xi(x,t),\eta(x,t)\big)\Big]}
\\
&=-c\textcolor{blue}{\Big[\frac{\partial^2 v}{\partial \xi^2}
                                     \big(\xi(x,t),\eta(x,t)\big)
                                                      \pdiff{\xi}{t}
+\frac{\partial^2\hfil v\hfil\,}{\partial\eta\partial \xi}
                                      \big(\xi(x,t),\eta(x,t)\big)
                               \pdiff{\eta}{t}\Big]} \\&\hskip0.2in
+c\textcolor{red}{\Big[\frac{\partial^2\hfil v\hfil\,}{\partial\xi\partial\eta}
                    \big(\xi(x,t),\eta(x,t)\big)
                   \pdiff{\xi}{t}
+\frac{\partial^2 v}{\partial\eta^2}\big(\xi(x,t),\eta(x,t)\big)
   \pdiff{\eta}{t}\Big]}\\
&=c^2\frac{\partial^2 v}{\partial \xi^2}\big(\xi(x,t),\eta(x,t)\big)
-2c^2\frac{\partial^2\hfil v\hfil\,}{\partial\xi\partial\eta}
                                        \big(\xi(x,t),\eta(x,t)\big)
+c^2\frac{\partial^2 v}{\partial\eta^2}\big(\xi(x,t),\eta(x,t)\big)
\end{align*}
so that 
\begin{equation*}
\frac{\partial^2 u}{\partial x^2}(x,t)
-\frac{1}{c^2}\frac{\partial^2 u}{\partial t^2}(x,t)
=4\frac{\partial^2\hfil v\hfil\,}{\partial\xi\partial\eta}
      \big(\xi(x,t),\eta(x,t)\big)
\end{equation*}
Hence
\begin{align*}
\frac{\partial^2 u}{\partial x^2}(x,t)
-\frac{1}{c^2}\frac{\partial^2 u}{\partial t^2}(x,t)=0
\text{ for all }(x,t)
&\iff
4\frac{\partial^2\hfil v\hfil\,}{\partial\xi\partial\eta}
      \big(\xi(x,t),\eta(x,t)\big)=0\hbox{ for all }(x,t)\\
&\iff
\frac{\partial^2\hfil v\hfil\,}{\partial\xi\partial\eta}
      \big(\xi,\eta\big)=0\text{ for all }(\xi,\eta)
\end{align*}

(b)
Now $\frac{\partial^2\hfil v\hfil\,}{\partial\xi\partial\eta}\big(\xi,\eta\big)
=\frac{\partial\hfill}{\partial\xi}
\Big[\frac{\partial v}{\partial\eta}\Big]=0$.
Temporarily rename $\pdiff{v}{\eta}=w$. The equation
$\pdiff{w}{\xi}(\xi,\eta)=0$ says that, for each fixed
$\eta$, $w(\xi,\eta)$ is a constant. The value of the constant may depend
on $\eta$. That is, $\pdiff{v}{\eta}(\xi,\eta)=w(\xi,\eta)
=H(\eta)$, for some function $H$. (As a check, observe that
$\pdiff{}{\xi}H(\eta)=0$.) 
So the derivative of $v$ with respect to $\eta$,
(viewing $\xi$ as a constant) is $H(\eta)$. 

Let $G(\eta)$ be any function whose derivative is $H(\eta)$
(i.e. an indefinite integral of $H(\eta)$). Then 
$\pdiff{}{\eta}\big[v(\xi,\eta)-G(\eta)]=H(\eta)-H(\eta)=0$. 
This is the case if and only if, for each fixed $\xi$, 
$v(\xi,\eta)-G(\xi,\eta)$  is a constant, independent of $\eta$. That is, 
if and  only if 
\begin{equation*}
v(\xi,\eta)-G(\eta)=F(\xi)
\end{equation*}  
for some function $F$. Hence
\begin{align*}
\frac{\partial^2 u}{\partial x^2}(x,t)
-\frac{1}{c^2}\frac{\partial^2 u}{\partial t^2}(x,t)=0
&\iff
\frac{\partial^2\hfil v\hfil\,}{\partial\xi\partial\eta}
      \big(\xi,\eta\big)=0\text{ for all }(\xi,\eta)\\
&\iff v(\xi,\eta)=F(\xi)+G(\eta)\text{ for some functions $F$ and $G$}\\
&\iff u(x,t)=v\big(\xi(x,t),\eta(x,t)\big)
=F\big(\xi(x,t)\big)+G\big(\eta(x,t)\big)\\
&\phantom{\iff u(x,t)\,}=F(x-ct)+G(x+ct)
\end{align*}

(c)
We'll give the interpretation of $F(x-ct)$. The case $G(x+ct)$ is similar.
Suppose that $u(x,t)=F(x-ct)$. Think of $u(x,t)$ as the height of water
at position $x$ and time $t$. Pick any number $z$. All points $(x,t)$
in space time for which $x-ct=z$ have the same value of $u$, namely $F(z)$.
So if you move so that your position is $x=z+ct$ (i.e. you move the right 
with speed $c$) you always see the same wave height. Thus $F(x-ct)$
represents a wave moving to the right with speed $c$. 
\begin{center}
   \includegraphics{packet.pdf}
\end{center}
Similarly, $G(x+ct)$ represents a wave moving to the left with speed $c$.

\end{solution}

%%%%%%%%%%%%%%%%%%%%%%%%%%%%%%%%
\begin{question}
 Evaluate 
\begin{enumerate}[(a)]
\item
$\pdiff{y}{z}$\ if\ $e^{yz}-x^2 z \ln y = \pi$
\item
$\diff{y}{x}$\ if\ $F(x,y,x^2-y^2)=0$
\item
$\left(\pdiff{y}{x}\right)_u$\ if\
$xyuv=1$ and $x+y+u+v=0$ 
\end{enumerate}
\end{question}

\begin{hint}
For each part, first determine which variables $y$ is a function
of.
\end{hint}

\begin{answer}
(a) $\displaystyle\pdiff{y}{z}(x,z)
           =\frac{x^2 \ln y(x,z)-y(x,z)e^{y(x,z)\,z}}
                     {ze^{y(x,z)\,z}-\frac{x^2 z}{y(x,z)}}$

(b) $\displaystyle\diff{y}{x}(x)
=-\frac{F_1\big(x,y(x),x^2-y(x)^2\big)+2x\,F_3\big(x,y(x),x^2-y(x)^2\big)}
{F_2\big(x,y(x),x^2-y(x)^2\big)-2y(x)\,F_3\big(x,y(x),x^2-y(x)^2\big)}$

(c) $\displaystyle\left(\pdiff{y}{x}\right)_u\!\!(x,u)
=\frac{y(x,u)\,v(x,u)-x\,y(x,u)}{x\,y(x,u)-x\,v(x,u)}$

\end{answer}

\begin{solution}
(a)
We are told to evaluate $\pdiff{y}{z}$. So $y$ has to be a function of $z$
and possibly some other variables.
We are also told that $x$, $y$, and $z$ are related by the single equation
$e^{yz}-x^2 z \ln y = \pi$.
So we are to think of $x$ and $z$ as being independent variables and think of 
$y(x,z)$ as being determined by solving $e^{yz}-x^2 z \ln y = \pi$ for $y$ as a function of $x$ and $z$. That is, the function $y(x,z)$ obeys
\begin{equation*}
e^{y(x,z)\,z}-x^2 z \ln y(x,z) = \pi
\end{equation*}
for all $x$ and $z$.
Applying 
$\pdiff{}{z}$ to both sides of this equation gives
\begin{align*}
&\left[y(x,z)+z\pdiff{y}{z}(x,z)\right]e^{y(x,z)\,z}-x^2 \ln y(x,z)
-x^2 z\frac{1}{y(x,z)}\pdiff{y}{z}(x,z) = 0\\
\implies & \pdiff{y}{z}(x,z)
=\frac{x^2 \ln y(x,z)-y(x,z)e^{y(x,z)\,z}}{ze^{y(x,z)\,z}-\frac{x^2 z}{y(x,z)}}
\end{align*}

(b) 
We are told to evaluate $\diff{y}{x}$. So $y$ has to be a function of the single variable $x$. We are also told that $x$ and $y$ are related by $F(x,y,x^2-y^2)=0$. So the function $y(x)$ has to obey 
\begin{equation*}
F\big(x,y(x),x^2-y(x)^2\big)=0
\end{equation*}
for all $x$. Applying 
$\diff{}{x}$ to both sides of that equation and using the chain rule gives
\begin{align*}
&F_1\big(x,y(x),x^2-y(x)^2\big)\,\ \diff{x}{x} 
+F_2\big(x,y(x),x^2-y(x)^2\big)\ \diff{y}{x}(x)
\\&\hskip2in
+F_3\big(x,y(x),x^2-y(x)^2\big)\,\diff{}{x}\left[x^2-y(x)^2\right]= 0
\\
\implies &F_1\big(x,y(x),x^2-y(x)^2\big) +F_2\big(x,y(x),x^2-y(x)^2\big)\ \diff{y}{x}(x)
\\&\hskip2in
+F_3\big(x,y(x),x^2-y(x)^2\big)\,\left[2x-2y(x)\diff{y}{x}(x)\right]= 0
\\
\implies & \diff{y}{x}(x)
=-\frac{F_1\big(x,y(x),x^2-y(x)^2\big)+2x\,F_3\big(x,y(x),x^2-y(x)^2\big)}
{F_2\big(x,y(x),x^2-y(x)^2\big)-2y(x)\,F_3\big(x,y(x),x^2-y(x)^2\big)}
\end{align*}

(c) 
The hard part of this question is figuring out what it is that we are to compute. We 
are asked to find some derivative of some function y. But what function? Four variables
appear in this question. Namely  $x$, $y$, $u$ and $v$. But we are not free to assign 
arbitrary values to all four of them. They have to be related by the two equations 
$xyuv=1$ and $x+y+u+v=0$. If we assign values to any two of  $x$, $y$, $u$ and $v$,
the values of the other two are to be determined by solving $xyuv=1$, $x+y+u+v=0$.
That is, we may choose any two of  $x$, $y$, $u$ and $v$ to be independent variables (i.e. 
variables that may be assigned any value). Then the other two variables are functions of those 
independent variables that are determined by solving the given equations.   

We are told to evaluate $\left(\pdiff{y}{x}\right)_u$. According to Notation \eref{CLP3}{notn partial}
in the CLP-3 text, $\left(\pdiff{y}{x}\right)_u$ is the partial derivative of $y$ with respect to $x$ 
with $u$ being held fixed. So $x$ and $u$ have to be independent variables and $y$ has to be a function 
of $x$ and $u$. The fourth variable $v$ also has to be a function of $x$ and $u$. The functions 
$y(x,u)$ and $v(x,u)$ must obey 
\begin{align*}
x\,y(x,u)\,u\,v(x,u)&=1 \\
x+y(x,u)+u+v(x,u)&=0
\end{align*}
for all $x$ and $u$.
Applying 
$\pdiff{}{x}$ to both sides of both of these equations
gives
\begin{align*}
y\,u\,v\ +\ x\,\pdiff{y}{x}\,u\,v\ +\ x\,y\,u\,\pdiff{v}{x}&=0
\\
1+\pdiff{y}{x}+0+\pdiff{v}{x}&=0
\end{align*}
Substituting, $\pdiff{v}{x}=-1-\pdiff{y}{x}$,
from the second equation, into the first equation gives
\begin{align*}
y\,u\,v\ +\ x\,\pdiff{y}{x}\,u\,v-x\,y\,u
           \left(1+\pdiff{y}{x}\right)=0
\end{align*}
Now $u$ cannot be $0$ because $x\,y(x,u)\,u\,v(x,u)=1$. So
\begin{align*}
y\,v\ +\ x\,\pdiff{y}{x}\,v-x\,y\left(1+\pdiff{y}{x}\right)=0
\implies & \left(\pdiff{y}{x}\right)_u\!\!(x,u)
=\frac{y(x,u)\,v(x,u)-x\,y(x,u)}{x\,y(x,u)-x\,v(x,u)}
\end{align*}

\end{solution}
