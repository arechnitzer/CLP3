%\documentclass[12pt]{article}

\questionheader{ex:s2.6}


%%%%%%%%%%%%%%%%%%
\subsection*{\Conceptual}
%%%%%%%%%%%%%%%%%%

\begin{question}
Let $x_0$ and $y_0$ be constants and let $m$ and $n$ be integers.
If $m<0$ assume that $x_0\ne 0$, and if $n<0$ assume that $y_0\ne 0$.
Define $P(x,y) = x^m y^n$.
\begin{enumerate}[(a)]
\item
Find the linear approximation to $P(x_0+\De x,y_0+\De y)$.
\item
Denote by
\begin{equation*}
P_\% = 100\left|\frac{P(x_0+\De x,y_0+\De y)-P(x_0,y_0)}{P(x_0,y_0)}\right|
\qquad
x_\% = 100\left|\frac{\De x}{x_0}\right|
\qquad
y_\% = 100\left|\frac{\De y}{y_0}\right|
\end{equation*}
the percentage errors in $P$, $x$ and $y$ respectively. Use the linear approximation to find an (approximate) upper bound on $P_\%$ in terms of
$m$, $n$, $x_\%$ and $y_\%$.
\end{enumerate}

\end{question}

\begin{hint}
Review Example \eref{CLP200}{eg errors in measurement} in the CLP-3 text.
Be careful when taking absolute values.
\end{hint}

\begin{answer}
(a) $P(x_0+\De x,y_0+\De y) 
       \approx P(x_0,y_0) + mx_0^{m-1}y_0^n\,\De x  + nx_0^m y_0^{n-1}\,\De y$

(b) $P_\% \le |m|\,x_\% + |n|\,y_\%$
\end{answer}

\begin{solution}
(a) The first order partial derivatives of $P(x,y)$ at $x=x_0$ and $y=y_0$ are
\begin{equation*}
P_x(x_0,y_0) = m x_0^{m-1} y_0^n\qquad
P_y(x_0,y_0) = n x_0^m y_0^{n-1}
\end{equation*}
So, by (\eref{CLP200}{eqn lin approx 2d}) in the CLP-3 text, the 
linear approximation is
\begin{align*}
P(x_0+\De x,y_0+\De y) 
       &\approx P(x_0,y_0) + P_x(x_0,y_0)\,\De x  + P_y(x_0,y_0)\,\De y\\
       &\approx P(x_0,y_0) + mx_0^{m-1}y_0^n\,\De x  + nx_0^m y_0^{n-1}\,\De y
\end{align*}

(b) By part (a)
\begin{align*}
\frac{P(x_0+\De x,y_0+\De y)-P(x_0,y_0)}{P(x_0,y_0)}
&\approx \frac{mx_0^{m-1}y_0^n\,\De x  + nx_0^m y_0^{n-1}\,\De y}{x_0^my_0^n}
 =m\frac{\De x}{x_0} + n\frac{\De y}{y_0}
\end{align*}
Hence
\begin{align*}
P_\% &\approx 100\left|m\frac{\De x}{x_0} + n\frac{\De y}{y_0}\right| \\
     &\le |m| 100\left|\frac{\De x}{x_0}\right| 
         + |n| 100\left|\frac{\De y}{y_0}\right| \\
     &\le |m| x_\% + |n| y_\%
\end{align*}

\emph{Warning.} The answer $m\,x_\% +n\,y_\%$, without absolute values on $m$
and $n$, can be seriously wrong. As an example, suppose that $m=1$, $n=-1$,
$x_0=y_0=1$, $\De x = 0.05$ and $\De y=-0.05$. Then
\begin{align*}
P_\%  &\approx 100\left|m\frac{\De x}{x_0} + n\frac{\De y}{y_0}\right| \\
      &= 100\left|(1)\frac{0.05}{1} + (-1)\frac{-0.05}{1}\right| \\
      & = 10\%
\end{align*}
while
\begin{align*}
m\,x_\% +n\,y_\% &= m\, 100\left|\frac{\De x}{x_0}\right| 
         + n\, 100\left|\frac{\De y}{y_0}\right| \\
               &=  (1) 100\left|\frac{0.05}{1}\right| 
         + (-1) 100\left|\frac{-0.05}{1}\right| \\
              &= 0
\end{align*}
The point is that $m$ and $n$ being of opposite sign does not guarantee that
there is a cancelation between the two terms of 
$m\frac{\De x}{x_0} + n\frac{\De y}{y_0}$, because $\frac{\De x}{x_0}$ and 
$\frac{\De y}{y_0}$ can also be of opposite sign.
\end{solution}


\begin{question}
Consider the following work.

\textcolor{blue}{
We compute, approximately, the $y$-coordinate of the point whose polar coordinates are $r=0.9$ and $\theta=2^\circ$. In general, the
$y$-coordinate of the point whose polar coordinates are $r$ and $\theta$
is $Y(r,\theta) = r\sin\theta$. The partial derivatives
\begin{equation*}
Y_r(r,\theta)=\sin\theta\qquad Y_\theta(r,\theta) = r\cos\theta
\end{equation*}  
So the linear approximation to $Y(r_0+\De r,\theta_0+\De\theta)$ with $r_0=1$
and $\theta_0=0$ is
\begin{align*}
Y(1+\De r,0+\De\theta)
&\approx Y(1,0) + Y_r(1,0)\,\De r + Y_\theta(1,0)\,\De\theta  \\
&= 0\ +\ (0)\,\De r\ +\ (1)\De\theta 
\end{align*}
Applying this with $\De r=-0.1$ and $\De\theta=2$ gives the (approximate)
$y$-coordinate
\begin{align*}
Y(0.9,2) = Y(1-0.1\,,\, 0+2)\approx 0\ +\ (0)\,(-0.1)\ +\ (1)(2)
         =2
\end{align*} 
}
This conclusion is ridiculous. We're saying that the $y$-coordinate is 
more than twice the distance from the point to the origin. What was the 
mistake? 

\end{question}

\begin{hint}
Units!
\end{hint}

\begin{answer}
We used that $\diff{}{\theta}\sin\theta = \cos\theta$. That is true only if $\theta$ is given in radians, not degrees.
\end{answer}

\begin{solution}
We used that $\diff{}{\theta}\sin\theta = \cos\theta$. That is true only if $\theta$ is given in radians, not degrees. 
(See Lemma \eref{CLP100}{lem DIFFsincos} and 
Warning \eref{CLP100}{warning:radians} in the CLP-1 text.) 
So we have to convert $2^\circ$
to radians, which is $2\times\frac{\pi}{180}=\frac{\pi}{90}$.
The correct computation is
\begin{align*}
Y\big(0.9,\tfrac{\pi}{90}\big) = Y\big(1-0.1\,,\, 0+\tfrac{\pi}{90}\big)
   \approx 0\ +\ (0)\,(-0.1)\ +\ (1)\big(\tfrac{\pi}{90}\big)
         =\tfrac{\pi}{90}
   \approx 0.035
\end{align*} 
Just out of general interest,
$
0.9\sin\tfrac{\pi}{90} =0.0314
$
to four decimal places.

\end{solution}

%%%%%%%%%%%%%%%%%%
\subsection*{\Procedural}
%%%%%%%%%%%%%%%%%%

%%%%%%%%%%%%%%%%%%%%%%%%%%%%%%%%
\begin{question}
Find an approximate value for $f(x,y)=\sin(\pi xy+\ln y)$ at $(0.01,1.05)$
without using a calculator or computer.
\end{question}

%\begin{hint}
%
%\end{hint}

\begin{answer}
$0.01\,\pi + 0.05 \approx 0.0814$
\end{answer}

\begin{solution}
Apply the linear approximation 
$f(0.01,1.05)\approx f(0,1)+f_x(0,1)(0.01)+f_y(0,1)(0.05)$,
with 
\begin{align*}
f(x,y)&=\sin(\pi xy+\ln y) & f(0,1)&=\sin 0=0\\
f_x(x,y)&=\pi y\cos(\pi xy+\ln y) & f_x(0,1)&=\pi\cos 0=\pi\\
f_y(x,y)&=\left(\pi x+\frac{1}{y}\right)\cos(\pi xy+\ln y) & 
             f_y(0,1)&=\cos 0=1
\end{align*}
This gives
\begin{align*}
f(0.01,1.05)&\approx f(0,1)+f_x(0,1)(0.01)+f_y(0,1)(0.05)
=0+\pi(0.01)+1(0.05) \\
&=0.01\,\pi + 0.05
\approx 0.0814
\end{align*}

\end{solution}

%%%%%%%%%%%%%%%%%%%%%%%%%%%%%%%%
\begin{question}[M200 2009A] %1b
Let $\displaystyle f(x,y)=\frac{x^2y}{x^4+2y^2}$.
Find an approximate value for $f (-0.9\,,\, 1.1)$
without using a calculator or computer.
\end{question}

%\begin{hint}
%
%\end{hint}

\begin{answer}
$0.3$
\end{answer}

\begin{solution}
We are going to need
the first order derivatives of $f(x,y)$ at $(x,y)=(-1,1)$.
So we find them first.
\begin{alignat*}{3}
f_x(x,y)&=\frac{2xy}{x^4+2y^2}-\frac{x^2y(4x^3)}{{(x^4+2y^2)}^2}\qquad &
f_x(-1,1)&=-\frac{2}{3} +\frac{4}{3^2}=-\frac{2}{9}
\\
f_y(x,y)&=\frac{x^2}{x^4+2y^2}-\frac{x^2y(4y)}{{(x^4+2y^2)}^2}\qquad &
f_y(-1,1)&=\frac{1}{3} -\frac{4}{3^2}=-\frac{1}{9}
\end{alignat*}

The linear approximation to $f(x,y)$ about $(-1,1)$ is
\begin{align*}
f(x,y)\approx=f(-1,1) + f_x(-1,1)\,(x+1) + f_y(-1,1)\,(y-1)
 =\frac{1}{3} -\frac{2}{9}\,(x+1) -\frac{1}{9}\,(y-1)
\end{align*}
In particular
\begin{align*}
f(-0.9,1.1)\approx \frac{1}{3} -\frac{2}{9}\,(0.1) -\frac{1}{9}\,(0.1)
=\frac{27}{90}=0.3
\end{align*}
\end{solution}

%%%%%%%%%%%%%%%%%%%%%%%%%%%%%%%%
\begin{question}
Four numbers, each at least zero and each at most 50, are rounded to
the first decimal place and then multiplied together. Estimate the maximum
possible error in the computed product.
\end{question}

\begin{hint}
Let the four numbers be $x_1$, $x_2$, $x_3$ and $x_4$.
Let the four rounded numbers be $x_1+\veps_1$, $x_2+\veps_2$, $x_3+\veps_3$ 
and $x_4+\veps_4$. If 
$P(x_1,x_2,x_3,x_4)=x_1x_2x_3x_4$, then the error in the product 
introduced by rounding is
$\big|P(x_1+\veps_1,x_2+\veps_2,x_3+\veps_3,x_4+\veps_4)
-P(x_1,x_2,x_3,x_4)\big|$.
\end{hint}

\begin{answer}
$25000$
\end{answer}

\begin{solution}
Let the four numbers be $x_1$, $x_2$, $x_3$ and $x_4$.
Let the four rounded numbers be $x_1+\veps_1$, $x_2+\veps_2$, $x_3+\veps_3$ 
and $x_4+\veps_4$. Then $0\le x_1,x_2,x_3,x_4\le 50$ and 
$|\veps_1|,|\veps_2|,|\veps_3|,|\veps_4|\le 0.05$. If 
$P(x_1,x_2,x_3,x_4)=x_1x_2x_3x_4$, then the error in the product 
introduced by rounding is, using the four variable variant of the 
linear approximation (\eref{CLP200}{eqn lin approx 3d}) of the CLP-3 text,
\begin{align*}
&\big|P(x_1+\veps_1,x_2+\veps_2,x_3+\veps_3,x_4+\veps_4)-P(x_1,x_2,x_3,x_4)\big|
\\
&\approx\big|\pdiff{P}{x_1}(x_1,x_2,x_3,x_4)\veps_1
+\pdiff{P}{x_2}(x_1,x_2,x_3,x_4)\veps_2
+\pdiff{P}{x_3}(x_1,x_2,x_3,x_4)\veps_3
+\pdiff{P}{x_4}(x_1,x_2,x_3,x_4)\veps_4\big|\\
&=\big|x_2x_3x_4\veps_1
+x_1x_3x_4\veps_2
+x_1x_2x_4\veps_3
+x_1x_2x_3\veps_4\big|\\
&\le 4\times 50\times 50\times 50\times 0.05
=25000
\end{align*}
\end{solution}

%%%%%%%%%%%%%%%%%%%%%%%%%%%%%%%%
\begin{question}[M200 2005D] %1
One side of a right triangle is measured to be $3$ with a maximum 
possible error of $\pm 0.1$, and the other side is measured to be 
$4$ with a maximum possible error of $\pm 0.2$.
Use the linear approximation to estimate the maximum possible error
in calculating the length of the hypotenuse of the right triangle.
\end{question}

\begin{hint}
Use Pythagoras to express the length of the hypotenuse
in terms of the lengths of the other two sides.
\end{hint}

\begin{answer}
$0.22$
\end{answer}

\begin{solution}
Denote by $x$ and $y$ the lengths of sides with $x=3\pm 0.1$
and $y=4\pm 0.2$. Then the length of the hypotenuse is 
$f(x,y)=\sqrt{x^2+y^2}$. Note that
\begin{alignat*}{5}
f(x,y)&=\sqrt{x^2+y^2} &
         f(3,4)&=5 \\
f_x(x,y)&=\frac{x}{\sqrt{x^2+y^2}}\qquad &
         f_x(3,4)&=\frac{3}{5} \\
f_y(x,y)&=\frac{y}{\sqrt{x^2+y^2}} &
         f_y(3,4)&=\frac{4}{5} 
\end{alignat*}
By the linear approximation
\begin{align*}
f(x,y)\approx f(3,4) + f_x(3,4)\,(x-3) + f_y(3,4)\,(y-4)
      =5 +\frac{3}{5}\,(x-3)  +\frac{4}{5}\,(y-4)
\end{align*}
So the approximate maximum error in calculating the length of the hypotenuse
is
\begin{align*}
\frac{3}{5}\,(0.1)  +\frac{4}{5}\,(0.2)
=\frac{1.1}{5} = 0.22
\end{align*}

\end{solution}


%%%%%%%%%%%%%%%%%%%%%%%%%%%%%%%%
\begin{question}[M200 2006A] %1
If two resistors of resistance $R_1$ and $R_2$ are wired in parallel, 
then the resulting resistance R satisfies the equation
$\frac{1}{R} =\frac{1}{R_1}+\frac{1}{R_2}$. Use the linear
approximation to estimate the change in $R$ if $R_1$ decreases from 
$2$ to $1.9$ ohms and $R_2$ increases from 8 to 8.1 ohms.
\end{question}

%\begin{hint}
%
%\end{hint}

\begin{answer}
$-0.06$
\end{answer}

\begin{solution}
The function $R(R_1,R_2)$ is defined implictly by
\begin{equation*}
\frac{1}{R(R_1,R_2)} =\frac{1}{R_1}+\frac{1}{R_2}
\tag{$*$}\end{equation*}
In particular
\begin{equation*}
\frac{1}{R(2,8)} = \frac{1}{2}+\frac{1}{8} = \frac{5}{8}
\implies R(2,8) = \frac{8}{5}
\end{equation*}

We wish to use the linear approximation
\begin{equation*}
R(R_1,R_2) \approx R(2,8) + \pdiff{R}{R_1}(2,8)\,(R_1-2)
                          + \pdiff{R}{R_2}(2,8)\,(R_2-8)
\end{equation*}
To do so, we need the partial derivatives $\pdiff{R}{R_1}(2,8)$
and $\pdiff{R}{R_2}(2,8)$. To find them, we differentiate $(*)$ with respect to
$R_1$ and $R_2$:
\begin{align*}
-\frac{1}{R(R_1,R_2)^2}\pdiff{R}{R_1}(R_1,R_2)
                 &= -\frac{1}{R_1^2}  \\
-\frac{1}{R(R_1,R_2)^2}\pdiff{R}{R_2}(R_1,R_2)
                 &= -\frac{1}{R_2^2}  \\
\end{align*}
Setting $R_1=2$ and $R_2=8$ gives
\begin{alignat*}{5}
-\frac{1}{(8/5)^2}\pdiff{R}{R_1}(2,8)
                 &= -\frac{1}{4} &\quad\implies\quad &
         &\pdiff{R}{R_1}(2,8) &=\frac{16}{25} \\
-\frac{1}{(8/5)^2}\pdiff{R}{R_2}(2,8)
                 &= -\frac{1}{64}  &\quad\implies\quad &
         &\pdiff{R}{R_2}(2,8) &=\frac{1}{25}
\end{alignat*}
So the specified change in $R$ is
\begin{align*}
R(1.9,8.1)-R(2,8) \approx  \frac{16}{25} (-0.1) + \frac{1}{25}(0.1)
=-\frac{15}{250} = -0.06
\end{align*}
\end{solution}

%%%%%%%%%%%%%%%%%%%%%%%%%%%%%%%%
\begin{question}
The total resistance $R$ of three resistors, $R_1$, $R_2$, $R_3$, connected
in parallel is determined by
\begin{equation*}
\frac{1}{R}=\frac{1}{R_1}+\frac{1}{R_2}+\frac{1}{R_3}
\end{equation*}
If the resistances, measured in Ohms, are $R_1=25\Om$, $R_2=40\Om$ and
$R_3=50\Om$, with a possible error of 0.5\% in each case, estimate the
maximum error in the calculated value of $R$.
\end{question}

\begin{hint}
Review the relationship between absolute error and percentage error
given in Definition \eref{CLP200}{def error} in the CLP-3 text.
\end{hint}

\begin{answer}
$\frac{1}{17}\approx 0.059$

\end{answer}

\begin{solution}
First, we compute the values of the partial derivatives of
$R(R_1,R_2,R_3)$ at the measured values of $R_1$, $R_2$, $R_3$. Applying
$\pdiff{}{R_i}$, with $i=1,2,3$ to both sides of
the defining equation 
\begin{equation*}
\frac{1}{R(R_1,R_2,R_3)}=\frac{1}{R_1}+\frac{1}{R_2}+\frac{1}{R_3}
\end{equation*}
for $R(R_1,R_2,R_3)$ gives
\begin{align*}
-\frac{1}{R(R_1,R_2,R_3)^2}\ \pdiff{R}{R_1}(R_1,R_2,R_3)
&=-\frac{1}{R_1^2}\\
-\frac{1}{R(R_1,R_2,R_3)^2}\ \pdiff{R}{R_2}(R_1,R_2,R_3)
&=-\frac{1}{R_2^2}\\
-\frac{1}{R(R_1,R_2,R_3)^2}\ \pdiff{R}{R_3}(R_1,R_2,R_3)
&=-\frac{1}{R_3^2}
\end{align*}
When $R_1=25\Om$, $R_2=40\Om$ and $R_3=50\Om$
\begin{equation*}
\frac{1}{R(25,40,50)}=\frac{1}{25}+\frac{1}{40}+\frac{1}{50}
=\frac{8+5+4}{200}\implies
R(25,40,50)=\frac{200}{17}=11.765
\end{equation*}
Substituting in these values of $R_1$, $R_2$, $R_3$ and $R$,
\begin{alignat*}{5}
\pdiff{R}{R_1}(25,40,50)
&=\frac{R(25,40,50)^2}{25^2}&=\frac{64}{17^2}&=0.221\\
\pdiff{R}{R_2}(25,40,50)
&=\frac{R(25,40,50)^2}{40^2}&=\frac{25}{17^2}&=0.0865\\
\pdiff{R}{R_3}(25,40,50)
&=\frac{R(25,40,50)^2}{50^2}&=\frac{16}{17^2}&=0.0554
\end{alignat*}
If the absolute errors in measuring $R_1$, $R_2$ and $R_3$ are denoted 
$\veps_1$, $\veps_2$ and $\veps_3$, respectively, then,
, using the linear approximation (\eref{CLP200}{eqn lin approx 3d}) 
of the CLP-3 text, the corresponding error $E$ in $R$ is
\begin{align*}
E &= R(25+\veps_1,40+\veps_2,50+\veps_3)-R(25,40,50) \\
&\approx \pdiff{R}{R_1}(25,40,50)\veps_1
       +\pdiff{R}{R_2}(25,40,50)\veps_2
       +\pdiff{R}{R_3}(25,40,50)\veps_3
\end{align*}
and obeys
\begin{align*}
|E|&\le \frac{64}{17^2}|\veps_1|+\frac{25}{17^2}|\veps_2|
                +\frac{16}{17^2}|\veps_3| \\
\text{or}\quad|E|&\le 0.221|\veps_1|+0.0865|\veps_2|+0.0554|\veps_3|
\end{align*}
We are told that the percentage error in each measurement is no more that
$0.5\%$. So
\begin{equation*}
|\veps_1|\le \frac{0.5}{100}25=\frac{1}{8}=0.125\qquad
|\veps_2|\le \frac{0.5}{100}40=\frac{1}{5}=0.2\qquad
|\veps_3|\le \frac{0.5}{100}50=\frac{1}{4}=0.25
\end{equation*}
so that
\begin{align*}
|E|&\le \frac{8}{17^2}+\frac{5}{17^2}+\frac{4}{17^2}
=\frac{1}{17}\\
\text{or}\quad|E|&\le 0.221\times0.125+0.0865\times0.2+0.0554\times0.25
=0.059
\end{align*}
\end{solution}

%%%%%%%%%%%%%%%%%%%%%%%%%%%%%%%%
\begin{question}
The specific gravity $S$ of an object is given by $\ S=\frac{A}{A-W}\ $
where $A$ is the weight of the object in air and $W$ is the weight of the
object in water. If $\ A=20\pm .01\ $ and $\ W=12\pm.02\ $ find the 
approximate percentage error in calculating $S$ from the given measurements.
\end{question}

\begin{hint}
Be very careful about signs. There is a trap hidden in this question.
As an example of the trap, suppose you know that $|\veps_1|\le 0.2$
and $|\veps_2|\le 0.1$. It does \emph{not} follow from this
that $\big|\veps_1-\veps_2|\le 0.2 -0.1 =0.1$. The reason is that 
it is possible to have $\veps_1=0.2$ and $\veps_2=-0.1$ and then
$\veps_1-\veps_2=0.3$. The correct way to bound $\big|\veps_1-\veps_2|$
is
\begin{equation*}
\big|\veps_1-\veps_2| \le |\veps_1|+ |\veps_2|\le 0.2+0.1\le 0.3
\end{equation*}
\end{hint}

\begin{answer}
$\frac{13}{40}\%=0.325\%$
\end{answer}

\begin{solution}
By the linear approximation
\begin{align*}
\De S\approx \pdiff{S}{A}(20,12)\,\De A
+\pdiff{S}{W}(20,12)\,\De W
\end{align*}
with $S(A,W)=\frac{A}{A-W} =1+\frac{W}{A-W}$. So
\begin{align*}
S(A,W)&=\frac{A}{A-W} & S(20,12)&=\frac{20}{8}=\frac{5}{2}\\
S_A(A,W)&=-\frac{W}{(A-W)^2} & S_A(20,12)&=-\frac{12}{8^2}=-\frac{3}{16}\\
S_W(A,W)&=\frac{A}{(A-W)^2} & S_W(20,12)&=\frac{20}{8^2}=\frac{5}{16}
\end{align*}
For any given $\De A$ and $\De W$, the percentage error is
\begin{equation*}
\left|100\frac{\De S}{S}\right|
=\left|100\frac{2}{5}\Big(-\frac{3}{16}\De A+\frac{5}{16}\De W\Big)\right|
\end{equation*}
We are told that $|\De A|\le 0.01$ and $|\De W|\le 0.02$. To maximize
$\big|100\frac{2}{5}\big(-\frac{3}{16}\De A+\frac{5}{16}\De W\big)\big|$
take $\De A=- 0.01$ and $\De W=+ 0.02$. So the maximum percentage error
is
\begin{equation*}
100\frac{2}{5}\left[-\frac{3}{16}(- 0.01)+\frac{5}{16}(0.02)\right]
= \frac{2}{5}\times \frac{13}{16}
=\frac{13}{40}=0.325\%
\end{equation*}
\end{solution}



%%%%%%%%%%%%%%%%%%%%%%%%%%%%%%%%
\begin{question}[M200 2008D] %2
The pressure in a solid is given by
\begin{equation*}
P(s,r) = sr(4s^2 - r^2 - 2)
\end{equation*}
where $s$ is the specific heat and $r$ is the density. We expect to measure 
$(s,r)$ to be approximately $(2,2)$ and would like to have the most 
accurate value for $P$.
There are two different ways to measure $s$ and $r$. Method $1$ 
has an error in $s$ of $\pm 0.01$ and an error in $r$ of $\pm 0.1$, 
while method 2 has an error of $\pm 0.02$ for both $s$ and $r$.

Should we use method 1 or method 2? Explain your reasoning carefully.
\end{question}

%\begin{hint}
%
%\end{hint}

\begin{answer}
Method 1 is better.
\end{answer}

\begin{solution}
The linear approximation to $P(s,r)$ at $(2,2)$ is
\begin{align*}
P(s,r)\approx P(2,2) +P_s(2,2)\,(s-2) + P_r(2,2)\,(r-2)
\end{align*} 
As
\begin{align*}
P(2,2) &= (2)(2)\big[4(2)^2-(2)^2-2\big] = 40
         \qquad\text{(which we don't actually need) } \\
P_s(2,2) &= \Big[12s^2r-r^3-2r\Big]_{s=r=2} = 84 \\
P_r(2,2) &= \Big[4s^3-3sr^2-2s\Big]_{s=r=2} = 4
\end{align*}
the linear approximation is
\begin{align*}
P(s,r)\approx 40 +84\,(s-2) + 4\,(r-2)
\end{align*}
Under method 1, the maximum error in $P$ will have magnitude
at most (approximately)
\begin{align*}
84(0.01) + 4(0.1) = 1.24
\end{align*}
Under method 2, the maximum error in $P$ will have magnitude
at most (approximately)
\begin{align*}
84(0.02) + 4(0.02) = 1.76
\end{align*}
Method 1 is better.
\end{solution}

%%%%%%%%%%%%%%%%%%%%%%%%%%%%%%%%
\begin{question}
A rectangular beam that is supported at its two ends and is subjected
to a uniform load sags by an amount 
\begin{equation*}
S=C\frac{p\ell^4}{w h^3}
\end{equation*}
where $p={\rm load}$, 
      $\ell={\rm length}$, 
      $h={\rm height}$, 
      $w={\rm width}$ and $C$ is a constant. 
Suppose $p\approx 100$, $\ell\approx 4$, $w\approx .1$ and $h\approx.2$. 
Will the sag of the beam be more sensitive to changes in the height of the 
beam or to changes in the width of the beam.
\end{question}

\begin{hint}
Determine, approximately, the change in sag when the height changes by a small
amount $\veps$ and also when the width changes by a small amount $\veps$.
Which is bigger?
\end{hint}

\begin{answer}
The sag will be more sensitive to changes in height.
\end{answer}

\begin{solution}
Using the four variable variant of the linear approximation
(\eref{CLP200}{eqn lin approx 3d}) of the CLP-3 text,
\begin{equation*}
\De S\approx \pdiff{S}{p}\De p
       +\pdiff{S}{\ell}\De \ell
       +\pdiff{S}{w}\De w
       +\pdiff{S}{h}\De h
   =C\frac{p\ell^4}{w h^3}\left[\frac{\De p}{p}+4\frac{\De\ell}{\ell}
                -\frac{\De w}{w}-3\frac{\De h}{h}   \right]
\end{equation*}
When $w\approx0.1$ and $h\approx0.2$,
\begin{equation*}
\frac{\De w}{w}\approx 10\De w\qquad
3\frac{\De h}{h}\approx 15\De h
\end{equation*}
So a change in height by $\De h=\veps$ produces a change in sag of 
about $\De S = 15\veps$ times $-C\frac{p\ell^4}{w h^3}$,
while a change $\De w$ in width by the same $\veps$ produces a change in 
sag of about $\De S=10\veps$ times the same $-C\frac{p\ell^4}{w h^3}$.
The sag is more sensitive to $\De h$.
\end{solution}


%%%%%%%%%%%%%%%%%%%%%%%%%%%%%%%%
\begin{question}[M200 2010D] %1c
Let $z=f(x,y)=\frac{2y}{x^2+y^2}$.
Find an approximate value for $f(-0.8,2.1)$.
\end{question}

%\begin{hint}
%
%\end{hint}

\begin{answer}
$0.84$
\end{answer}

\begin{solution}
The first order partial derivatives of $f$ are
\begin{alignat*}{3}
f_x(x,y) & = -\frac{4xy}{{(x^2+y^2)}^2}\quad &
      f_x(-1,2) & = \frac{8}{25} \\
f_y(x,y) & = \frac{2}{x^2+y^2}-\frac{4y^2}{{(x^2+y^2)}^2}\quad &
      f_y(-1,2) & = \frac{2}{5}-\frac{16}{25}
                  =-\frac{6}{25} \\
\end{alignat*}
The linear approximation of $f(x,y)$ about $(-1,2)$ is
\begin{align*}
f(x,y)&\approx f(-1,2) + f_x(-1,2)\,(x+1) + f_y(-1,2)\,(y-2) \\
      &=\frac{4}{5} +\frac{8}{25}\,(x+1) - \frac{6}{25}\,(y-2)
\end{align*}
In particular, for $x=-0.8$ and $y=2.1$,
\begin{align*}
f(-0.8,2.1)&\approx \frac{4}{5} +\frac{8}{25}\,(0.2) - \frac{6}{25}\,(0.1) \\
           &=0.84
\end{align*}
\end{solution}

%%%%%%%%%%%%%%%%%%%%%%%%%%%%%%%%
\begin{question}[M200 2012D] %3
Suppose that a function $z = f (x, y)$ is implicitly defined by an equation:
\begin{equation*}
         xyz + x + y^2 + z^3 = 0
\end{equation*}
\begin{enumerate}[(a)]
\item
Find $\pdiff{z}{x}$.
\item
If $f(-1, 1) < 0$, find the linear approximation of the function 
$z = f (x, y)$ at $(-1, 1)$.
\item
If $f(-1, 1) < 0$, use the linear approximation in (b) to approximate $f(-1.02, 0.97)$.
\end{enumerate}
\end{question}

%\begin{hint}
%
%\end{hint}

\begin{answer}
(a) $f_x(x,y) = -\frac{y\,f(x,y) +1}{3f(x,y)^2 +xy}$\qquad
(b) $f(x,y) \approx  -1 -\frac{3}{2} (y-1)$\qquad
(c) $-0.955$
\end{answer}

\begin{solution}
(a) The function $f(x,y)$ obeys
\begin{equation*}
xy\,f(x,y) + x + y^2 + f(x,y)^3 =0
\tag{$*$}\end{equation*} 
for all $x$ and $y$ (sufficiently close to $(-1,1)$). 
Differentiating $(*)$ with respect to $x$ gives
\begin{align*}
y\,f(x,y) +xy\,f_x(x,y) +  1  + 3f(x,y)^2 f_x(x,y) = 0
\implies
f_x(x,y) = -\frac{y\,f(x,y) +1}{3f(x,y)^2 +xy}
\end{align*}
Without knowing $f(x,y)$ explicitly, there's not much that we can do with this.

(b) $f(-1,1)$ obeys
\begin{equation*}
(-1)(1)\,f(-1,1) + (-1) + (1)^2 + f(-1,1)^3 =0
\iff f(-1,1)^3 -f(-1,1) =0
\end{equation*}
Since $f(-1, 1) < 0$ we may divide this equation by $f(-1, 1) < 0$,
giving $f(-1,1)^2 - 1=0$. Since $f(-1, 1) < 0$, we must have $f(-1, 1)=-1$.
By part (a) 
\begin{align*}
f_x(-1, 1) = -\frac{(1)\,f(-1, 1) +1}{3f(-1, 1)^2 +(-1)(1)}
           = 0
\end{align*}
To get the linear approximation, we still need $f_y(-1,1)$.
Differentiating $(*)$ with respect to $y$ gives
\begin{equation*}
x\,f(x,y) + xy\,f_y(x,y)  + 2y + 3f(x,y)^2 f_y(x,y) =0
\end{equation*} 
Then setting $x=-1$, $y=1$ and $f(-1,1)=-1$ gives
\begin{equation*}
(-1)\,(-1) + (-1)(1)\,f_y(-1,1)  + 2(1) + 3(-1)^2 f_y(-1,1) =0
\implies f_y(-1,1) =-\frac{3}{2}
\end{equation*} 
So the linear approximation is
\begin{equation*}
f(x,y) \approx f(-1,1) + f_x(-1,1)\,(x+1) + f_y(-1,1)\,(y-1)
           = -1 -\frac{3}{2} (y-1)
\end{equation*}

(c) By part (b),
\begin{align*}
f(-1.02,0.97) \approx  -1 -\frac{3}{2} (0.97-1) =-0.955
\end{align*}
\end{solution}

\begin{question}[M200 2014D] %3
Let $z = f(x,y)$ be given implicitly by
\begin{equation*}
  e^z + yz = x + y.
\end{equation*}

\begin{enumerate}[(a)]
\item
Find the differential $\dee{z}$.
\item
Use linear approximation at the point $(1,0)$ to approximate $f(0.99,0.01)$.
\end{enumerate}
\end{question}

%\begin{hint}
%
%\end{hint}

\begin{answer}
(a) The differential at $x=a$, $y=b$ is
$
\frac{\dee{x}}{e^{f(a,b)}+b} + \frac{1-f(a,b)}{e^{f(a,b)}+b} \,\dee{y}
$

(b) $f\big(0.99\,,\,0.01\big) \approx 0$
\end{answer}

\begin{solution}
By definition, the differential at $x=a$, $y=b$ is
\begin{equation*}
f_x(a,b)\,\dee{x} + f_y(a,b)\,\dee{y}
\end{equation*}
so we have to determine the partial derivatives $f_x(a,b)$ and $f_y(a,b)$.
We are told that
\begin{equation*}
e^{f(x,y)} + y\,f(x,y) = x + y
\end{equation*}
for all $x$ and $y$.
Differentiating this equation with respect to $x$
and with respect to $y$ gives, by the chain rule,
\begin{align*}
e^{f(x,y)}f_x(x,y) + y\,f_x(x,y) = 1 \\
e^{f(x,y)}f_y(x,y) + f(x,y) +y\,f_y(x,y) = 1 
\end{align*}
Solving the first equation for $f_x$ and the second for $f_y$ gives
\begin{align*}
f_x(x,y) &=  \frac{1}{e^{f(x,y)}+y} \\
f_y(x,y) &= \frac{1-f(x,y)}{e^{f(x,y)}+y} 
\end{align*}
So the differential at $x=a$, $y=b$ is
\begin{equation*}
\frac{\dee{x}}{e^{f(a,b)}+b} + \frac{1-f(a,b)}{e^{f(a,b)}+b} \,\dee{y}
\end{equation*} 
Since we can't solve explicitly for $f(a,b)$ for general $a$ and $b$.
There's not much more that we can do with this. 

(b) In particular, when $a=1$ and $b=0$, we have 
\begin{equation*}
e^{f(1,0)} + 0\,f(1,0) = 1 + 0
\implies 
e^{f(1,0)}  = 1 
\implies 
f(1,0)=0
\end{equation*}
and the linear approximation simpifies to
\begin{equation*}
f\big(1+\dee{x}\,,\,\dee{y}\big)
  \approx f(1,0) + \frac{\dee{x}}{e^{f(1,0)}+0} 
        + \frac{1-f(1,0)}{e^{f(1,0)}+0}   \,\dee{y}
  = \dee{x} +\dee{y}
\end{equation*}
Choosing $\dee{x} = -0.01$ and $\dee{y} = 0.01$, we have
\begin{equation*}
f\big(0.99\,,\,0.01\big)
  \approx  -0.01 + 0.01
  =0
\end{equation*}
\end{solution}

%%%%%%%%%%%%%%%%%%%%%%%%%%%%%%%%
\begin{question}[M200 2004A] %2
Two sides and the enclosed angle of a triangle are measured to be
$3\pm.1$m, $4\pm.1$m and $90\pm 1^\circ$ respectively. The length of the
third side is then computed using the cosine law $C^2=A^2+B^2-2AB\cos\theta$.
What is the approximate maximum error in the computed value of $C$?
\end{question}

\begin{hint}
$1^\circ = \frac{\pi}{180}$ radians
\end{hint}

\begin{answer}
$\frac{\pi}{75}+0.14\approx 0.182$
\end{answer}

\begin{solution}
Let $C(A,B,\theta)=\sqrt{A^2+B^2-2AB\cos\theta}$. Then 
$C\big(3,4,\frac{\pi}{2}\big)=5$. Differentiating 
$C^2=A^2+B^2-2AB\cos\theta$ gives
\begin{alignat*}{5}
2C\frac{\partial C}{\partial A}(A,B,\theta)&=2A-2B\cos\theta &\quad&\implies\quad &
10\frac{\partial C}{\partial A}\big(3,4,\tfrac{\pi}{2}\big)&=6\cr
2C\frac{\partial C}{\partial B}(A,B,\theta)&=2B-2A\cos\theta &  &\implies &
10\frac{\partial C}{\partial B}\big(3,4,\tfrac{\pi}{2}\big)&=8\cr
2C\frac{\partial C}{\partial \theta}(A,B,\theta)&=2AB\sin\theta &  &\implies &
10\frac{\partial C}{\partial \theta}\big(3,4,\tfrac{\pi}{2}\big)&=24\cr
\end{alignat*}
Hence the approximate maximum error in the computed value of $C$ is
\begin{align*}
|\De C|&\approx \left|
\frac{\partial C}{\partial A}\big(3,4,\tfrac{\pi}{2}\big)\De A
+\frac{\partial C}{\partial B}\big(3,4,\tfrac{\pi}{2}\big)\De B
+\frac{\partial C}{\partial \theta}\big(3,4,\tfrac{\pi}{2}\big)\De\theta\right|
\\
&\le(0.6)(0.1)+(0.8)(0.1)+(2.4)\frac{\pi}{180}\cr
&=\frac{\pi}{75}+0.14\le 0.182
\end{align*}
\end{solution}

%%%%%%%%%%%%%%%%%%%%%%%%%%%%%%%%%%%%
\begin{question} [M200 2003D] %2
Use differentials to find a reasonable approximation to the
value of $f(x,y)=xy\sqrt{x^2+y^2}$ at $x=3.02$, $y=3.96$. Note that
$3.02\approx 3$ and $3.96\approx 4$.
\end{question}

%\begin{hint}
%\end{hint}

\begin{answer}
$59.560$
\end{answer}

\begin{solution}
Substituting $(x_0,y_0)=(3,4)$ and $(x,y)=(3.02,3.96)$ into
\begin{equation*}
f(x,y)\approx f(x_0,y_0)+f_x(x_0,y_0)(x-x_0)+f_y(x_0,y_0)(y-y_0)
\end{equation*}
gives 
\begin{align*}
f(3.02,3.96)&\approx f(3,4)+0.02f_x(3,4)-0.04f_y(3,4)\\
&=60+0.02\left(20+\frac{36}{5}\right)-0.04\left(15+\frac{48}{5}\right)\\
&=59.560
\end{align*}
since
\begin{equation*}
f_x(x,y)=y\sqrt{x^2+y^2}+\frac{x^2y}{\sqrt{x^2+y^2}}\qquad
f_y(x,y)=x\sqrt{x^2+y^2}+\frac{xy^2}{\sqrt{x^2+y^2}}
\end{equation*}
\end{solution}

%%%%%%%%%%%%%%%%%%%%%%%%%%%%%%%%
\begin{question}[M200 2000D] %2
Use differentials to estimate the volume of metal in a closed
metal can with diameter 8cm and height 12cm if the metal is 0.04cm thick. 
\end{question}

%\begin{hint}
%
%\end{hint}

\begin{answer}
$\pi\times 128\times 0.04=
5.12\pi\approx 16.1$cc
\end{answer}

\begin{solution}
The volume of a cylinder of diameter $d$ and height $h$ 
is $V(d,h)=\pi\big(\frac{d}{2}\big)^2h$. The wording of the question is
a bit ambiguous in that it does not specify if the given dimensions are
inside dimensions or outside dimensions. Assume that they are outside
dimensions. Then the volume of the can, including the metal, is $V(8,12)$
and the volume of the interior, excluding the metal, is 
\begin{align*}
V(8-2\times0.04\,,\,12-2\times 0.04)
&\approx  V(8,12)
+V_d(8,12)(-2\times 0.04)
+V_h(8,12)(-2\times 0.04)\\
&=V(8,12)
+\frac{1}{2}\pi\times 8\times 12\times(-2\times 0.04)
+\pi\left(\frac{8}{2}\right)^2\!\!(-2\times 0.04)\\
&=V(8,12)
-\pi\times 128\times 0.04
\end{align*}
So the volume of metal is approximately $\pi\times 128\times 0.04=
5.12\pi\approx 16.1$cc. (To this level of approximation, it doesn't matter
whether the dimensions are inside or outside dimensions.)
\end{solution}

%%%%%%%%%%%%%%%%%%%%%%%%%%%%%%%%
\begin{question}[M200 2000A] %4
Let $z$ be a function of $x$, $y$ such that 
\begin{equation*}
z^3 - z + 2xy - y^2 = 0,\qquad z(2, 4) = 1.
\end{equation*}
\begin{enumerate}[(a)]
\item
Find the linear approximation to $z$ at the point $(2, 4)$. 
\item 
Use your answer in (a) to estimate the value of $z$ at 
$(2.02, 3.96)$. 
\end{enumerate}
\end{question}

%\begin{hint}
%
%\end{hint}

\begin{answer}
(a) $z(x,y)\approx 1-4x+2y$\qquad
(b) $0.84$
\end{answer}

\begin{solution}
(a) The function $z(x,y)$ obeys
\begin{equation*}
z(x,y)^3-z(x,y)+2xy-y^2=0
\end{equation*}
for all $(x,y)$ near $(2,4)$. Differentiating with respect to $x$ and $y$
\begin{align*}
3z(x,y)^2\pdiff{z}{x}(x,y)
       -\pdiff{z}{x}(x,y)+2y&=0 \\
3z(x,y)^2\pdiff{z}{y}(x,y)
       -\pdiff{z}{y}(x,y)+2x-2y&=0 
\end{align*}
Substituting in $x=2$, $y=4$ and $z(2,4)=1$ gives
\begin{alignat*}{5}
3\pdiff{z}{x}(2,4)
       -\pdiff{z}{x}(2,4)+8&=0 &  &\iff &
    \pdiff{z}{x}(2,4) & = -4 \\
3\pdiff{z}{y}(2,4)
       -\pdiff{z}{y}(2,4)+4-8&=0 &  &\iff &
    \pdiff{z}{y}(2,4) & = 2
\end{alignat*}
The linear approximation is
\begin{align*}
z(x,y)&\approx z(2,4)+z_x(2,4)(x-2)+z_y(2,4)(y-4)= 1-4(x-2)+2(y-4) \\
&=1-4x+2y
\end{align*}

(b) Substituting in $x=2.02$ and $y=3.96$ gives
\begin{align*}
z(2.02,3.96)\approx 1-4\times0.02+2\times(-0.04)
=0.84
\end{align*}
\end{solution}




%%%%%%%%%%%%%%%%%%
\subsection*{\Application}
%%%%%%%%%%%%%%%%%%

%%%%%%%%%%%%%%%%%%%%%%%%%%%%%%%%
\begin{question}[M200 2006D] %1
Consider the surface given by:
\begin{equation*}
z^3 - xyz^2 - 4x = 0.
\end{equation*}
\begin{enumerate}[(a)]
\item 
Find expressions for $\pdiff{z}{x}$, $\pdiff{z}{y}$ as functions of $x$, $y$, 
$z$.
\item
Evaluate $\pdiff{z}{x}$, $\pdiff{z}{y}$ at $(1, 1, 2)$.
\item
Measurements are made with errors, so that $x = 1 \pm 0.03$ and 
$y = 1 \pm 0.02$.
Find the corresponding maximum error in measuring $z$.
\item
A particle moves over the surface along the path whose projection in the
$xy$--plane is given in terms of the angle $\theta$ as
\begin{equation*}
x(\theta) = 1 + \cos\theta,\ y(\theta) = \sin\theta
\end{equation*}
from the point $A : x = 2,\ y = 0$ to the point $B : x = 1,\ y = 1$.
Find $\diff{z}{\theta}$ at points $A$ and $B$.
\end{enumerate}
\end{question}

%\begin{hint}
%
%\end{hint}

\begin{answer}
(a) $\pdiff{z}{x} = \frac{4+yz^2}{3z^2-2xyz}$,
    $\pdiff{z}{x} = \frac{xz^2}{3z^2-2xyz}$\qquad
(b) $\pdiff{z}{x}(1,1) = 1$,
    $\pdiff{z}{y}(1,1) = \frac{1}{2}$\qquad
(c) $\pm 0.04$\qquad

(d) At $A$, $ \diff{z}{\theta} =\frac{2}{3}$.\qquad
At $B$, $\diff{z}{\theta}  = -1$.
\end{answer}

\begin{solution}
(a)  We are told that
\begin{equation*}
z(x,y)^3 - xy\,z(x,y)^2 - 4x = 0
\end{equation*}
for all $(x,y)$ (sufficiently near $(1,1)$).
Differentiating this equation with respect to $x$ gives
\begin{align*}
& 3z(x,y)^2\,\pdiff{z}{x}(x,y) -y\, z(x,y)^2 -2xy\,z(x,y)\pdiff{z}{x}(x,y) - 4=0
\\ \implies 
&\pdiff{z}{x} = \frac{4+yz^2}{3z^2-2xyz}
\end{align*}
and differentiating with respect to $y$ gives
\begin{align*}
& 3z(x,y)^2\,\pdiff{z}{y}(x,y) -x\, z(x,y)^2 -2xy\,z(x,y)\pdiff{z}{y}(x,y) =0
\\ \implies 
&\pdiff{z}{y} = \frac{xz^2}{3z^2-2xyz}
\end{align*}

(b) 
When $(x,y,z)=(1,1,2)$,
\begin{align*}
\pdiff{z}{x}(1,1) = \frac{4+(1)(2)^2}{3(2)^2-2(1)(1)(2)} = 1\qquad
\pdiff{z}{y}(1,1) = \frac{(1)(2)^2}{3(2)^2-2(1)(1)(2)} = \frac{1}{2}
\end{align*}

(c) Under the linear approximation at $(1,1)$
\begin{equation*}
z(x,y) \approx z(1,1) + z_x(1,1)\,(x-1) + z_y\,(1,1)\,(y-1)
        = 2 + (x-1) +\frac{1}{2}(y-1)
\end{equation*}
So errors of $\pm 0.03$ in $x$ and $\pm 0.02$ in $y$ leads of errors
of about 
\begin{equation*}
\pm\left[0.03 + \frac{1}{2}(0.02)\right]
=\pm 0.04
\end{equation*}
in $z$.

(d) By the chain rule
\begin{align*}
\diff{}{\theta} z\big(x(\theta),y(\theta)\big)
&= z_x\big(x(\theta),y(\theta)\big)\,x'(\theta)
  +z_y\big(x(\theta),y(\theta)\big)\,y'(\theta) \\
&=-z_x\big(1 + \cos\theta,\sin\theta\big)\,\sin\theta
  +z_y\big(1 + \cos\theta,\sin\theta\big)\,\cos\theta
\end{align*}
At $A$, $x=2$, $y=0$, $z=2$ (since $z^3-(2)(0)z^2-4(2)=0$) and $\theta=0$, 
so that
\begin{align*}
\pdiff{z}{x}(2,0) = \frac{4+(0)(2)^2}{3(2)^2-2(2)(0)(2)} = \frac{1}{3}\qquad
\pdiff{z}{y}(2,0) = \frac{(2)(2)^2}{3(2)^2-2(2)(0)(2)} = \frac{2}{3}
\end{align*}
and
\begin{align*}
\diff{z}{\theta} 
&=-\frac{1}{3}\sin(0) +\frac{2}{3}\cos(0) =\frac{2}{3}
\end{align*}

At $B$, $x=1$, $y=1$, $z=2$ and $\theta=\frac{\pi}{2}$, 
so that, by part (b),
\begin{align*}
\pdiff{z}{x}(1,1) = 1\qquad
\pdiff{z}{y}(1,1) = \frac{1}{2}
\end{align*}
and
\begin{align*}
\diff{z}{\theta} 
&=-\sin\frac{\pi}{2} +\frac{1}{2}\cos\frac{\pi}{2} = -1
\end{align*}
\end{solution}

%%%%%%%%%%%%%%%%%%%%%%%%%%%%%%%%
\begin{question}[M200 2007A] %2
Consider the function $f$ that maps each point $(x, y)$ in $\bbbr^2$ 
to $ye^{-x}$.
\begin{enumerate}[(a)]
\item
 Suppose that $x = 1$ and $y = e$, but errors of size $0.1$ are 
made in measuring each of $x$ and $y$. Estimate the maximum error 
that this could cause in $f(x,y)$.

\item
 The graph of the function $f$ sits in $\bbbr^3$ , and the point $(1, e, 1)$ 
lies on that graph. Find a nonzero vector that is perpendicular to that 
graph at that point.
\end{enumerate}
\end{question}

%\begin{hint}
%
%\end{hint}

\begin{answer}
 (a) $\frac{1+e^{-1}}{10}$\qquad
 (b) any nonzero constant times $\llt -1\,,\,e^{-1}\,,\,-1\rgt$
\end{answer}

\begin{solution}
We are going to need the first order partial derivatives of 
$f(x,y)=ye^{-x}$ at $(x,y)=(1,e)$. Here they are.
\begin{align*}
f_x(x,y)&= -ye^{-x} & f_x(1,e)&=-e\,e^{-1}=-1 \\
f_y(x,y)&=e^{-x}    & f_y(1,e)&= e^{-1}
\end{align*}

(a) The linear approximation to $f(x,y)$ at $(x,y)=(1,e)$ is
\begin{align*}
f(x,y) \approx f(1,e) + f_x(1,e)\,(x-1) +f_y(1,e)\,(y-e)
       = 1 -(x-1) +e^{-1}(y-e)
\end{align*}
The maximum error is then approximately
\begin{align*}
-1(-0.1) +e^{-1}(0.1) =\frac{1+e^{-1}}{10}
\end{align*}

(b) The equation of the graph is $g(x,y,z) = f(x,y) -z =0$.
Any vector that is a nonzero constant times 
\begin{align*}
\vnabla g(1,e,1) =\llt f_x(1,e)\,,\,f_y(1,e)\,,\,-1\rgt
                 =\llt -1\,,\,e^{-1}\,,\,-1\rgt
\end{align*}
is perpendicular to $g=0$ at $(1,e,1)$.
\end{solution}

%%%%%%%%%%%%%%%%%%%%%%%%%%%%%%%%
\begin{question}[M200 2009D] %1
A surface is defined implicitly by $z^4 - xy^2 z^2 + y = 0$.
\begin{enumerate}[(a)]
\item
Compute $\pdiff{z}{x}$, $\pdiff{z}{y}$ in terms of $x$, $y$, $z$.

\item 
Evaluate $\pdiff{z}{x}$ and $\pdiff{z}{y}$ at $(x, y, z) = (2, -1/2, 1)$.

\item
If $x$ decreases from $2$ to $1.94$, and 
   $y$ increases from $-0.5$ to $-0.4$,
find the approximate change in $z$ from $1$.

\item
Find the equation of the tangent plane to the surface at the point 
$(2, -1/2, 1)$.

\end{enumerate}
\end{question}

%\begin{hint}
%
%\end{hint}

\begin{answer}
(a) $\pdiff{z}{x} = \frac{y^2 z^2}{4z^3-2xy^2z}$, 
    $\pdiff{z}{y} = \frac{2xy\, z^2-1}{4z^3-2xy^2 z}$

(b)
$
\pdiff{z}{x}(2,-1/2) =\frac{1}{12}
$,
$
\pdiff{z}{y}(2,-1/2) =-1
$

(c) $f(1.94,-0.4) - 1  \approx -0.105$

(d)  $\frac{x}{12} -y -z = -\frac{1}{3}$
\end{answer}

\begin{solution}
(a) 
We are told that for all $x,y$ (with $(x,y,z)$ near $(2,-1/2,1)$),
the function $z(x,y)$ obeys
\begin{align*}
z(x,y)^4 -xy^2 z(x,y)^2 +y=0
\tag{$*$}\end{align*}
Differentiating $(*)$ with respect to $x$ gives
\begin{align*}
&4z(x,y)^3\pdiff{z}{x}(x,y) - y^2z(x,y)^2 -2xy^2z(x,y)\pdiff{z}{x}(x,y) =0 \\
&\implies 
\pdiff{z}{x}(x,y) = \frac{y^2 z(x,y)^2}{4z(x,y)^3-2xy^2z(x,y)}
\end{align*}
Similarly, differentiating this equation with respect to $y$ gives
\begin{align*}
&4z(x,y)^3\pdiff{z}{y}(x,y) - 2xy\,z(x,y)^2 -2xy^2z(x,y)\pdiff{z}{y}(x,y) +1=0 \\
&\implies 
\pdiff{z}{y}(x,y) = \frac{2xy\, z(x,y)^2-1}{4z(x,y)^3-2xy^2z(x,y)}
\end{align*}

(b) Substituting $(x, y, z) = (2, -1/2, 1)$ into the results of part (a)
gives
\begin{align*}
\pdiff{z}{x}(2,-1/2) &= \frac{1/4}{4-1} =\frac{1}{12}\\
\pdiff{z}{y}(2,-1/2) &= \frac{-2-1}{4-1}=-1
\end{align*}

(c) Under the linear approximation about $(2,-1/2)$,
\begin{align*}
f(x,y) &\approx f(2,-1/2) + f_x(2,-1/2)\,(x-2) + f_y(2,-1/2)\,(y+1/2) \\
       &= 1 +\frac{1}{12}(x-2) - (y+0.5)
\end{align*}
In particular
\begin{align*}
f(1.94,-0.4) \approx 1 -\frac{0.06}{12}-0.1 
\end{align*}
so that
\begin{align*}
f(1.94,-0.4) - 1  \approx -0.105
\end{align*}

(d) The tangent plane is
\begin{align*}
z&=f(2,-1/2) + f_x(2,-1/2)\,(x-2) + f_y(2,-1/2)\,(y+1/2) \\
 &= 1 +\frac{1}{12}(x-2) - (y+0.5)
\end{align*}
or
\begin{align*}
\frac{x}{12} -y -z = -\frac{1}{3}
\end{align*}
\end{solution}

%%%%%%%%%%%%%%%%%%%%%%%%%%%%%%%%
\begin{question}[M200 2010A] %1b
A surface $z = f (x, y)$ has derivatives $\pdiff{f}{x}=3$ and
$\pdiff{f}{y}=-2$ at $(x, y, z) = (1, 3, 1)$.
\begin{enumerate}[(a)]
\item
If $x$ increases from $1$ to $1.2$, and $y$ decreases from $3$ to $2.6$,
find the change in $z$ using a linear approximation.
\item
Find the equation of the tangent plane to the surface at the point $(1, 3, 1)$.
\end{enumerate}

\end{question}

%\begin{hint}
%
%\end{hint}

\begin{answer}
(a) $1.4$\qquad
(b) $3x-2y-z = -4$
\end{answer}

\begin{solution}
(a) The linear approximation to $f(x,y)$ at $(1,3)$ is
\begin{align*}
f(x,y) \approx f(1,3) + f_x(1,3)\,(x-1) +f_y(1,3)\,(y-3)
      = 1 + 3(x-1) -2(y-3)
\end{align*}
So the change is $z$ is approximately
\begin{align*}
3(1.2-1) -2(2.6-3)
= 1.4
\end{align*}

(b) The equation of the tangent plane is
\begin{align*}
z = f(1,3) + f_x(1,3)\,(x-1) +f_y(1,3)\,(y-3) = 1 + 3(x-1) -2(y-3)
\end{align*}
or
\begin{equation*}
3x-2y-z = -4
\end{equation*}

\end{solution}

%%%%%%%%%%%%%%%%%%%%%%%%%%%%%%%%
\begin{question}[M200 2011A] %2
According to van der Waal's equation, a gas satisfies the equation
\begin{equation*}
(pV^2 + 16)(V - 1) = T V^2 ,
\end{equation*}
where $p$, $V$ and $T$ denote pressure, volume and temperature respectively. 
Suppose the gas is now at pressure $1$, volume $2$ and temperature $5$. Find the 
approximate change in its volume if $p$ is increased by $0.2$ and $T$ is increased by $0.3$.
\end{question}

%\begin{hint}
%
%\end{hint}

\begin{answer}
$0.1$
\end{answer}

\begin{solution}
Think of the volume as being the function $V(p,T)$ of pressure and temperature
that is determined implicitly (at least for $p\approx 1$, $T\approx 5$ and $V\approx 2$) by the equation
\begin{equation*}
\big(pV(p,T)^2 + 16\big)\big(V(p,T) - 1\big) = T V(p,T)^2 
\tag{$*$}\end{equation*}
To determine the approximate change in $V$, we will use the linear
approximation to $V(p,T)$ at $p=1$, $T=5$. So we will need the
partial derivatives $V_p(1,5)$ and $V_T(1,5)$. As the equation $(*)$
is valid for all $p$ near $1$ and $T$ near 5, we may differentiate $(*)$
with respect to $p$, giving
\begin{align*}
&\big(V^2 +2 pV V_p\big)\big(V - 1\big) 
+ \big(pV ^2 + 16\big)V_p = 2T V V_p 
\end{align*}
and we may also differentiate $(*)$ with respect to $T$, giving
\begin{align*}
&\big(2 pV V_T\big)\big(V - 1\big) + \big(pV ^2 + 16\big)V_T = V^2 + 2T V V_T
\end{align*}
In particular, when $p=1$, $V=2$, $T=5$,
\begin{alignat*}{3}
\big(4 +4 V_p(1,5)\big)\big(2 - 1\big) 
+ \big(4 + 16\big)V_p(1,5) &= 20 V_p(1,5) 
&\quad\implies
V_p(1,5)&=-1 \\
4 V_T(1,5)\big(2 - 1\big) + \big(4 + 16\big)V_T(1,5) &= 4 + 20 V_T(1,5)
&\quad\implies
V_T(1,5)&=1
\end{alignat*}
so that the change in $V$ is
\begin{align*}
V(1.2\,,\,5.3)-V(1,5) \approx  V_p(1,5)\,(0.2) + V_T(1,5)\,(0.3)
                        = -0.2+0.3
                        = 0.1
\end{align*}
\end{solution}

%%%%%%%%%%%%%%%%%%%%%%%%%%%%%%%%
\begin{question}[M200 2011D] %1b,c
Consider the function $f(x, y) = e^{-x^2 +4y^2}$.
\begin{enumerate}[(a)]
\item
Find the equation of the tangent plane to the graph $z = f (x,y)$ at the 
point where $(x, y) = (2, 1)$.
\item
Find the tangent plane approximation to the value of $f(1.99, 1.01)$ using the
tangent plane from part (a).
\end{enumerate}
\end{question}

%\begin{hint}
%
%\end{hint}

\begin{answer}
(a) $z=1-4x+8y$\qquad
(b) $1.12$
\end{answer}

\begin{solution}
Since
\begin{align*}
f_x(2,1) & = -2x e^{-x^2 +4y^2}\Big|_{(x,y)=(2,1)} = -4 \\
f_y(2,1) & = \phantom{-}8y e^{-x^2 +4y^2}\Big|_{(x,y)=(2,1)} = 8
\end{align*}
The tangent plane to $z=f(x,y)$ at $(2,1)$ is
\begin{align*}
z &= f(2,1) +f_x(2,1)\,(x-2) +f_y(2,1)\,(y-1)
  = 1 -4 (x-2) +8(y-1) \\
 &= 1-4x +8y
\end{align*}
and the tangent plane approximation to the value of $f(1.99, 1.01)$ is
\begin{align*}
f(1.99, 1.01) \approx 1 -4 (1.99-2) +8(1.01-1)
                = 1.12
\end{align*}
\end{solution}

%%%%%%%%%%%%%%%%%%%%%%%%%%%%%%%%
\begin{question}[M200 2012a] %2
Let $z = f (x, y) = \ln(4x^2 + y^2)$.
\begin{enumerate}[(a)]
\item
Use a linear approximation of the function $z = f (x, y)$ at $(0, 1)$ to estimate
$f(0.1, 1.2)$.

\item
Find a point $P (a, b, c)$ on the graph of $z = f(x, y)$ such that 
the tangent plane to the graph of $z = f (x, y)$ at the point $P$ is parallel to the plane $2x + 2y - z = 3$.
\end{enumerate}
\end{question}

%\begin{hint}
%
%\end{hint}

\begin{answer}
(a) $f(0.1,1.2) \approx  0.4$\qquad
(b) $\left(\frac{1}{5}\,,\,\frac{4}{5}\,,\,\ln\frac{4}{5}\right)$
\end{answer}

\begin{solution}
(a) The linear approximation to $f(x,y)$ at $(a,b)$ is
\begin{align*}
f(x,y) &\approx f(a,b)  + f_x(a,b)\,(x-a) + f_y(a,b)\,(y-b) \\
       &= \ln(4a^2 + b^2) + \frac{8a}{4a^2+b^2}\, (x-a)
                              + \frac{2b}{4a^2+b^2}\, (y-b)
\end{align*}
In particular, for $a=0$ and $b=1$,
\begin{align*}
f(x,y) &\approx  2\, (y-1)
\end{align*}
and, for $x=0.1$ and $y=1.2$,
\begin{align*}
f(0.1,1.2) &\approx  0.4
\end{align*}

(b)
The point $(a,b,c)$ is on the surface $z=f(x,y)$ if and only if
\begin{equation*}
c = f(a,b) = \ln(4a^2 + b^2) 
\end{equation*}
Note that this forces $4a^2 + b^2$ to be nonzero.
The tangent plane to the surface $z = f (x, y)$ at the point $(a,b,c)$ 
is parallel to the plane $2x + 2y - z = 3$ if and only if 
$\llt 2\,,\,2\,,\,-1\rgt$ is a normal vector for the tangent plane.
That is, there is a nonzero number $t$ such that
\begin{align*}
\llt 2\,,\,2\,,\,-1\rgt
=t\llt f_x(a,b)\,,\,f_y(a,b)\,,\ -1\rgt
=t\llt \frac{8a}{4a^2+b^2} \,,\, \frac{2b}{4a^2+b^2} \,,\,-1\rgt
\end{align*}
For the $z$--coordinates to be equal, $t$ must be $1$.
Then, for the $x$-- and $y$--coordinates to be equal, we need
\begin{align*}
\frac{8a}{4a^2+b^2} &= 2 \\
\frac{2b}{4a^2+b^2} &= 2
\end{align*}
Note that these equations force both $a$ and $b$ to be nonzero.
Dividing these equations gives $\frac{8a}{2b}=1$ and hence $b=4a$.
Substituting $b=4a$ into either of the two equations gives
\begin{align*}
\frac{8a}{20a^2}=2 \implies a=\frac{1}{5}
\end{align*}
So $a=\frac{1}{5}$, $b=\frac{4}{5}$ and 
\begin{align*}
c=\ln\left(\frac{4}{5^2}+\frac{4^2}{5^2}\right)
 =\ln\frac{4}{5}
\end{align*}
\end{solution}

\begin{question}[M200 2013D] %1b
\begin{enumerate}[(a)]
\item 
Find the equation of the tangent plane to the surface 
$x^2 z^3 + y \sin(\pi x) = -y^2$ at the point $P = (1,1,-1)$.
\item
Let $z$ be defined implicitly by $x^2 z^3 + y \sin(\pi x) = -y^2$. 
Find $\pdiff{z}{x}$ at the point  $P = (1,1,-1)$.
\item
Let $z$ be the same implicit function as in part (ii), defined by 
the equation $x^2 z^3 + y \sin(\pi x) = -y^2$. Let $x = 0.97$, and $y = 1$. 
Find the approximate value of $z$.
\end{enumerate}
\end{question}

%\begin{hint}
%
%\end{hint}

\begin{answer}
(a)  $-(2+\pi)x +2y +3z = -\pi-3$

(b)  $\pdiff{z}{x}(1,1) = \frac{\pi+2}{3}$

(c)  $z(0.97,1) \approx -\frac{\pi+102}{100}$
\end{answer}

\begin{solution}
(a) The surface has equation $G(x,y,z) = x^2 z^3 + y \sin(\pi x) + y^2 =0$.
So a normal vector to the surface at $(1,1-1)$ is
\begin{align*}
\vnabla G(1,1,-1) 
&= \big[ \big(2x z^3 +\pi y\cos(\pi x)\big)\hi +\big(\sin(\pi x) +2y\big)\hj
           +3z^2 x^2\,\hk\big]_{(x,y,z)=(1,1,-1)} \\
&= \big(-2-\pi\big)\hi +2\,\hj +3\,\hk
\end{align*}
So the equation of the tangent plane is
\begin{align*}
\big(-2-\pi\big)(x-1) + 2(y-1) +3 (z+1) =0\quad\text{or}\quad
-(2+\pi)x +2y +3z = -\pi-3
\end{align*}


(b) The functions $z(x,y)$ obeys
\begin{align*}
x^2 z(x,y)^3 +y\sin(\pi x) +y^2 = 0
\end{align*}
for all $x$ and $y$. Differentiating this equation with respect to $x$ gives
\begin{align*}
2x z(x,y)^3 + 3 x^2 z(x,y)^2 \pdiff{z}{x}(x,y) + \pi y \cos(\pi x) =0
\end{align*}
Evaluating at $(1,1,-1)$ gives
\begin{align*}
-2 + 3 \pdiff{z}{x}(1,1) -\pi =0
\implies \pdiff{z}{x}(1,1) = \frac{\pi+2}{3}
\end{align*}

(c) Using the linear approximation about $(x,y)=(1,1)$,
\begin{align*}
z(x,1) \approx z(1,1) +\pdiff{z}{x}(1,1)\ (x-1)
\end{align*}
gives
\begin{align*}
z(0.97,1) \approx -1 +\frac{\pi+2}{3}\ (-0.03)
          = -1 -\frac{\pi+2}{100}
          =  -\frac{\pi+102}{100}
\end{align*}
\end{solution}

%%%%%%%%%%%%%%%%%%%%%%%%%%%%%%%%%%%%%%%%%
\begin{question} [M200 2001A] % 3
The surface $x^4+y^4+z^4+xyz=17$ passes through $(0,1,2)$,
and near this point the surface determines $x$ as a function, $x=F(y,z)$,
of $y$ and $z$.
\begin{enumerate}[(a)]
\item
Find $F_y$ and $F_z$ at $(x,y,z)=(0,1,2)$.

\item 
Use the tangent plane approximation (also known as linear,
first order or differential approximation) to find the approximate value
of $x$ (near $0$) such that $(x,1.01, 1.98)$ lies on the surface.
\end{enumerate}
\end{question}

%\begin{hint}
%
%\end{hint}

\begin{answer}
(a) $F_y(1,2)=-2$,\ $F_z(1,2)=-16$\qquad
(b) $0.3$
\end{answer}

\begin{solution}
(a) The function $F(y,z)$ obeys $F(y,z)^4+y^4+z^4+F(y,z)yz=17$ for all $y$ and $z$ near $y=1$, $z=2$.
Applying the derivatives $\pdiff{}{y}$ and $\pdiff{}{z}$ 
to this equation gives
\begin{align*}
4F(y,z)^3F_y(y,z)+4y^3+F_y(y,z)yz +F(y,z)z&=0 \\
4F(y,z)^3F_z(y,z)+4z^3+F_z(y,z)yz +F(y,z)y&=0
\end{align*}
Substiututing $F(1,2)=0$, $y=1$ and $z=2$ gives
\begin{alignat*}{3}
4+2F_y(1,2)&=0 & &\ \implies\  & &F_y(1,2)=-2 \\
32+2F_z(1,2)&=0 & &\ \implies\  & & F_z(1,2)=-16
\end{alignat*}

(b) Using the tangent plane to $x=F(y,z)$ at $y=1$ and $z=2$, which is
\begin{equation*}
x \approx F(1,2) +F_y(1,2)\,(y-1) +F_z(1,2)\,(z-2)
\end{equation*}
with $y=1.01$ and $z=1.98$ gives
\begin{align*}
x=F(1.01, 1.98)
&\approx F(1,2)+F_y(1,2)(1.01-1)+F_z(1,2)(1.98-2) \\
&=0-2(.01)-16(-0.02)
=0.3
\end{align*}
\end{solution}








