%\documentclass[12pt]{article}

\questionheader{ex:s2.3}


%%%%%%%%%%%%%%%%%%
\subsection*{\Conceptual}
%%%%%%%%%%%%%%%%%%

%%%%%%%%%%%%%%%%%%%%%%%%%%%%
%\Instructions{Questions~\ref{prob_s1.0first} through \ref{prob_s1.0last} provide practice with.}
%%%%%%%%%%%%%%%%%%%%

%%%%%%%%%%%%%%%%%%%%%%%%%%%%%%%%%%%%%%%%%%%%%%%%%%%%%%%
\begin{question}
Let all of the third order partial derivatives of the function 
$f(x,y,z)$ exist and be continuous. Show that
\begin{equation*}
f_{xyz}(x,y,z)
=f_{xzy}(x,y,z)
=f_{yxz}(x,y,z)
=f_{yzx}(x,y,z)
=f_{zxy}(x,y,z)
=f_{zyx}(x,y,z)
\end{equation*}
\end{question}

\begin{hint}
Repeatedly use (Clairaut's) Theorem \eref{CLP200}{thm mixed partials}
in the CLP-3 text. 
\end{hint}

\begin{answer}
See the solution.
\end{answer}

\begin{solution}
We have to derive a bunch of equalities.
\begin{itemize}
\item
Fix any real number $x$ and set $g(y,z)=f_x(x,y,z)$. By 
(Clairaut's) Theorem \eref{CLP200}{thm mixed partials} in the CLP-3 text
$g_{yz}(y,z)=g_{zy}(y,z)$, so
\begin{equation*}
f_{xyz}(x,y,z) = g_{yz}(y,z) =g_{zy}(y,z) = f_{xzy}(x,y,z)
\end{equation*}
\item
For every fixed real number $z$, 
(Clairaut's) Theorem \eref{CLP200}{thm mixed partials} in the CLP-3 text
gives $f_{xy}(x,y,z)=f_{yx}(x,y,z)$. So
\begin{equation*}
f_{xyz}(x,y,z) = \pdiff{}{z} f_{xy}(x,y,z)= \pdiff{}{z} f_{yx}(x,y,z)
=f_{yxz}(x,y,z)
\end{equation*}
So far, we have
\begin{equation*}
f_{xyz}(x,y,z) = f_{xzy}(x,y,z)=f_{yxz}(x,y,z)
\end{equation*}
\item
Fix any real number $y$ and set $g(x,z)=f_y(x,y,z)$. By 
(Clairaut's) Theorem \eref{CLP200}{thm mixed partials} in the CLP-3 text
$g_{xz}(x,z)=g_{zx}(x,z)$. So
\begin{equation*}
f_{yxz}(x,y,z) = g_{xz}(x,z) =g_{zx}(x,z) = f_{yzx}(x,y,z)
\end{equation*}
So far, we have
\begin{equation*}
f_{xyz}(x,y,z) = f_{xzy}(x,y,z)=f_{yxz}(x,y,z)= f_{yzx}(x,y,z)
\end{equation*}
\item
For every fixed real number $y$, 
(Clairaut's) Theorem \eref{CLP200}{thm mixed partials} in the CLP-3 text
gives $f_{xz}(x,y,z)=f_{zx}(x,y,z)$. So
\begin{equation*}
f_{xzy}(x,y,z) = \pdiff{}{y} f_{xz}(x,y,z)= \pdiff{}{y} f_{zx}(x,y,z)
=f_{zxy}(x,y,z)
\end{equation*}
So far, we have
\begin{equation*}
f_{xyz}(x,y,z) = f_{xzy}(x,y,z)=f_{yxz}(x,y,z)= f_{yzx}(x,y,z)=f_{zxy}(x,y,z)
\end{equation*}
\item
Fix any real number $z$ and set $g(x,y)=f_z(x,y,z)$. By 
(Clairaut's) Theorem \eref{CLP200}{thm mixed partials} in the CLP-3 text
$g_{xy}(x,y)=g_{yx}(x,y)$. So
\begin{equation*}
f_{zxy}(x,y,z) = g_{xy}(x,y) =g_{yx}(x,y) = f_{zxy}(x,y,z)
\end{equation*}
We now have all of 
\begin{equation*}
f_{xyz}(x,y,z) = f_{xzy}(x,y,z)=f_{yxz}(x,y,z)= f_{yzx}(x,y,z)=f_{zxy}(x,y,z)
= f_{zxy}(x,y,z)
\end{equation*}
\end{itemize}
\end{solution}

%%%%%%%%%%%%%%%%%%%%%%%%%%%%%%%%%%%%%%%%%%%%%%%%%%%%%%%%%%
\begin{question}
Find, if possible, a function $f(x,y)$ for which $f_x(x,y)=e^y$ and
$f_y(x,y)=e^x$.
\end{question}

\begin{hint}
If $f(x,y)$ obeying the specified conditions exists, then it is necessary
that $f_{xy}(x,y)=f_{yx}(x,y)$.
\end{hint}

\begin{answer}
No such $f(x,y)$ exists.
\end{answer}

\begin{solution}
No such $f(x,y)$ exists, because if it were to exist, then we would have
that $f_{xy}(x,y)=f_{yx}(x,y)$. But
\begin{align*}
f_{xy}(x,y)&=\pdiff{}{y}f_x(x,y)=\pdiff{}{y}e^y=e^y \\
f_{yx}(x,y)&=\pdiff{}{x}f_y(x,y)=\pdiff{}{x}e^x=e^x 
\end{align*}
are not equal.
\end{solution}


%%%%%%%%%%%%%%%%%%%%%%%%%%%%%%%%

%%%%%%%%%%%%%%%%%%
\subsection*{\Procedural}
%%%%%%%%%%%%%%%%%%

%%%%%%%%%%%%%%%%%%%%%%%%%%%%%%%%
\begin{question}
Find the specified partial derivatives.
\begin{enumerate}[(a)]
\item
$f(x,y) = x^2y^3$;
$f_{xx}(x,y)$, $f_{xyy}(x,y)$, $f_{yxy}(x,y)$
\item 
$f(x,y) = e^{xy^2}$;
$f_{xx}(x,y)$, $f_{xy}(x,y)$, $f_{xxy}(x,y)$, $f_{xyy}(x,y)$
\item
$\displaystyle f(u,v,w) = \frac{1}{u+2v+3w}\ $, 
   $\displaystyle \frac{\partial^3 f}{\partial u\partial v\partial w}(u,v,w)\ $,
   $\displaystyle \frac{\partial^3 f}{\partial u\partial v\partial w}(3,2,1)$


\end{enumerate}
\end{question}

%\begin{hint}
%\end{hint}

\begin{answer}
(a) $f_{xx}(x,y) = 2y^3$\qquad
    $f_{yxy}(x,y) = f_{xyy}(x,y) = 12xy$

(b) $f_{xx}(x,y)= y^4e^{xy^2}$\quad
    $f_{xy}(x,y)= \big(2y+2xy^3\big)e^{xy^2}$\quad
    $f_{xxy}(x,y)= \big(4y^3 + 2xy^5\big)e^{xy^2}$\quad
    $f_{xyy}(x,y) = \big(2+10xy^2+4x^2y^4\big)e^{xy^2}$

(c) $\displaystyle\frac{\partial^3 f}{\partial u\,\partial v\,\partial w}(u,v,w)
        = -\frac{36}{(u+2v+3w)^4}$\qquad
    $\displaystyle\frac{\partial^3 f}{\partial u\,\partial v\,\partial w}(3,2,1)
         = -0.0036
         = -\frac{9}{2500}$

\end{answer}

\begin{solution}
(a) We have
\begin{align*}
f_x(x,y) &= 2xy^3 &
f_{xx}(x,y) &= 2y^3 \\
 & &
f_{xy}(x,y) &= 6xy^2 &
f_{yxy}(x,y) = f_{xyy}(x,y) &= 12xy
\end{align*}

(b) We have
\begin{align*}
f_x(x,y) &= y^2e^{xy^2} &
f_{xx}(x,y) &= y^4e^{xy^2} &
f_{xxy}(x,y) &= 4y^3e^{xy^2} + 2xy^5e^{xy^2}
\\
 & &
f_{xy}(x,y) &= 2ye^{xy^2}+2xy^3e^{xy^2} &
f_{xyy}(x,y) &= \big(2+4xy^2+6xy^2+4x^2y^4\big)e^{xy^2}\\
 & &
 & &
 &= \big(2+10xy^2+4x^2y^4\big)e^{xy^2}
\end{align*}

(c) We have
\begin{align*}
\pdiff{f}{u}(u,v,w) &= -\frac{1}{(u+2v+3w)^2} \\
\frac{\partial^2 f}{\partial u\,\partial v}(u,v,w) &= \frac{4}{(u+2v+3w)^3} \\
\frac{\partial^3 f}{\partial u\,\partial v\,\partial w}(u,v,w)
        &= -\frac{36}{(u+2v+3w)^4} 
\end{align*}
In particular
\begin{align*}
\frac{\partial^3 f}{\partial u\,\partial v\,\partial w}(3,2,1)
        &= -\frac{36}{(3+2\times 2+3\times 1)^4} 
         = -\frac{36}{10^4}
         = -\frac{9}{2500}
\end{align*}

\end{solution}


%%%%%%%%%%%%%%%%%%%%%%%%%%%%%%%%
\begin{question}
Find all second partial derivatives of $f(x,y)=\sqrt{x^2+5y^2}$.
\end{question}

%\begin{hint}
%\end{hint}

\begin{answer}
$f_{xx}=\frac{5y^2}{(x^2+5y^2)^{3/2}}$\quad
$f_{xy}=f_{yx}=-\frac{5xy}{(x^2+5y^2)^{3/2}}$\quad
$f_{yy}=\frac{5x^2}{(x^2+5y^2)^{3/2}}$
\end{answer}

\begin{solution}
Let $f(x,y)=\sqrt{x^2+5y^2}$. Then
\begin{align*}
f_x&=\frac{x}{\sqrt{x^2+5y^2}} &
f_{xx}&=\frac{1}{\sqrt{x^2+5y^2}}-\frac{1}{2}\frac{(x)(2x)}{(x^2+5y^2)^{3/2}} &
f_{xy}&=-\frac{1}{2}\frac{(x)(10y)}{(x^2+5y^2)^{3/2}}
\cr
f_y&=\frac{5y}{\sqrt{x^2+5y^2}} &
f_{yy}&=\frac{5}{\sqrt{x^2+5y^2}}-\frac{1}{2}\frac{(5y)(10y)}{(x^2+5y^2)^{3/2}}&
f_{yx}&=-\frac{1}{2}\frac{(5y)(2x)}{(x^2+5y^2)^{3/2}}
\end{align*}
Simplifying, and in particular using that $\frac{1}{\sqrt{x^2+5y^2}}
=\frac{x^2+5y^2}{(x^2+5y^2)^{3/2}}$,
\begin{equation*}
f_{xx}=\frac{5y^2}{(x^2+5y^2)^{3/2}}\qquad
f_{xy}=f_{yx}=-\frac{5xy}{(x^2+5y^2)^{3/2}}\qquad
f_{yy}=\frac{5x^2}{(x^2+5y^2)^{3/2}} 
\end{equation*}
\end{solution}


%%%%%%%%%%%%%%%%%%%%%%%%%%%%%%%%
\begin{question}
Find the specified partial derivatives.
\begin{enumerate}[(a)]
\item
$f(x,y,z) = \arctan\big(e^{\sqrt{xy}}\big)$; $f_{xyz}(x,y,z)$
\item 
$f(x,y,z) = \arctan\big(e^{\sqrt{xy}}\big)
            +\arctan\big(e^{\sqrt{xz}}\big)
            +\arctan\big(e^{\sqrt{yz}}\big)$; $f_{xyz}(x,y,z)$
\item
$f(x,y,z) = \arctan\big(e^{\sqrt{xyz}}\big)$; $f_{xx}(1,0,0)$
\end{enumerate}
\end{question}

\begin{hint}
(a) This higher order partial derivative can be evaluated extremely efficiently
by carefully choosing the order of evaluation of the derivatives.

(b) This higher order partial derivative can be evaluated extremely efficiently
by carefully choosing a \emph{different order of evaluation} of the derivatives
for each of the three terms.

(c) Set $g(x) = f(x,0,0)$. Then $f_{xx}(1,0,0)=g''(1)$.
\end{hint}

\begin{answer}
(a) $f_{xyz}(x,y,z)=0$\qquad
(b) $f_{xyz}(x,y,z)=0$\qquad
(c) $f_{xx}(1,0,0)=0$
\end{answer}

\begin{solution}
(a) 
As $f(x,y,z) = \arctan\big(e^{\sqrt{xy}}\big)$ is independent of $z$,
we have $f_z(x,y,z) = 0$ and hence
\begin{equation*}
f_{xyz}(x,y,z)
=f_{zxy}(x,y,z)
=0
\end{equation*}

(b) Write $u(x,y,z) = \arctan\big(e^{\sqrt{xy}}\big)$,
          $v(x,y,z) = \arctan\big(e^{\sqrt{xz}}\big)$ and
          $w(x,y,z) = \arctan\big(e^{\sqrt{yz}}\big)$.
Then
\begin{itemize}
\item
As $u(x,y,z) = \arctan\big(e^{\sqrt{xy}}\big)$ is independent of $z$,
we have $u_z(x,y,z) = 0$ and hence
$
u_{xyz}(x,y,z)
=u_{zxy}(x,y,z)
=0
$
\item
As $v(x,y,z) = \arctan\big(e^{\sqrt{xz}}\big)$ is independent of $y$,
we have $v_y(x,y,z) = 0$ and hence
$
v_{xyz}(x,y,z)
=v_{yxz}(x,y,z)
=0
$
\item
As $w(x,y,z) = \arctan\big(e^{\sqrt{yz}}\big)$ is independent of $x$,
we have $w_x(x,y,z) = 0$ and hence
$
w_{xyz}(x,y,z)
=0
$
\end{itemize}
As $f(x,y,z)=u(x,y,z)+v(x,y,z)+w(x,y,z)$, we have
\begin{equation*}
f_{xyz}(x,y,z)=u_{xyz}(x,y,z)+v_{xyz}(x,y,z)+w_{xyz}(x,y,z)=0
\end{equation*}

(c) In the course of evaluating $f_{xx}(x,0,0)$, both $y$ and $z$ are held fixed at $0$. Thus, if we set $g(x) = f(x,0,0)$, then  $f_{xx}(x,0,0)=g''(x)$.
Now
\begin{equation*}
g(x) = f(x,0,0) = \arctan\big(e^{\sqrt{xyz}}\big)\Big|_{y=z=0}
                =\arctan(1)
                =\frac{\pi}{4}
\end{equation*}
for all $x$. So $g'(x)=0$ and $g''(x)=0$ for all $x$. In particular,
\begin{equation*}
f_{xx}(1,0,0) = g''(1) = 0
\end{equation*}

\end{solution}



%%%%%%%%%%%%%%%%%%%%%%%%%%%%%%%%
\begin{question}[M200 2002A] %2
 Let $f(r,\theta)=r^m\cos m\theta$ be a function of $r$ and $\theta$,
where $m$ is a positive integer.
\begin{enumerate}[(a)]
\item
 Find the second order partial derivatives $f_{rr}$, $f_{r\theta}$,
$f_{\theta\theta}$ and evaluate their respective values at $(r,\theta)=(1,0)$.

\item
 Determine the value of the real number $\la$ so that $f(r,\theta)$
satisfies the differential equation 
\begin{equation*}
f_{rr}+\frac{\la}{r}f_r+\frac{1}{r^2}f_{\theta\theta}=0
\end{equation*}
\end{enumerate}
\end{question}

%\begin{hint}
%\end{hint}

\begin{answer}
(a) $f_{rr}(1,0)=m(m-1),\ 
f_{r\theta}(1,0)=0,\ 
f_{\theta\theta}(1,0)=-m^2$\qquad
(b) $\la=1$
\end{answer}

\begin{solution}
(a)
The first order derivatives are
\begin{equation*}
f_r(r,\theta)=mr^{m-1}\cos m\theta\qquad
f_\theta(r,\theta)=-mr^m\sin m\theta
\end{equation*}
The second order derivatives are
\begin{equation*}
f_{rr}(r,\theta)=m(m-1)r^{m-2}\cos m\theta\quad
f_{r\theta}(r,\theta)=-m^2r^{m-1}\sin m\theta\quad
f_{\theta\theta}(r,\theta)=-m^2r^m\cos m\theta
\end{equation*}
so that
\begin{equation*}
f_{rr}(1,0)=m(m-1),\ 
f_{r\theta}(1,0)=0,\ 
f_{\theta\theta}(1,0)=-m^2
\end{equation*}

(b) By part (a), the expression
\begin{equation*}
f_{rr}+\frac{\la}{r}f_r+\frac{1}{r^2}f_{\theta\theta}
=m(m-1)r^{m-2}\cos m\theta+\la mr^{m-2}\cos m\theta-m^2r^{m-2}\cos m\theta
\end{equation*}
vanishes for all $r$ and $\theta$ if and only if 
\begin{equation*}
m(m-1)+\la m-m^2=0\iff m(\la-1)=0\iff \la=1
\end{equation*}
\end{solution}

%%%%%%%%%%%%%%%%%%
\subsection*{\Application}
%%%%%%%%%%%%%%%%%%

%%%%%%%%%%%%%%%%%%%%%%%%%%%%%%%%
\begin{question}
Let $\al>0$ be a constant.
Show that $\displaystyle u(x,y,z,t) =\frac{1}{t^{3/2}} 
             e^{-(x^2+y^2+z^2)/(4\al t)}$ satisfies the heat equation
\begin{equation*}
u_t =  \al\big(u_{xx} + u_{yy} + u_{zz} \big)
\end{equation*}
for all $t>0$
\end{question}

%\begin{hint}
%\end{hint}

\begin{answer}
See the solution.
\end{answer}

\begin{solution}
As
\begin{align*}
u_t(x,y,z,t)
&=-\frac{3}{2}\frac{1}{t^{5/2}} e^{-(x^2+y^2+z^2)/(4\al t)}
  +\frac{1}{4\al\,t^{7/2}}(x^2+y^2+z^2) e^{-(x^2+y^2+z^2)/(4\al t)} \\
u_x(x,y,z,t)
&=-\frac{x}{2\al\,t^{5/2}} e^{-(x^2+y^2+z^2)/(4\al t)} \\
u_{xx}(x,y,z,t)
&=-\frac{1}{2\al\,t^{5/2}} e^{-(x^2+y^2+z^2)/(4\al t)}
  +\frac{x^2}{4\al^2\,t^{7/2}} e^{-(x^2+y^2+z^2)/(4\al t)} \\
u_{yy}(x,y,z,t)
&=-\frac{1}{2\al\,t^{5/2}} e^{-(x^2+y^2+z^2)/(4\al t)}
  +\frac{y^2}{4\al^2\,t^{7/2}} e^{-(x^2+y^2+z^2)/(4\al t)} \\
u_{zz}(x,y,z,t)
&=-\frac{1}{2\al\,t^{5/2}} e^{-(x^2+y^2+z^2)/(4\al t)}
  +\frac{z^2}{4\al^2\,t^{7/2}} e^{-(x^2+y^2+z^2)/(4\al t)} 
\end{align*}
we have
\begin{align*}
\al\big(u_{xx} + u_{yy} + u_{zz} \big)
&=-\frac{3}{2\,t^{5/2}} e^{-(x^2+y^2+z^2)/(4\al t)}
  +\frac{x^2+y^2+z^2}{4\al\,t^{7/2}} e^{-(x^2+y^2+z^2)/(4\al t)}
=u_t 
\end{align*}

\end{solution}



