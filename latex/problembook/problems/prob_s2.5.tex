%\documentclass[12pt]{article}
\newcommand{\vt}{\mathbf{t}}

\questionheader{ex:s2.5}


%%%%%%%%%%%%%%%%%%
\subsection*{\Conceptual}
%%%%%%%%%%%%%%%%%%

%%%%%%%%%%%%%%%%%%%%%%%%%%%%
%\Instructions{Questions~\ref{prob_s1.0first} through \ref{prob_s1.0last} provide practice with.}
%%%%%%%%%%%%%%%%%%%%

%%%%%%%%%%%%%%%%%%%%%%%%%%%%%%%%
\begin{question}
Is it reasonable to say that the surfaces $x^2+y^2+(z-1)^2=1$ and
$x^2+y^2+(z+1)^2=1$ are tangent to each other at $(0,0,0)$?
\end{question}

\begin{hint}
What are the tangent planes to the two surfaces at $(0,0,0)$?
\end{hint}

\begin{answer}
Yes. The plane $z=0$ is the tangent plane to both surfaces at $(0,0,0)$.
\end{answer}

\begin{solution}
Write $F(x,y,z) = x^2+y^2+(z-1)^2-1$ and $G(x,y,z) = x^2+y^2+(z+1)^2-1$.
Let $S_1$ denote the surface $F(x,y,z)=0$ and $S_2$ denote the surface 
$G(x,y,z)=0$.
First note that $F(0,0,0)=G(0,0,0)=0$ so that the point $(0,0,0)$ lies
on both $S_1$ and $S_2$. The gradients of $F$ and $G$ are 
\begin{align*}
\vnabla F(x,y,z)
  &=\llt\pdiff{F}{x}(x,y,z)\,,\,
        \pdiff{F}{y}(x,y,z)\,,\,
        \pdiff{F}{z}(x,y,z)\rgt 
    =\llt 2x\,,\,2y\,,\,2(z-1)\rgt \\
\vnabla G(x,y,z)
  &=\llt\pdiff{G}{x}(x,y,z)\,,\,
        \pdiff{G}{y}(x,y,z)\,,\,
        \pdiff{G}{z}(x,y,z)\rgt 
    =\llt 2x\,,\,2y\,,\,2(z+1)\rgt 
\end{align*}
In particular,
\begin{equation*}
\vnabla F(0,0,0)=\llt 0,0,-2\rgt\qquad
\vnabla G(0,0,0)=\llt 0,0,2\rgt
\end{equation*}
so that the vector $\hk=-\frac{1}{2}\vnabla F(0,0,0)
                       =\frac{1}{2}\vnabla G(0,0,0)$
is normal to both surfaces at $(0,0,0)$. So the tangent plane to 
both $S_1$ and $S_2$ at $(0,0,0)$ is
\begin{equation*}
\hk\cdot\llt x-0,y-0,z-0\rgt=0\qquad\text{or}\qquad z=0
\end{equation*}
Denote by $P$ the plane $z=0$. 
Thus $S_1$ is tangent to $P$ at $(0,0,0)$ and $P$ is tangent to $S_2$ 
at $(0,0,0)$. So it is reasonable to say that $S_1$ and $S_2$ are tangent at
$(0,0,0)$.
\end{solution}


%%%%%%%%%%%%%%%%%%%%%%%%%%%%%
\begin{question}\label{tan_line_plane}
Let the point $\vr_0= (x_0,y_0,z_0)$ lie on the surface $G(x,y,z)=0$.
Assume that $\vnabla G(x_0,y_0,z_0)\ne\vZero$. Suppose that the 
parametrized curve $\vr(t)=\big(x(t),y(t),z(t)\big)$ is contained in the surface
and that $\vr(t_0)=\vr_0$. Show that the tangent line to the curve at $\vr_0$
lies in the tangent plane to $G=0$ at $\vr_0$.

\end{question}

\begin{hint}
Apply the chain rule to $G\big(\vr(t)\big)=0$.
\end{hint}

\begin{answer}
See the solution.
\end{answer}

\begin{solution} 
Denote by $S$ the surface $G(x,y,z)=0$ and by $C$ the parametrized curve 
$\vr(t)=\big(x(t),y(t),z(t)\big)$. To start, we'll find the tangent plane to $S$ at $\vr_0$ and the tangent line to $C$ at $\vr_0$.  
\begin{itemize}
\item
The tangent vector to $C$ at $\vr_0$ is 
$\llt x'(t_0)\,,\,y'(t_0)\,,\,z'(t_0) \rgt$, so the parametric equations for
the tangent line to $C$ at $\vr_0$ are
\begin{equation*}
x-x_0 = t x'(t_0)\qquad
y-y_0 = t y'(t_0)\qquad
z-z_0 = t z'(t_0)
\tag{$E_1$}\end{equation*}

\item
The gradient 
$\llt\pdiff{G}{x}\big( x_0\,,\,y_0\,,\,z_0\big)\,,\,
\pdiff{G}{y}\big( x_0\,,\,y_0\,,\,z_0\big)\,,\,
\pdiff{G}{z}\big( x_0\,,\,y_0\,,\,z_0\big)\rgt$ is a normal vector 
to the surface $S$ at $(x_0,y_0,z_0)$. So the tangent plane
to the surface $S$ at $(x_0,y_0,z_0)$ is
\begin{equation*}
\llt\pdiff{G}{x}\big( x_0\,,\,y_0\,,\,z_0\big)\,,\,
\pdiff{G}{y}\big( x_0\,,\,y_0\,,\,z_0\big)\,,\,
\pdiff{G}{z}\big( x_0\,,\,y_0\,,\,z_0\big)\rgt \cdot
\llt x-x_0\,,\, y-y_0\,,\,z-z_0\rgt = 0
\end{equation*}
or
\begin{equation*}
\pdiff{G}{x}\big( x_0\,,\,y_0\,,\,z_0\big)\ (x-x_0)
+\pdiff{G}{y}\big( x_0\,,\,y_0\,,\,z_0\big)\ (y-y_0)
+\pdiff{G}{z}\big( x_0\,,\,y_0\,,\,z_0\big)\ (z-z_0) = 0
\tag{$E_2$}\end{equation*}

\end{itemize}
Next, we'll show that the tangent vector  
$\llt x'(t_0)\,,\,y'(t_0)\,,\,z'(t_0) \rgt$ to $C$ at $\vr_0$ and the normal vector $\llt\pdiff{G}{x}\big( x_0\,,\,y_0\,,\,z_0\big)\,,\,
\pdiff{G}{y}\big( x_0\,,\,y_0\,,\,z_0\big)\,,\,
\pdiff{G}{z}\big( x_0\,,\,y_0\,,\,z_0\big)\rgt$ to $S$ at $\vr_0$ are 
perpendicular to each other. To do so, we observe that,
for every $t$, the point $\big(x(t),y(t),z(t)\big)$
lies on the surface $G(x,y,z)=0$ and so obeys
\begin{align*}
G\big(x(t),y(t),z(t)\big) =0
\end{align*}
Differentiating this equation with respect to $t$ gives,
by the chain rule,
\begin{align*}
0&= \diff{}{t}G\big(x(t),y(t),z(t)\big) \\
&=\pdiff{G}{x}\big( x(t)\,,\,y(t)\,,\,z(t)\big)\ x'(t)
+\pdiff{G}{y}\big( x(t)\,,\,y(t)\,,\,z(t)\big)\ y'(t)
+\pdiff{G}{z}\big( x(t)\,,\,y(t)\,,\,z(t)\big)\ z'(t)
\end{align*}
Then setting $t=t_0$ gives 
\begin{equation*}
\pdiff{G}{x}\big( x_0\,,\,y_0\,,\,z_0\big)\ x'(t_0)
+\pdiff{G}{y}\big( x_0\,,\,y_0\,,\,z_0\big)\ y'(t_0)
+\pdiff{G}{z}\big( x_0\,,\,y_0\,,\,z_0\big)\ z'(t_0) = 0
\tag{$E_3$}
\end{equation*}
Finally, we are in a position to show that if $(x,y,z)$ is any point on 
the tangent line to $C$ at $\vr_0$, then $(x,y,z)$ is also on the 
tangent plane to $S$ at $\vr_0$. As $(x,y,z)$ is on the tangent line to $C$ 
at $\vr_0$ then there is a $t$ such that, by $(E_1)$,
\begin{align*}
&\pdiff{G}{x}\big( x_0\,,\,y_0\,,\,z_0\big)\ \textcolor{blue}{\{x-x_0\}}
+\pdiff{G}{y}\big( x_0\,,\,y_0\,,\,z_0\big)\ \textcolor{blue}{\{y-y_0\}}
+\pdiff{G}{z}\big( x_0\,,\,y_0\,,\,z_0\big)\ \textcolor{blue}{\{z-z_0\}}
\\
&=\pdiff{G}{x}\big( x_0\,,\,y_0\,,\,z_0\big)\ 
               \textcolor{blue}{\big\{ t\,x'(t_0)\big\}}
+\pdiff{G}{y}\big( x_0\,,\,y_0\,,\,z_0\big)\ 
               \textcolor{blue}{\big\{ t\,y'(t_0)\big\}}
+\pdiff{G}{z}\big( x_0\,,\,y_0\,,\,z_0\big)\ 
               \textcolor{blue}{\big\{ t\,z'(t_0)\big\}}\\
&=\textcolor{blue}{t}\left[\pdiff{G}{x}\big( x_0\,,\,y_0\,,\,z_0\big)\ 
                \textcolor{blue}{x'(t_0)}
+\pdiff{G}{y}\big( x_0\,,\,y_0\,,\,z_0\big)\ 
                \textcolor{blue}{y'(t_0)}
+\pdiff{G}{z}\big( x_0\,,\,y_0\,,\,z_0\big)\ 
                 \textcolor{blue}{z'(t_0)} \right]
=0
\end{align*}
by $(E_3)$. That is, $(x,y,z)$ obeys the equation, $(E_2)$, of the tangent plane to $S$ at $\vr_0$ and so is on that tangent plane.  So the tangent 
line to $C$ at $\vr_0$ is contained in the tangent plane to $S$ at $\vr_0$.

\end{solution}

%%%%%%%%%%%%%%%%%%%%%%%%%%%%%%%%
\begin{question}
Let $F(x_0,y_0,z_0)=G(x_0,y_0,z_0)=0$ and let the vectors
$\vnabla F(x_0,y_0,z_0)$ and $\vnabla G(x_0,y_0,z_0)$ be nonzero and not
be parallel to each other. Find the equation of the normal plane to the 
curve of intersection of the surfaces $F(x,y,z)=0$ and $G(x,y,z)=0$ at
$(x_0,y_0,z_0)$. By definition, that normal plane is the plane through
$(x_0,y_0,z_0)$ whose normal vector is the tangent vector to the curve of
intersection at $(x_0,y_0,z_0)$. 
\end{question}

\begin{hint}
To find a tangent vector to the curve of intersection of the 
surfaces $F(x,y,z)=0$ and $G(x,y,z)=0$ at $(x_0,y_0,z_0)$,
use Q[\ref{tan_line_plane}] twice, once for the surface $F(x,y,z)=0$ and
once for the surface $G(x,y,z)=0$.
\end{hint}

\begin{answer}
The normal plane is $\vn\cdot\llt x-x_0\,,\,y-y_0\,,\,z-z_0\rgt =0$,
where the normal vector 
$\vn = \vnabla F(x_0,y_0,z_0)\times \vnabla G(x_0,y_0,z_0)$.
\end{answer}

\begin{solution}
Use $S_1$ to denote the surface $F(x,y,z)=0$, 
    $S_2$ to denote the surface $G(x,y,z)=0$ and 
    $C$ to denote the curve of intersection of $S_1$ and $S_2$.
\begin{itemize}
\item
Since $C$ is contained in $S_1$, the tangent line to $C$ at $(x_0,y_0,z_0)$
is contained in the tangent plane to $S_1$ at $(x_0,y_0,z_0)$, by 
Q[\ref{tan_line_plane}]. In particular, any tangent vector, $\vt$, to 
$C$ at $(x_0,y_0,z_0)$ must be perpendicular to $\vnabla F(x_0,y_0,z_0)$,
the normal vector to $S_1$ at $(x_0,y_0,z_0)$.

\item
Since $C$ is contained in $S_2$, the tangent line to $C$ at $(x_0,y_0,z_0)$
is contained in the tangent plane to $S_2$ at $(x_0,y_0,z_0)$, by 
Q[\ref{tan_line_plane}]. In particular, any tangent vector, $\vt$, to 
$C$ at $(x_0,y_0,z_0)$ must be perpendicular to $\vnabla G(x_0,y_0,z_0)$,
the normal vector to $S_2$ at $(x_0,y_0,z_0)$.
\end{itemize}
So any tangent vector to $C$ at $(x_0,y_0,z_0)$ must be perpendiular to both
$\vnabla F(x_0,y_0,z_0)$ and $\vnabla G(x_0,y_0,z_0)$.
One such tangent vector is
\begin{align*}
\vt = \vnabla F(x_0,y_0,z_0)\times \vnabla G(x_0,y_0,z_0)
\end{align*}
(Because the vectors $\vnabla F(x_0,y_0,z_0)$ and $\vnabla G(x_0,y_0,z_0)$
are nonzero and not parallel, $\vt$ is nonzero.) So the normal plane in 
question passes through $(x_0,y_0,z_0)$ and has normal vector $\vn=\vt$.
Consquently, the normal plane is
\begin{equation*}
\vn\cdot\llt x-x_0\,,\,y-y_0\,,\,z-z_0\rgt =0 \qquad\text{where }
\vn=\vt=\vnabla F(x_0,y_0,z_0)\times \vnabla G(x_0,y_0,z_0)
\end{equation*}


\end{solution}


%%%%%%%%%%%%%%%%%%%%%%%%%%%%%%%%
\begin{question}
Let $f(x_0,y_0)=g(x_0,y_0)$ and let 
$\llt f_x(x_0,y_0), f_y(x_0,y_0)\rgt\ne \llt g_x(x_0,y_0), g_y(x_0,y_0)\rgt$. Find the equation of the tangent line to the 
curve of intersection of the surfaces $z=f(x,y)$ and $z=g(x,y)$ at
$(x_0\,,\,y_0\,,\,z_0=f(x_0,y_0))$.
\end{question}

\begin{hint}
To find a tangent vector to the curve of intersection of the 
surfaces $z=f(x,y)$ and $z=g(x,y)$ at $(x_0,y_0,z_0)$,
use Q[\ref{tan_line_plane}] twice, once for the surface $z=f(x,y)$ and
once for the surface $z=g(x,y)$.
\end{hint}

\begin{answer}
Tangent line is
\begin{align*}
x&=x_0+t\big[g_y(x_0,y_0)-f_y(x_0,y_0)\big] \\
y&=y_0+t\big[f_x(x_0,y_0)-g_x(x_0,y_0)\big] \\
z&=z_0+ t\big[f_x(x_0,y_0)g_y(x_0,y_0)-f_y(x_0,y_0)g_x(x_0,y_0)\big]
\end{align*}
\end{answer}

\begin{solution}
Use $S_1$ to denote the surface $z=f(x,y)$, 
    $S_2$ to denote the surface $z=g(x,y)$ and 
    $C$ to denote the curve of intersection of $S_1$ and $S_2$.
\begin{itemize}
\item
Since $C$ is contained in $S_1$, the tangent line to $C$ at $(x_0,y_0,z_0)$
is contained in the tangent plane to $S_1$ at $(x_0,y_0,z_0)$, by 
Q[\ref{tan_line_plane}]. In particular, any tangent vector, $\vt$, to 
$C$ at $(x_0,y_0,z_0)$ must be perpendicular to $-f_x(x_0,y_0)\,\hi
-f_y(x_0,y_0)\,\hj+\hk$,
the normal vector to $S_1$ at $(x_0,y_0,z_0)$.
(See Theorem \eref{CLP200}{thm tan plane f} in the CLP-3 text.)

\item
Since $C$ is contained in $S_2$, the tangent line to $C$ at $(x_0,y_0,z_0)$
is contained in the tangent plane to $S_2$ at $(x_0,y_0,z_0)$, by 
Q[\ref{tan_line_plane}]. In particular, any tangent vector, $\vt$, to 
$C$ at $(x_0,y_0,z_0)$ must be perpendicular to $-g_x(x_0,y_0)\,\hi
-g_y(x_0,y_0)\,\hj+\hk$,
the normal vector to $S_2$ at $(x_0,y_0,z_0)$.
\end{itemize}
So any tangent vector to $C$ at $(x_0,y_0,z_0)$ must be perpendicular to
both of the vectors $-f_x(x_0,y_0)\,\hi-f_y(x_0,y_0)\,\hj+\hk$ and
$-g_x(x_0,y_0)\,\hi -g_y(x_0,y_0)\,\hj+\hk$.
One such tangent vector is
\begin{align*}
&\vt = \big[-f_x(x_0,y_0)\,\hi - f_y(x_0,y_0)\,\hj+\hk\big]\times 
       \big[-g_x(x_0,y_0)\,\hi - g_y(x_0,y_0)\,\hj+\hk\big] \\
    &=\det\left[\begin{matrix}
                     \hi &        \hj &   \hk \\
           -f_x(x_0,y_0) & -f_y(x_0,y_0) & 1 \\
           -g_x(x_0,y_0) & -g_y(x_0,y_0) & 1
                \end{matrix}\right] \\
    &=\llt g_y(x_0,y_0)-f_y(x_0,y_0)\,,\,
           f_x(x_0,y_0)-g_x(x_0,y_0)\,,\,
           f_x(x_0,y_0)g_y(x_0,y_0)-f_y(x_0,y_0)g_x(x_0,y_0)\rgt
\end{align*}
So the tangent line in question passes through $(x_0,y_0,z_0)$ and has 
direction vector $\vd=\vt$. Consquently, the tangent line is 
\begin{equation*}
\llt x-x_0\,,\,y-y_0\,,\,z-z_0\rgt = t\,\vd
\end{equation*}
or
\begin{align*}
x&=x_0+t\big[g_y(x_0,y_0)-f_y(x_0,y_0)\big] \\
y&=y_0+t\big[f_x(x_0,y_0)-g_x(x_0,y_0)\big] \\
z&=z_0+ t\big[f_x(x_0,y_0)g_y(x_0,y_0)-f_y(x_0,y_0)g_x(x_0,y_0)\big]
\end{align*}


\end{solution}




%%%%%%%%%%%%%%%%%%
\subsection*{\Procedural}
%%%%%%%%%%%%%%%%%%

%%%%%%%%%%%%%%%%%%%%%%%%%%%%%%%%
\begin{question}[M200 2009A] %1a
Let $\displaystyle f(x,y)=\frac{x^2y}{x^4+2y^2}$.
Find the tangent plane to the surface $z = f(x,y)$ at the point
$\left( -1\,,\,1\,,\,\frac{1}{3}\right)$.
\end{question}

%\begin{hint}
%
%\end{hint}

\begin{answer}
$2x+y+9z=2$
\end{answer}

\begin{solution}
We are going to use Theorem \eref{CLP200}{thm tan plane f} in the CLP-3 text.
To do so, we need the first order derivatives of $f(x,y)$ 
at $(x,y)=(-1,1)$. So we find them first.
\begin{alignat*}{3}
f_x(x,y)&=\frac{2xy}{x^4+2y^2}-\frac{x^2y(4x^3)}{{(x^4+2y^2)}^2}\qquad &
f_x(-1,1)&=-\frac{2}{3} +\frac{4}{3^2}=-\frac{2}{9}
\\
f_y(x,y)&=\frac{x^2}{x^4+2y^2}-\frac{x^2y(4y)}{{(x^4+2y^2)}^2}\qquad &
f_y(-1,1)&=\frac{1}{3} -\frac{4}{3^2}=-\frac{1}{9}
\end{alignat*}

The tangent plane is
\begin{align*}
z&=f(-1,1) + f_x(-1,1)\,(x+1) + f_y(-1,1)\,(y-1)
 =\frac{1}{3} -\frac{2}{9}\,(x+1) -\frac{1}{9}\,(y-1) \\
 &=\frac{2}{9}-\frac{2}{9}x-\frac{1}{9}y
\end{align*}
or $2x+y+9z=2$.
\end{solution}

%%%%%%%%%%%%%%%%%%%%%%%%%%%%%%%%
\begin{question}[M200 2015D] %1c
Find the tangent plane to
\begin{equation*}
\frac{27}{\sqrt{x^2+y^2+z^2+3}}=9
\end{equation*}
at the point $(2, 1, 1)$.
\end{question}

%\begin{hint}
%
%\end{hint}

\begin{answer}
$2x+y+z = 6$
\end{answer}

\begin{solution}
The equation of the given surface is of the form $G(x,y,z)=9$
with $G(x,y,z) =\frac{27}{\sqrt{x^2+y^2+z^2+3}}$. So,
by Theorem \eref{CLP200}{thm tan plane G} in the CLP-3 text, a normal
vector to the surface at $(2,1,1)$ is
\begin{align*}
\vnabla G(2,1,1)
  &=-\frac{1}{2}\ \frac{27}{(x^2+y^2+z^2+3)^{3/2}}\big(2x\,,\,2y\,,\,2z\big)
                                          \bigg|_{(x,y,z)=(2,1,1)} \\
  &=-\llt 2\,,\,1\,,\,1\rgt
\end{align*}
and the equation of the tangent plane is
\begin{equation*}
-\llt 2\,,\,1\,,\,1\rgt\cdot \llt x-2\,,\,y-1\,,\,z-1\rgt=0\qquad\text{or}\qquad
2x+y+z = 6
\end{equation*}
\end{solution}

%%%%%%%%%%%%%%%%%%%%%%%%%%%%%%%%
\begin{question}
Find the equations of the tangent plane and the normal line to the graph
of the specified function at the specified point.
\begin{enumerate}[(a)]
\item
$f(x,y)=x^2-y^2$ at $(-2,1)$

\item
$f(x,y)=e^{xy}$ at $(2,0)$
\end{enumerate}
\end{question}

%\begin{hint}
%\end{hint}

\begin{answer}
(a)  The tangent plane is  $4x+2y+z=-3$
      and the normal line is  $\llt x,y,z\rgt=\llt -2,1,3\rgt+t\llt 4,2,1\rgt$.

(b) The tangent plane is  $2y-z=-1$
      and the normal line is  $\llt x,y,z\rgt=\llt 2,0,1\rgt+t\llt 0,2,-1\rgt$.


\end{answer}

\begin{solution}
(a)
The specified graph is $z=f(x,y)=x^2-y^2$ or $F(x,y,z)=x^2-y^2-z=0$. 
Observe that $f(-2,1)=3$. The vector 
\begin{align*}
\vnabla F(-2,1,3) 
&= \llt F_x(x,y,z),F_y(x,y,z),F_z(x,y,z)\rgt\Big|_{(x,y,z)=(-2,1,3)} \\
&= \llt 2x,-2y,-1\rgt\Big|_{(x,y,z)=(-2,1,3)} \\
&= \llt -4,-2,-1\rgt
\end{align*} 
is a normal vector to the graph at $(-2,1,3)$. 
So the tangent plane is
\begin{equation*}
-4(x+2)-2(y-1)-(z-3)=0\text{ or } 4x+2y+z=-3
\end{equation*}
and the normal line is 
\begin{equation*}
\llt x,y,z\rgt=\llt -2,1,3\rgt+t\llt 4,2,1\rgt
\end{equation*}

(b)
The specified graph is $z=f(x,y)=e^{xy}$ or $F(x,y,z)=e^{xy}-z=0$. 
Observe that $f(2,0)=1$. The vector 
\begin{align*}
\vnabla F(2,0,1) 
&= \llt F_x(x,y,z),F_y(x,y,z),F_z(x,y,z)\rgt\Big|_{(x,y,z)=(2,0,1)} \\
&= \llt ye^{xy},xe^{xy},-1\rgt\Big|_{(x,y,z)=(2,0,1)} \\
&= \llt 0,2,-1\rgt
\end{align*} 
is a normal vector to the graph at $(2,0,1)$. 
So the tangent plane is
\begin{equation*}
0(x-2)+2(y-0)-(z-1)=0\text{ or } 2y-z=-1
\end{equation*}
and the normal line is 
\begin{equation*}
\llt x,y,z\rgt=\llt 2,0,1\rgt+t\llt 0,2,-1\rgt
\end{equation*}
\end{solution}

%%%%%%%%%%%%%%%%%%%%%%%%%%%%%%%%
\begin{question}[M200 2005D] %4
Consider the surface $z = f(x,y)$ defined implicitly by the equation 
$xyz^2 + y^2 z^3 = 3 + x^2$. Use a 3--dimensional gradient vector 
to find the equation of the tangent plane to this surface at the point
$(-1, 1, 2)$. Write your answer in the form $z = ax + by + c$, where 
$a$, $b$ and $c$ are constants.
\end{question}

%\begin{hint}
%
%\end{hint}

\begin{answer}
$z = -\frac{3}{4} x- \frac{3}{2} y + \frac{11}{4}$
\end{answer}

\begin{solution}
We may use $G(x,y,z) = xyz^2 + y^2 z^3 - 3 - x^2 = 0$ as an equation for
the surface.  Note that $(-1,1,2)$ really is on the surface since
\begin{align*}
G(-1,1,2) = (-1)(1)(2)^2 + (1)^2 (2)^3 - 3 - (-1)^2 
          = -4 + 8 - 3 - 1
          =0
\end{align*}
By Theorem \eref{CLP200}{thm tan plane G} in the CLP-3 text, since
\begin{alignat*}{5}
G_x(x,y,z)&=yz^2 -2x \qquad & 
    G_x(-1,1,2)&=6  \\
G_y(x,y,z)&=xz^2 +2yz^3 \qquad & 
    G_y(-1,1,2)&=12  \\
G_z(x,y,z)&=2xyz+3y^2z^2 \qquad & 
    G_z(-1,1,2)&=8  
\end{alignat*}
one normal vector to the surface at $(-1,1,2)$ is 
 $\vnabla G(-1,1,2) = \llt 6\,,\,12\,,\,8\rgt$ and an equation
of the tangent plane to the surface at $(-1,1,2)$ is
\begin{align*}
\llt 6\,,\,12\,,\,8\rgt \cdot
     \llt x+1\,,\,y-1\,,\,z-2\rgt = 0\qquad\text{or}\qquad
6x+12 y+ 8z = 22
\end{align*}
or
\begin{equation*}
z = -\frac{3}{4} x- \frac{3}{2} y +\frac{11}{4}
\end{equation*}
\end{solution}

%%%%%%%%%%%%%%%%%%%%%%%%%%%%%%%%
\begin{question}[M200 2008D] %1
A surface is given by
\begin{equation*}
z = x^2 - 2xy + y^2 .
\end{equation*}

\begin{enumerate}[(a)]
\item
Find the equation of the tangent plane to the surface at $x = a$, $y = 2a$.

\item 
For what value of $a$ is the tangent plane parallel to the plane 
$x - y + z = 1$?
\end{enumerate}
\end{question}

%\begin{hint}
%
%\end{hint}

\begin{answer}
(a) $2ax -2ay +z = -a^2$\qquad
(b) $a=\frac{1}{2}$.
\end{answer}

\begin{solution}
(a)
The surface is $G(x,y,z)=z-x^2+2xy-y^2=0$. When $x=a$ and $y=2a$
 and $(x,y,z)$ is on the surface, we have $z= a^2-2(a)(2a) +(2a)^2=a^2$.
So, by Theorem \eref{CLP200}{thm tan plane G} in the CLP-3 text, 
a normal vector to this surface at $(a,2a,a^2)$ is
\begin{align*}
\vnabla G(a,2a,a^2) = \llt -2x+2y\,,\,2x-2y\,,\,1\rgt\Big|_{(x,y,z)=(a,2a,a^2)}
                    = \llt 2a\,,\,-2a\,,\,1\rgt
\end{align*}
and the equation of the tangent plane is 
\begin{align*}
\llt 2a\,,\,-2a\,,\,1\rgt\cdot\llt x-a\,,\,y-2a\,,\,z-a^2\rgt =0
\qquad\text{or}\qquad
2ax -2ay +z = -a^2
\end{align*}

(b) The two planes are parallel when their two normal vectors,
namely $\llt 2a\,,\,-2a\,,\,1\rgt$ and $\llt 1\,,\,-1\,,\,1\rgt$,
are parallel. This is the case if and only if $a=\frac{1}{2}$.
\end{solution}

%%%%%%%%%%%%%%%%%%%%%%%%%%%%%%%%
\begin{question}[M200 2010D] %1b
Find the tangent plane and normal line to the surface 
$z=f(x,y)=\frac{2y}{x^2+y^2}$ at $(x,y)=(-1,2)$.
\end{question}

%\begin{hint}
%
%\end{hint}

\begin{answer}
The tangent plane is $\frac{8}{25}x-\frac{6}{25}y-z=-\frac{8}{5}$.\\ 
  \null\hskip0.3in   The normal line is 
   $\llt x,y,z\rgt = \llt -1,2,\frac{4}{5}\rgt 
                  +t \llt \frac{8}{25}\,,\,-\frac{6}{25}\,,\,-1\rgt$.
\end{answer}

\begin{solution}
The first order partial derivatives of $f$ are
\begin{alignat*}{3}
f_x(x,y) & = -\frac{4xy}{{(x^2+y^2)}^2}\quad &
      f_x(-1,2) & = \frac{8}{25} \\
f_y(x,y) & = \frac{2}{x^2+y^2}-\frac{4y^2}{{(x^2+y^2)}^2}\quad &
      f_y(-1,2) & = \frac{2}{5}-\frac{16}{25}
                  =-\frac{6}{25} \\
\end{alignat*}
So, by Theorem \eref{CLP200}{thm tan plane f} in the CLP-3 text, 
a normal vector to the surface at $(x,y)=(-1,2)$ is
$\llt \frac{8}{25}\,,\,-\frac{6}{25}\,,\,-1\rgt$.
As $f(-1,2)= \frac{4}{5}$, the tangent plane is
\begin{align*}
\llt \frac{8}{25}\,,\,-\frac{6}{25}\,,\,-1\rgt\cdot\llt x+1\,,\,y-2\,,\,
         z -\frac{4}{5}\rgt=0\quad \text{or}\quad
\frac{8}{25}x-\frac{6}{25}y-z=-\frac{8}{5} % =-\frac{40}{25}
\end{align*} 
and the normal line is
\begin{align*}
\llt x,y,z\rgt = \llt -1,2,\frac{4}{5}\rgt 
                  +t \llt \frac{8}{25}\,,\,-\frac{6}{25}\,,\,-1\rgt
\end{align*}
\end{solution}

\begin{question}[M200 2013D] %1f
Find all the points on the surface $x^2 + 9y^2 + 4z^2 = 17$ 
where the tangent plane is parallel to the plane $x - 8z = 0$.
\end{question}

\begin{hint}
Let $(x,y,z)$ be a desired point. Then  
\begin{itemize}\itemsep1pt \parskip0pt \parsep0pt %\itemindent-15pt
\item 
$(x,y,z)$ must be on the surface and
\item
the normal vector to the surface at $(x,y,z)$ must be parallel to the
plane's normal vector.
\end{itemize}
\end{hint}

\begin{answer}
$\pm(1,0,-2)$
\end{answer}

\begin{solution}
A normal vector to the surface $x^2 + 9y^2 + 4z^2 = 17$
at the point $(x,y,z)$ is $\llt 2x\,,\, 18y\,,\,8z\rgt$. 
A normal vector to the plane $x - 8z = 0$ is $\llt 1\,,\,0\,,\,-8\rgt$.
So we want $\llt 2x\,,\, 18y\,,\,8z\rgt$ to be parallel to
$\llt 1\,,\,0\,,\,-8\rgt$, i.e. to be a nonzero constant times
$\llt 1\,,\,0\,,\,-8\rgt$. This is the case whenever $y=0$ and $z=-2x$
with $x\ne 0$. In addition, we want $(x,y,z)$ to lie on the surface
$x^2 + 9y^2 + 4z^2 = 17$. So we want $y=0$, $z=-2x$ and
\begin{align*}
17= x^2 + 9y^2 + 4z^2 =x^2 +4(-2x)^2=17x^2
\implies x=\pm 1
\end{align*} 
So the allowed points are $\pm(1,0,-2)$.
\end{solution}

\begin{question}[M200 2014D] %4
Let $S$ be the surface $z = x^2 + 2y^2 + 2y - 1$. Find all points 
$P (x_0,y_0,z_0)$ on $S$ with $x_0 \ne 0$ such that the normal line 
at $P$ contains the origin $(0,0,0)$.
\end{question}

\begin{hint}
First find a parametric equation for the normal line to $S$ at $(x_0,y_0,z_0)$.
Then the requirement that $(0,0,0)$ lies on that normal line gives 
three equations in the four unknowns $x_0,y_0,z_0$ and $t$. The requirement
that $(x_0,y_0,z_0)$ lies on $S$ gives a fourth equation. Solve this system of four equations.
\end{hint}


\begin{answer}
$\big(\frac{1}{\sqrt{2}}\,,\,-1\,,\,-\frac{1}{2}\big)$
and 
  $\big(-\frac{1}{\sqrt{2}}\,,\,-1\,,\,-\frac{1}{2}\big)$
\end{answer}
\begin{solution}
The equation of $S$ is of the form $G(x,y,z) = x^2 + 2y^2 + 2y-z = 1$.
So one normal vector to $S$ at the point $(x_0,y_0,z_0)$ is 
\begin{equation*}
\vnabla G(x_0,y_0,z_0)  = 2x_0\,\hi + (4y_0+2)\,\hj -\hk
\end{equation*}
and the normal line to $S$ at $(x_0,y_0,z_0)$ is
\begin{equation*}
(x,y,z) = (x_0,y_0,z_0) +t\llt 2x_0\,,\,4y_0+2\,,\, -1\rgt
\end{equation*}
For this normal line to pass through the origin, there must be a $t$
with
\begin{align*}
(0,0,0) = (x_0,y_0,z_0) +t\llt 2x_0\,,\,4y_0+2\,,\, -1\rgt
\end{align*}
or
\begin{align*}
x_0 + 2x_0\,t & =0 \tag{E1}\\
y_0 +(4y_0+2)t &=0 \tag{E2}\\
z_0 -t &=0 \tag{E3}
\end{align*}
Equation (E3) forces $t=z_0$. Substituting this into equations (E1) and (E2)
gives
\begin{align*}
x_0(1+2z_0) & =0 \tag{E1}\\
y_0 +(4y_0+2)z_0 &=0 \tag{E2}
\end{align*}
The question specifies that $x_0\ne 0$, so (E1) forces $z_0=-\frac{1}{2}$.
Substituting $z_0=-\frac{1}{2}$ into (E2) gives
\begin{equation*}
-y_0-1=0 \implies y_0=-1
\end{equation*}
Finally $x_0$ is determined by the requirement that $(x_0,y_0,z_0)$
must lie on $S$ and so must obey
\begin{equation*}
z_0 = x_0^2 + 2y_0^2 + 2y_0 - 1
\implies -\frac{1}{2} = x_0^2 + 2(-1)^2 +2(-1)-1
\implies x_0^2 = \frac{1}{2}
%\implies x_0 = \pm \frac{1}{\sqrt{2}}
\end{equation*}
So the allowed points $P$ are 
  $\big(\frac{1}{\sqrt{2}}\,,\,-1\,,\,-\frac{1}{2}\big)$
and 
  $\big(-\frac{1}{\sqrt{2}}\,,\,-1\,,\,-\frac{1}{2}\big)$.
\end{solution}



%%%%%%%%%%%%%%%%%%%%%%%%%%%%%%%%
\begin{question}[M226 2009D] %1b
Find all points on the hyperboloid $z^2=4x^2+y^2-1$
where the tangent plane is parallel to the plane $2x-y+z=0$.
\end{question}

\begin{hint}
Two (nonzero) vectors $\vv$ and $\vw$ are parallel if and only if there  
is a $t$ such that $\vv=t\,\vw$.
Don't forget that the point has to be on the hyperboloid.
\end{hint}

\begin{answer}
$\pm \big(\half,-1,-1\big)$
\end{answer}

\begin{solution} 
Let $(x_0,y_0,z_0)$  be a point on the hyperboloid $z^2=4x^2+y^2-1$
where the tangent plane is parallel to the plane $2x-y+z=0$. A normal vector
to the plane $2x-y+z=0$ is $\llt 2,-1,1\rgt$. Because the hyperboloid is
$G(x,y,z)=4x^2+y^2-z^2-1$ and $\vnabla G(x,y,z) = \llt 8x,2y,-2z\rgt$,
 a normal vector to the hyperboloid at $(x_0,y_0,z_0)$ is 
$\vnabla G(x_0,y_0,z_0)=\llt 8x_0,2y_0,-2z_0\rgt$. 
So $(x_0,y_0,z_0)$ satisfies the required conditions if and only if there is a nonzero $t$ obeying
\begin{align*}
&\llt 8x_0,2y_0,-2z_0\rgt =t\llt 2,-1,1\rgt \text{ and }
 z_0^2=4x_0^2+y_0^2-1\\
&\iff x_0=\frac{t}{4},\ y_0=z_0=-\frac{t}{2}\text{ and }
 z_0^2=4x_0^2+y_0^2-1\\
&\iff \frac{t^2}{4}= \frac{t^2}{4}+ \frac{t^2}{4}-1\text{ and }
     x_0=\frac{t}{4},\ y_0=z_0=-\frac{t}{2}\\
& \iff t=\pm 2\qquad
(x_0,y_0,z_0)=\pm \big(\half,-1,-1\big)
\end{align*}

\end{solution}

%%%%%%%%%%%%%%%%%%%%%%%%%%%%%%%%
\begin{question}
Find a vector of length $\sqrt{3}$ which is tangent to the curve of
intersection of the surfaces $\ z^2=4x^2+9y^2\ $ and $\ 6x+3y+2z=5\ $ at $\ (2,1,-5)$.
\end{question}

\begin{hint}
The curve lies in the surface $z^2=4x^2+9y^2$. So the tangent vector to the
curve is perpendicular to the normal vector to $z^2=4x^2+9y^2$ at $(2,1,-5)$.

The curve also lies in the surface $6x+3y+2z=5$. So the tangent vector to the
curve is also perpendicular to the normal vector to $6x+3y+2z=5$ at $(2,1,-5)$.
\end{hint}

\begin{answer}
$\pm\sqrt{3}\frac{\llt 3,14,-30\rgt}{|\llt 3,14,-30\rgt|}
=\pm\sqrt{\frac{3}{1105}}\llt 3,14,-30\rgt$
\end{answer}

\begin{solution}
One vector normal to the surface $F(x,y,z)=4x^2+9y^2-z^2=0$ at $(2,1,-5)$ is
\begin{align*}
\vnabla F(2,1,-5) = \llt 8x,18y,-2z\rgt\Big|_{(2,1,-5)}=\llt 16,18,10\rgt
\end{align*}
One vector normal to the surface $G(x,y,z)=6x+3y+2z=5$ at 
$(2,1,-5)$ is 
\begin{align*}
\vnabla G(2,1,-5) = \llt 6,3,2\rgt
\end{align*}
Now
\begin{itemize}
\item
The curve lies in the surface $z^2=4x^2+9y^2$. So the tangent vector to the
curve at $(2,1,-5)$ is perpendicular to the normal vector 
$\frac{1}{2}\llt 16,18,10\rgt=\llt 8,9,5\rgt$.
\item
The curve also lies in the surface $6x+3y+2z=5$. So the tangent vector to the
curve at $(2,1,-5)$ is also perpendicular to the normal vector $\llt 6,3,2\rgt$.
\item
So the tangent vector to the curve at $(2,1,-5)$ is parallel to
\begin{align*}
\llt 8,9,5\rgt\times \llt 6,3,2\rgt
=\det\left[\begin{matrix}
                     \hi & \hj & \hk \\
                     8   &  9  & 5 \\
                     6   &  3  & 2
                \end{matrix}\right]
=\llt 3,14,-30\rgt
\end{align*}
\end{itemize}
The desired vectors are
\begin{equation*}
\pm\sqrt{3}\frac{\llt 3,14,-30\rgt}{|\llt 3,14,-30\rgt|}
=\pm\sqrt{\frac{3}{1105}}\llt 3,14,-30\rgt
\end{equation*}
\end{solution}




%%%%%%%%%%%%%%%%%%
\subsection*{\Application}
%%%%%%%%%%%%%%%%%%

%%%%%%%%%%%%%%%%%%%%%%%%%%%%%%%%
\begin{question}
Find all horizontal planes that are tangent to the surface with equation
\begin{equation*}
z=xy e^{-(x^2+y^2)/2}
\end{equation*}
What are the largest and smallest values of $z$ on this surface?
\end{question}

\begin{hint}
At the highest and lowest points of the surface, the tangent plane is horizontal.
\end{hint}

\begin{answer}
The horizontal tangent planes are $z=0$, $z=e^{-1}$ and $z=-e^{-1}$.
The largest and smallest values of $z$ are $e^{-1}$ and $-e^{-1}$, respectively.
\end{answer}

\begin{solution}
Let $(x_0,y_0,z_0)$ be any point on the surface. A vector
normal to the surface at $(x_0,y_0,z_0)$ is
\begin{align*}
&\vnabla\Big(xy e^{-(x^2+y^2)/2}-z\Big)\bigg|_{(x_0,y_0,z_0)} 
\\&\hskip1in
=\llt y_0 e^{-(x_0^2+y_0^2)/2}-x_0^2y_0 e^{-(x_0^2+y_0^2)/2},
      x_0 e^{-(x_0^2+y_0^2)/2}-x_0y_0^2 e^{-(x_0^2+y_0^2)/2},-1\rgt
\end{align*}
The tangent plane to the surface at $(x_0,y_0,z_0)$ is horizontal
if and only if this vector is vertical, which is the case 
if and only if its $x$- and $y$-components are
zero, which in turn is the case if and only if
\begin{align*}
&y_0(1-x_0^2)=0\text{ and }x_0(1-y_0^2)=0\\
&\iff\big\{y_0=0\text{ or }x_0=1\text{ or }x_0=-1\big\}
   \text{ and }\big\{x_0=0\text{ or }y_0=1\text{ or }y_0=-1\big\}\\
&\iff (x_0,y_0)=(0,0)\text{ or }(1,1)\text{ or }(1,-1)
   \text{ or }(-1,1)\text{ or }(-1,-1)
\end{align*}
The values of $z_0$ at these points are $0$, $e^{-1}$, $-e^{-1}$, $-e^{-1}$ 
and $e^{-1}$, respectively. So the horizontal tangent planes are
$z=0$, $z=e^{-1}$ and $z=-e^{-1}$.
At the highest and lowest points of the surface, the tangent plane is horizontal.
So the largest and smallest values of $z$ are $e^{-1}$ and $-e^{-1}$, respectively.


\end{solution}

%%%%%%%%%%%%%%%%%%%%%%%%%%%%%%%%
\begin{question}[M200 2004A] %5
Let $S$ be the surface
\begin{equation*}
xy-2x+yz+x^2+y^2+z^3=7
\end{equation*}
\begin{enumerate}[(a)]
\item
Find the tangent plane and normal line to the surface $S$
at the point $(0,2,1)$.
\item  
The equation defining $S$ implicitly defines $z$ as a 
function of $x$ and $y$ for $(x,y,z)$ near $(0,2,1)$. Find expressions for 
$\pdiff{z}{x}$ and 
$\pdiff{z}{y}$. Evaluate
$\pdiff{z}{y}$ at $(x,y,z)=(0,2,1)$.
\item   
Find an expression for 
$\frac{\partial^2\hfil z\hfil\,}{\partial x\partial y}$. 
\end{enumerate}
\end{question}

%\begin{hint}
%
%\end{hint}

\begin{answer}
(a) $y+z=3$\qquad $\vr(t)=\llt 0,2,1\rgt+t\llt 0,5,5\rgt$

(b) $z_x(x,y)=\frac{2-2x-y}{y+3z(x,y)^2}$\qquad
    $z_y(x,y)=-\frac{x+2y+z(x,y)}{y+3z(x,y)^2}$\qquad
    $z_y(0,2)=-1$

(c) $z_{xy}(x,y)=\frac{1}{y+3z(x,y)^2}
-\frac{2-2x-y}{{[y+3z(x,y)^2]}^2}
\left(1-6z(x,y)\frac{x+2y+z(x,y)}{y+3z(x,y)^2}\right)$
\end{answer}

\begin{solution}
(a) 
 A normal vector to the surface at $(0,2,1)$ is
\begin{align*}
\vnabla\big(xy-2x+yz+x^2+y^2+z^3-7\big)\big|_{(0,2,1)}
&=\llt y-2+2x\,,\,x+z+2y\,,\,y+3z^2\rgt\big|_{(0,2,1)}\\
&=\llt 0, 5, 5\rgt
\end{align*}
So the tangent plane is 
\begin{align*}
0(x-0)+5(y-2)+5(z-1)=0\text{ or }y+z=3
\end{align*}
The vector parametric equations for the normal line are
\begin{equation*}
\vr(t)=\llt 0,2,1\rgt+t\llt 0,5,5\rgt
\end{equation*}

(b) Differentiating
\begin{equation*}
xy-2x+y\,z(x,y)+x^2+y^2+z(x,y)^3=7
\end{equation*}
gives
\begin{alignat*}{5}
y-2+y\,z_x(x,y)+2x+3z(x,y)^2z_x(x,y)&=0 &   &\implies &
z_x(x,y)&=\frac{2-2x-y}{y+3z(x,y)^2}\\
x+z(x,y)+y\,z_y(x,y)+2y+3z(x,y)^2z_y(x,y)&=0 &   &\implies &
z_y(x,y)&=-\frac{x+2y+z(x,y)}{y+3z(x,y)^2}
\end{alignat*}
In particular, at $(0,2,1)$, $z_y(0,2)=-\frac{4+1}{2+3}=-1$. 

(c)
Differentiating $z_x$ with respect to $y$ gives
\begin{align*}
z_{xy}(x,y)
&=-\frac{1}{y+3z(x,y)^2}
-\frac{2-2x-y}{{[y+3z(x,y)^2]}^2}\big(1+6z(x,y)z_y(x,y)\big)\\
&=-\frac{1}{y+3z(x,y)^2}
-\frac{2-2x-y}{{[y+3z(x,y)^2]}^2}
\left(1-6z(x,y)\frac{x+2y+z(x,y)}{y+3z(x,y)^2}\right)
\end{align*}
As an alternate solution, we could also  differentiate $z_y$ with respect 
to $x$. This gives
\begin{align*}
z_{yx}(x,y)
&=-\frac{1+z_x(x,y)}{y+3z(x,y)^2}
+\frac{x+2y+z(x,y)}{{[y+3z(x,y)^2]}^2}6z(x,y)z_x(x,y)\cr
&=-\frac{1}{y+3z(x,y)^2}\left(1+\frac{2-2x-y}{y+3z(x,y)^2}\right)
+\frac{x+2y+z(x,y)}{{[y+3z(x,y)^2]}^2}6z(x,y)\frac{2-2x-y}{y+3z(x,y)^2}
\end{align*}
\end{solution}


%%%%%%%%%%%%%%%%%%%%%%%%%%%%%%%%
\begin{question}[M200 2000D] %1
\begin{enumerate}[(a)]
\item
Find a vector perpendicular at the point
$(1,1,3)$ to the surface with equation $x^2+z^2=10$.

\item 
Find a vector tangent at the same point to the curve of 
intersection of the surface in part (a) with surface $y^2+z^2=10$.

\item 
Find parametric equations for the line tangent to that curve
at that point.
\end{enumerate}
\end{question}

\begin{hint}
(b) If $\vv$ is tangent, at a point $P$, to the curve of intersection of the
surfaces $S_1$ and $S_2$, then $\vv$ 
\begin{itemize}\itemsep1pt \parskip0pt \parsep0pt %\itemindent-15pt
\item
has to be tangent to $S_1$ at $P$, and so must be perpendicular to the 
normal vector to $S_1$ at $P$ and
\item
has to be tangent to $S_2$ at $P$, and so must be perpendicular to the 
normal vector to $S_2$ at $P$.
\end{itemize}
\end{hint}


\begin{answer}
(a) $\llt 1,0,3\rgt$\qquad
(b) $\llt 3,3,-1\rgt$\qquad
(c) $\vr(t)=\llt 1,1,3\rgt+t\llt 3,3,-1\rgt$
\end{answer}

\begin{solution}
(a) 
A vector perpendicular to $x^2+z^2=10$ at $(1,1,3)$ is
\begin{equation*}
\vnabla(x^2+z^2)\big|_{(1,1,3)}
=(2x\hi+2z\hk)\big|_{(1,1,3)}
=2\hi+6\hk\hbox{ or }
\frac{1}{2} \llt 2,0,6\rgt=\llt 1,0,3\rgt
\end{equation*}

(b) A vector perpendicular to $y^2+z^2=10$ at $(1,1,3)$ is
\begin{equation*}
\vnabla(y^2+z^2)\big|_{(1,1,3)}
=(2y\hj+2z\hk)\big|_{(1,1,3)}
=2\hj+6\hk\hbox{ or }\frac{1}{2} \llt 0,2,6\rgt=\llt 0,1,3\rgt
\end{equation*}
A vector is tangent to the specified curve at the specified point if and only
if it  perpendicular to both $(1,0,3)$ and $(0,1,3)$. One such vector is
\begin{equation*}
\llt 0,1,3\rgt\times\llt1,0,3\rgt
=\det\left[\begin{matrix}
                     \hi & \hj & \hk \\
                     0   &  1  & 3 \\
                     1   &  0  & 3
                \end{matrix}\right]
=\llt 3,3,-1\rgt
\end{equation*}

(c) The specified tangent line passes through $(1,1,3)$ and has direction
vector $\llt 1,1,3\rgt$ and so has vector parametric equation
$$
\vr(t)=\llt 1,1,3\rgt+t\llt 3,3,-1\rgt
$$
\end{solution}

%%%%%%%%%%%%%%%%%%%%%%%%%%%%%%%%
\begin{question}[M200 2000A] %2
Let $P$ be the point where the curve 
\begin{equation*}
\vr(t) = t^3\,\hi + t\,\hj + t^2\,\hk,\qquad (0 \le t <\infty)
\end{equation*}
 intersects the surface 
\begin{equation*}
z^3 + xyz -2 = 0
\end{equation*}
Find the (acute) angle between the curve and the surface at $P$. 
\end{question}

\begin{hint}
 The angle between the curve and the surface at $P$ is $90^\circ$
minus the angle between the curve and the normal vector to the surface at $P$.
\end{hint}

\begin{answer}
$49.11^\circ$ (to two decimal places)
\end{answer}

\begin{solution}
$\vr(t)=\llt x(t)\,,\,y(t)\,,\,z(t)\rgt$ intersects $z^3 + xyz -2 = 0$ when
\begin{equation*}
z(t)^3+x(t)\,y(t)\,z(t)-2=0\iff \big(t^2\big)^3 + \big(t^3)(t)\big(t^2\big)-2=0
\iff 2t^6=2\iff t=1
\end{equation*}
since $t$ is required to be positive.
The direction vector for the curve at $t=1$ is
\begin{equation*}
\vr'(1)=3\,\hi+\hj+2\,\hk
\end{equation*}
A normal vector for the surface at $\vr(1)=\llt 1,1,1\rgt$ is
\begin{equation*}
\vnabla(z^3+xyz)\big|_{(1,1,1)}=[yz\hi+xz\hj+(3z^2+xy)\hk]_{(1,1,1)}
=\hi+\hj+4\hk
\end{equation*}
The angle $\theta$ between the curve and the normal vector to the surface
is determined by
\begin{align*}
\big|\llt 3,1,2\rgt\big|\,\big|\llt 1,1,4\rgt\big|\cos\theta
             =\llt 3,1,2\rgt \cdot\llt 1,1,4\rgt
&\iff \sqrt{14}\sqrt{18}\cos\theta=12 \\
&\iff \sqrt{7\times 36}\cos\theta=12 \\
&\iff \cos\theta=\frac{2}{\sqrt{7}} \\
&\iff \theta=40.89^\circ
\end{align*}
The angle between the curve and the surface is 
$90-40.89=49.11^\circ$ (to two decimal places).
\end{solution}

%%%%%%%%%%%%%%%%%%%%%%%%%%%%%%%%
\begin{question}
Find the distance from the point $(1,1,0)$ to the circular
paraboloid with equation $z=x^2+y^2$.
\end{question}

\begin{hint}
Let $D(x,y)$ be the distance (or the square of the distance) from $(1,1,0)$ 
to the point $\big(x,y, x^2+y^2)$ on the paraboloid. We wish to 
minimize $D(x,y)$. That is, to find the lowest point on the graph 
$z=D(x,y)$. At this lowest point, the tangent plane to $z=D(x,y)$ is 
horizontal.
\end{hint}

\begin{answer}
$\frac{\sqrt{3}}{2}$
\end{answer}

\begin{solution}
Let $(x,y,z)$ be any point on the paraboloid $z=x^2+y^2$. The square of the
distance from $(1,1,0)$ to this point is 
\begin{align*}
D(x,y)&=(x-1)^2+(y-1)^2+z^2\\
&=(x-1)^2+(y-1)^2+{(x^2+y^2)}^2
\end{align*}
We wish to minimize $D(x,y)$. That is, to find the lowest point on the
graph $z=D(x,y)$. At this lowest point, the tangent plane to $z=D(x,y)$
is horizontal. So at the minimum, the normal vector to $z=D(x,y)$
has $x$ and $y$ components zero. So
\begin{alignat*}{5}
0&=\pdiff{D}{x}(x,y)&&= 2(x-1)+2(x^2+y^2)(2x)\\
0&=\pdiff{D}{y}(x,y)&&= 2(y-1)+2(x^2+y^2)(2y)
\end{alignat*}
By symmetry (or multiplying the first equation by $y$, multiplying the
second equation by $x$ and subtracting) the solution will have $x=y$
with
\begin{equation*}
0=2(x-1)+2(x^2+x^2)(2x)=8x^3+2x-2
\end{equation*}
Observe that the value of $8x^3+2x-2=2(4x^3+x-1)$ at $x=\frac{1}{2}$ is $0$.
(See Appendix \eref{CLP101}{ap:roots} of the CLP-2 text for some 
useful tricks that can help you guess roots of polynomials with 
integer coefficients.) So $\big(x-\frac{1}{2}\big)$ is a factor of
\begin{equation*}
4x^3+x-1
={\textstyle 4\big(x^3+\frac{x}{4}-\frac{1}{4}\big)
            =4\big(x-\frac{1}{2}\big)\big(x^2+\half x+\half\big)}
\end{equation*}
and the minimizing $(x,y)$ obeys $x=y$ and
\begin{equation*}
0=8x^3+2x-2
=8\big(x-\half\big)\big(x^2+\half x+\half\big)=0
\end{equation*}
By the quadratic root formula, $x^2+\half x+\half$ has no real roots, so
the only solution is $x=y=\half$, $z=\big(\half\big)^2+\big(\half\big)^2=\half$
and the distance is $\sqrt{\big(\half-1\big)^2+\big(\half-1\big)^2
+\big(\half\big)^2}=\frac{\sqrt{3}}{2}$.
\end{solution}
